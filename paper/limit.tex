\documentclass[12pt]{article}

\usepackage{amssymb,amsmath,amsthm}
\usepackage[top=1in, bottom=1in, left=1.25in, right=1.25in]{geometry}
\usepackage{enumerate,palatino,bm}
\usepackage[final]{graphicx}
\usepackage[colorlinks=true,citecolor=blue,linkcolor=red,urlcolor=blue]{hyperref}

\newtheorem{theorem}{Theorem}
\newtheorem{conjecture}{Conjecture}

\newcommand{\RR}{\ensuremath{\mathbb R}}
\newcommand{\ZZ}{\ensuremath{\mathbb Z}}

\newcommand{\bn}{\ensuremath{\mathbf{n}}}
\newcommand{\bu}{\ensuremath{\mathbf{u}}}

\newcommand{\bzero}{\ensuremath{\bm{0}}}

\newcommand{\cK}{\ensuremath{\mathcal{K}}}
\newcommand{\cX}{\ensuremath{\mathcal{X}}}

\newcommand{\grad}{\ensuremath{\nabla}}
\newcommand{\eps}{\ensuremath{\epsilon}}


\title{Bounds on the glacier surface elevation \\ in a regularized Stokes model}
\author{Ed Bueler}

\begin{document}
\maketitle

We consider the glacier geometry problem of determining the surface elevation when the dynamical model for the ice is a Stokes stress balance.  This time-dependent model is approximated by a single backward-Euler step of the regularized surface kinematical equation, but in variational inequality form, that is, as a weak-form free-boundary problem.  The main idea in this short note is that if we conjecture coercivity for the regularized surface motion then we can bound a Sobolev norm of the surface elevation solution in terms of the average climate and a norm of the bed elevation gradient.

The time-dependent model couples the surface kinematical equation to the Stokes stress balance.  This model is mathematically-formulated in reference \cite{Bueler2025}.  Suppose $\Omega\subset \RR^2$ is a bounded domain, with notation $x=(x_1,x_2)\in\Omega$ for a location in the map-plane.  We assume the climate data $a(t,x)$ is in $C([0,\infty); L^1(\Omega))$, and we fix a bed elevation function $b\in C^1(\bar\Omega)$.   Reference \cite{Bueler2025} proposes the following Banach space for surface elevations:\footnote{FIXME: with norm $\|s\|_{\cX}^4 = \int_\Omega |s|^4 + \int_\Omega |\grad s|^4$ in this note, but \cite{Bueler2025} has a correctly-scaled norm}
\begin{equation}
\cX = W^{1,4}(\Omega). \label{eq:defineX}
\end{equation}

From $b$ we define a closed and convex set of admissible surface elevations:
\begin{equation}
\cK=\{\sigma\in\cX\,:\,\sigma\ge b \text{ and } \sigma|_{\partial\Omega}=b|_{\partial\Omega}\} \label{eq:defineK}
\end{equation}
If $s\in \cK$ then we define $\bu=(u_1,u_2,w)$ to be the solution of the Stokes problem over the 3D domain
\begin{equation}
\Lambda(s) = \left\{(x,z)\,:\,b(x) < z < s(x)\right\} \subset \Omega \times \RR. \label{eq:domainfroms}
\end{equation}
The surface trace of the velocity is denoted $\bu|_s=(u_1|_s,u_2|_s,w|_s)$; \cite{Bueler2025} gives a bound on this trace in terms of $s$.  This surface velocity is extended by zero to all of $\Omega$, that is, it is zero outside of the portion of $\Omega$ which is covered by ice ($s>b$).

The surface kinematical equation \cite{GreveBlatter2009} for a glacier says that the surface velocity combines with the climate to determine the evolution of the surface elevation
\begin{equation}
\frac{\partial s}{\partial t} - \bu|_s \cdot \bn_s = a. \label{eq:ske}
\end{equation}
Here $\bn_s = \left<-\frac{\partial s}{\partial x_1},-\frac{\partial s}{\partial x_2},w\right>$ is an upward surface-normal vector.  However, this equation is only true on the portion of $\Omega$ where there is ice.  As explained in \cite{Bueler2025}, the full surface elevation solves a nonlinear complementarity problem over all of $\Omega$, and the weak form of this problem is a variational inequality \cite{KinderlehrerStampacchia1980}.

In the current work we only consider a regularized form of this model.  For $\eps>0$ and $\eta_0>0$, and physical constant $\Gamma>0$, for $s\in\cK$ and $\omega \in\cX$ define the SIA-regularized surface motion as
\begin{equation}
\Phi^\eps(s)[\omega] = \int_\Omega \Big(\big(u_1|_s,u_2|_s\big) \cdot \grad s - (1-\eps) w|_s\Big) \omega + \eps\, \Gamma \eta_0^4 |\grad s|^2 \grad s \cdot \grad \omega. \label{eq:defineregularizedPhi}
\end{equation}
This is a convex combination of vertical velocity expressions which modifies the un-regularized expression $\Phi(s)=-\bu|_s\cdot \bn_s$, regarded as a functional acting on $\cX$.  Evidence for, and discussion of, the following conjecture is given in \cite{Bueler2025}.

\begin{conjecture} \label{conj:regcoercive}  There exists $\eps \in (0,1)$, $\eta_0>0$, and $\alpha>0$, so that $\Phi^\eps:\cK \to \cX'$ and
\begin{equation}
\left(\Phi^\eps(\sigma) - \Phi^\eps(s)\right)[\sigma-s] \ge \alpha \|\sigma-s\|_{\cX}^4 \qquad \text{for all } \sigma,s\in\cK. \label{eq:regcoercive}
\end{equation}
That is, we conjecture that for some regularization constants $\eps,\eta_0$, the regularized surface motion $\Phi^\eps$ is $4$-coercive over admissible surfaces in $\cX=W^{1,4}(\Omega)$.
\end{conjecture}

Consider a backward-Euler time-step of \eqref{eq:ske} over $[0,\Delta t]$ with $\Delta t > 0$.  Let
\begin{equation}
\bar a(x) = \frac{1}{\Delta t} \int_{0}^{\Delta t} a(t,x)\,dt \label{eq:averagea}
\end{equation}
be the average climate.  For a previous surface elevation $s^0\in\cX$, define the source
\begin{equation}
\ell(x) = s^0(x) + \int_{0}^{\Delta t} a(t,x)\,dt = s^0(x) + \Delta t\, \bar a(x). \label{eq:source}
\end{equation}
For a test function $\omega\in\cX$, define the regularized functional for the step:
\begin{equation}
F^\eps_{\Delta t}(s)[\omega] = \Delta t\,\Phi^\eps(s)[\omega] + \int_\Omega s \omega. \label{eq:regularizedF}
\end{equation}

We seek the surface elevation solution $s$, at time $\Delta t$, of the following variational inequality (VI) problem:
\begin{equation}
F^\eps_{\Delta t}(s)[\sigma-s] \ge \ell[\sigma-s] \quad \text{for all } \sigma \in \cK. \label{eq:regularizedvi}
\end{equation}
One might emphasize that $s$ depends on $\Delta t$: $s(x) \approx s(\Delta t,x)$ where $s(t,x)$ is the exact solution of the time-dependent model.

Reference \cite{Bueler2025} shows that, under Conjecture \ref{conj:regcoercive} and an additional conjecture that the surface trace of the Stokes velocity is Lipschitz in the surface elevation, the VI problem \eqref{eq:regularizedvi} has a unique solution.  The main result of this note is the following \emph{a priori} bound on a solution in terms of the data $a,b$ and the regularization constants.

\begin{theorem}
Suppose Conjecture \ref{conj:regcoercive}.  There are positive constants $c_0,c_1$, depending only on $\Omega$, so that, for a solution $s\in\cK$ of VI problem \eqref{eq:regularizedvi}, the corresponding ice thickness $H=s-b$ satisfies the bound
\begin{equation}
\frac{1}{2} (\alpha \Delta t)^{4/3} \|H\|_{\cX}^4 + (\alpha \Delta t)^{1/3} \|H\|_{L^2}^2 \le c_1 \Big(c_0\left\|H^0+\Delta t\,\bar a\right\|_{L^1}  + \Delta t \eps \Gamma \eta_0^4 \|\grad b\|_{L^4}^3\Big)^{4/3} \label{eq:thebound}
\end{equation}
where $H^0=s^0-b$ is the initial thickness and $\bar a$ is given by \eqref{eq:averagea}.
\end{theorem}

\begin{proof}
Note $b\in\cK$.  Substitute $\sigma=b$ into \eqref{eq:regularizedvi} and rewrite as follows:
\begin{equation}
\Delta t \Phi^\eps(s)[s-b] \le \int_\Omega (\ell - s) (s - b).
\end{equation}
Add $\|s-b\|_{L^2}^2=\int_\Omega (s - b)^2$ to both sides to get:
\begin{equation}
\Delta t \Phi^\eps(s)[s-b] + \|s-b\|_{L^2}^2 \le \int_\Omega (\ell - b) (s - b). \label{eq:first}
\end{equation}

Observe from \eqref{eq:defineregularizedPhi} that, because $\bu|_b=\bzero$ when there is no glacier,
\begin{equation}
\Phi^\eps(b)[\omega] = \int_\Omega \eps \Gamma \eta_0^4 |\grad b|^2 \grad b \cdot \grad\omega.
\end{equation}
Subtract $\Delta t\Phi^\eps(b)[s-b]$ from both sides of \eqref{eq:first} to get
\begin{align}
\Delta t (\Phi^\eps(s) - \Phi^\eps(b))[s-b] + \|s-b\|_{L^2}^2 &\le \int_\Omega (\ell - b) (s - b) \\
 &\quad + \Delta t \int_\Omega \eps \Gamma \eta_0^4 |\grad b|^2 \grad b \cdot \grad(s-b). \notag
\end{align}
Now apply coercivity \eqref{eq:regcoercive} on the left side of this inequality, and write $H=s-b$:
\begin{equation}
\alpha \Delta t \|H\|_{\cX}^4 + \|H\|_{L^2}^2 \le \int_\Omega (\ell - b) H + \Delta t \int_\Omega \eps \Gamma \eta_0^4 |\grad b|^2 \grad b \cdot \grad H.
\end{equation}
Apply H\"older's inequality to each integral, using exponents $4/3,4$ in the second:
\begin{equation}
\alpha \Delta t \|H\|_{\cX}^4 + \|H\|_{L^2}^2 \le \|\ell - b\|_{L^1} \|H\|_{L^\infty} + \Delta t \eps \Gamma \eta_0^4 \|\grad b\|_{L^4}^3 \|H\|_{\cX}.
\end{equation}
By Sobolev's inequality, there is $c_0>0$ depending on $\Omega$ so that
\begin{equation}
\alpha \Delta t \|H\|_{\cX}^4 + \|H\|_{L^2}^2 \le \|H\|_{\cX} \left(c_0\|\ell - b\|_{L^1}  + \Delta t \eps \Gamma \eta_0^4 \|\grad b\|_{L^4}^3\right)
\end{equation}

Let $\delta>0$.  By Young's inequality \cite{Evans2010} with exponents $4$ and $4/3$, there is $c_2>0$ independent of $\delta$ so that
\begin{equation}
\alpha \Delta t \|H\|_{\cX}^4 + \|H\|_{L^2}^2 \le \delta \|H\|_{\cX}^4 + c_2 \delta^{-1/3} \left(c_0\|\ell - b\|_{L^1}  + \Delta t \eps \Gamma \eta_0^4 \|\grad b\|_{L^4}^3\right)^{4/3}.
\end{equation}
Choose $\delta = \alpha \Delta t/2$, and note that $\ell-b=H^0 + \Delta t\,\bar a$, to show \eqref{eq:thebound}.
\end{proof}

In the short-time limit $\Delta t\to 0$, bound \eqref{eq:thebound} essentially says $H_0\ge 0$, which is not interesting.

Considering a time-dependent climate, even a periodic one, the solution geometry of the time-dependent model \cite{Bueler2025} generally does not have a long-time limit.  However, recalling that $a \in C([0,\infty); L^1(\Omega))$, in many cases a long-time average climate does exist, as the limit
\begin{equation}
a^\infty(x) = \lim_{\Delta t \to \infty} \bar a(x) = \lim_{\Delta t \to \infty} \frac{1}{\Delta t} \int_0^{\Delta t} a(t,x)\,dx. \label{eq:longaverage}
\end{equation}
This limit exists when $a$ is periodic in $t$, e.g.~with a yearly or even an ice-age period, and in such cases $a^\infty = \bar a$ is the average over the period.  When $a^\infty$ exists then the solution of a backward-Euler step problem \eqref{eq:regularizedvi} is well-behaved, and for large $\Delta t$ its norm is bounded in terms of $a^\infty$ and $\grad b$.

\begin{theorem}
Again suppose Conjecture \ref{conj:regcoercive}, and that limit \eqref{eq:longaverage} exists.  Then there exists $M>0$, which can be written in terms of $\|a^\infty\|_{L^1}$ and $\|\grad b\|_{L^4}$, but which is independent of $\Delta t$, so that if $\Delta t$ is sufficiently large then $s$ solving \eqref{eq:regularizedvi} is bounded by $M$:
\begin{equation}
\|H\|_{\cX} = \|s-b\|_{\cX} \le M.  \label{eq:longbound}
\end{equation}
\end{theorem}

\begin{proof}
Drop the $L^2$ term from the left of \eqref{eq:thebound}, and divide by $(\alpha \Delta t)^{4/3}/2$ to get this bound:
\begin{equation}
\|H\|_{\cX}^4 \le \left(c_3 \left\|\frac{1}{\Delta t} H^0+\bar a\right\|_{L^1} + c_4 \|\grad b\|_{L^4}^3\right)^{4/3}. \label{eq:longbound:first}
\end{equation}
(The constants $c_i$ are independent of $\Delta t$.)  By dominated convergence the $\Delta t\to\infty$ limit on the right of \eqref{eq:longbound:first} is $m=\left(c_3 \|a^\infty\|_{L^1}  + c_4 \|\grad b\|_{L^4}^3\right)^{4/3}$.  Thus $\|H\|_{\cX}^4$ is eventually bounded by $M=m+1$.
\end{proof}


{\small
\bibliographystyle{siamplain}
\bibliography{estimate}
}
\end{document}
