\documentclass[12pt]{article}

\usepackage{amssymb,amsmath,amsthm}
\usepackage[top=1in, bottom=1in, left=1.25in, right=1.25in]{geometry}
\usepackage{enumerate,palatino,bm}
\usepackage[final]{graphicx}
\usepackage[colorlinks=true,citecolor=blue,linkcolor=red,urlcolor=blue]{hyperref}

\newtheorem{theorem}{Theorem}
\newtheorem{conjecture}{Conjecture}

\newcommand{\RR}{\ensuremath{\mathbb R}}
\newcommand{\ZZ}{\ensuremath{\mathbb Z}}

\newcommand{\bu}{\ensuremath{\mathbf{u}}}

\newcommand{\bzero}{\ensuremath{\bm{0}}}

\newcommand{\cK}{\ensuremath{\mathcal{K}}}
\newcommand{\cX}{\ensuremath{\mathcal{X}}}

\newcommand{\grad}{\ensuremath{\nabla}}
\newcommand{\eps}{\ensuremath{\epsilon}}


\title{A bound on the surface elevation solution \\ in terms of the climate}
\author{Ed Bueler}

\begin{document}
\maketitle

We consider a glacier geometry problem, to determine the surface elevation if the dynamical model is a Stokes balance or similar.  We consider a single, regularized backward-Euler step.  The main idea in this short note is that if we conjecture coercivity for the regularized surface motion then we can bound a Sobolev norm of the steady-state surface elevation in terms of the average climate.

Suppose $\Omega\subset \RR^2$ is a bounded domain, with notation $x=(x_1,x_2)\in\Omega$ for a location in the map-plane.  We use the following Banach space for surface elevations:
\begin{equation}
\cX = W^{1,4}(\Omega), \label{eq:defineX}
\end{equation}
with norm $\|s\|_{\cX}^4 = \int_\Omega |s|^4 + \int_\Omega |\grad s|^4$ for this note.

The climate data $a(t,x)$ is assumed to be in $C([0,\infty); L^{4/3}(\Omega))$.  We fix a bed elevation function $b\in C^1(\bar\Omega)$, and define
\begin{equation}
\cK=\{\sigma\in\cX\,:\,\sigma\ge b \text{ and } \sigma|_{\partial\Omega}=b|_{\partial\Omega}\} \label{eq:defineK}
\end{equation}

Let $\bu=(u_1,u_2,w)$ be the solution of the Stokes problem over the domain
\begin{equation}
\Lambda(s) = \left\{(x,z)\,:\,b(x) < z < s(x)\right\} \subset \Omega \times \RR. \label{eq:domainfroms}
\end{equation}
The surface trace of the velocity is denoted $\bu|_s=(u_1|_s,u_2|_s,w|_s)$.  This surface velocity is extended by zero to all of $\Omega$, that is, it is zero outside of the part which is covered by ice ($s>b$).  Now, for $\eps>0$ and $\eta_0>0$, and physical constant $\Gamma>0$, define the SIA-regularized surface motion
\begin{equation}
\Phi^\eps(s)[\omega] = \int_\Omega \Big(\big(u_1|_s,u_2|_s\big) \cdot \grad s - (1-\eps) w|_s\Big) \omega + \eps\, \Gamma \eta_0^4 |\grad s|^2 \grad s \cdot \grad \omega. \label{eq:defineregularizedPhi}
\end{equation}

Reference \cite{Bueler2025} makes the following conjecture that $\Phi^\eps$ is $4$-coercive.

\begin{conjecture} \label{conj:regcoercive}  There exists $\eps \in (0,1)$ and $H_0>0$, and $\alpha>0$, so that
\begin{equation}
\left(\Phi^\eps(\sigma) - \Phi^\eps(s)\right)[\sigma-s] \ge \alpha \|\sigma-s\|_{\cX}^4 \qquad \text{for all } \sigma,s\in\cK. \label{eq:regcoercive}
\end{equation}
\end{conjecture}

Consider a backward-Euler time-step over $[0,\Delta t]$ with $\Delta t > 0$.  For the previous surface elevation $s^0\in\cX$, define the source term
\begin{equation}
\ell(x) = s^0(x) + \int_{0}^{\Delta t} a(t,x)\,dt. \label{eq:be:source}
\end{equation}
For a test function $\omega\in\cX$, define the regularized functional for the step:
\begin{equation}
F^\eps_{\Delta t}(s)[\omega] = \Delta t\,\Phi^\eps(s)[\omega] + \int_\Omega s \omega, \label{eq:regularizedF}
\end{equation}
The variational inequality (VI) problem for this step is to find the surface elevation solution $s$, of the time-step, to
\begin{equation}
F^\eps_{\Delta t}(s)[\sigma-s] \ge \ell[\sigma-s] \quad \text{for all } \sigma \in \cK. \label{eq:regularizedvi}
\end{equation}
Note that $s$ depends on $\Delta t$.

Reference \cite{Bueler2025} shows that, under Conjecture \ref{conj:regcoercive} and an additional conjecture that the surface trace of the Stokes velocity is Lipschitz in the surface elevation, for any $\Delta t>0$ the VI problem \eqref{eq:regularizedvi} has a unique solution.

\begin{theorem}
Suppose Conjecture \ref{conj:regcoercive}.  There are positive constants $c_0,c_1$, depending only on $\Omega$, so that, for the solution $s\in\cK$ of VI problem \eqref{eq:regularizedvi}, the corresponding ice thickness $H=s-b$ satisfies the bound
\begin{equation}
\frac{1}{2} \alpha \Delta t \|H\|_{\cX}^4 + \|H\|_{L^2}^2 \le c_1 (\alpha \Delta t)^{-1/3} \left({\large \strut}c_0\left\|\ell-b\right\|_{L^1}  + \Delta t \eps \Gamma \eta_0^4 \|\grad b\|_{L^4}^3\right)^{4/3}. \label{eq:thebound}
\end{equation}
\end{theorem}

\begin{proof}
Note $b\in\cK$.  Substitute $\sigma=b$ into \eqref{eq:regularizedvi} and rewrite as follows:
\begin{equation}
\Delta t \Phi^\eps(s)[s-b] \le \int_\Omega (\ell - s) (s - b).
\end{equation}
Add $\|s-b\|_{L^2}^2=\int_\Omega (s - b)^2$ to both sides and rewrite to get:
\begin{equation}
\Delta t \Phi^\eps(s)[s-b] + \|s-b\|_{L^2}^2 \le \int_\Omega (\ell - b) (s - b). \label{eq:first}
\end{equation}
Observe from \eqref{eq:defineregularizedPhi} that, because $\bu|_b=\bzero$ when there is no glacier,
\begin{equation}
\Phi^\eps(b)[\omega] = \int_\Omega \eps \Gamma \eta_0^4 |\grad b|^2 \grad b \cdot \grad\omega.
\end{equation}
Subtract $\Delta t\Phi^\eps(b)[s-b]$ from both sides of \eqref{eq:first} to get
\begin{align}
\Delta t (\Phi^\eps(s) - \Phi^\eps(b))[s-b] + \|s-b\|_{L^2}^2 &\le \int_\Omega (\ell - b) (s - b) \\
 &\quad + \Delta t \int_\Omega \eps \Gamma \eta_0^4 |\grad b|^2 \grad b \cdot \grad(s-b). \notag
\end{align}
Apply coercivity \eqref{eq:regcoercive} on the left.  Then apply H\"older's inequality to each integral on the right, and write $H=s-b$:
\begin{align}
\alpha \Delta t \|H\|_{\cX}^4 + \|H\|_{L^2}^2 &\le \|\ell - b\|_{L^1} \|H\|_{L^\infty} + \Delta t \eps \Gamma \eta_0^4 \left\||\grad b|^3\right\|_{L^{4/3}} \|\grad H\|_{L^4} \\
 &\le \|\ell - b\|_{L^1} \|H\|_{L^\infty} + \Delta t \eps \Gamma \eta_0^4 \|\grad b\|_{L^4}^3 \|H\|_{\cX} \notag
\end{align}
By Sobolev's inequality, there is $c_0>0$ depending on $\Omega$ so that
\begin{equation}
\alpha \Delta t \|H\|_{\cX}^4 + \|H\|_{L^2}^2 \le \|H\|_{\cX} \left({\large \strut}c_0\|\ell - b\|_{L^1}  + \Delta t \eps \Gamma \eta_0^4 \|\grad b\|_{L^4}^3\right)
\end{equation}
Let $\delta>0$.  By Young's inequality \cite{Evans2010} with exponents $4$ and $4/3$, there is $c_2>0$ independent of $\delta$ so that
\begin{equation}
\alpha \Delta t \|H\|_{\cX}^4 + \|H\|_{L^2}^2 \le \delta \|H\|_{\cX}^4 + c_2 \delta^{-1/3} \left({\large \strut}c_0\|\ell - b\|_{L^1}  + \Delta t \eps \Gamma \eta_0^4 \|\grad b\|_{L^4}^3\right)^{4/3}.
\end{equation}
Choose $\delta = \alpha \Delta t/2$ to show \eqref{eq:thebound}.
\end{proof}

Define the long-term limit of the climate as
\begin{equation}
\bar a(x) = \lim_{\Delta t \to \infty} \frac{1}{\Delta t} \int_0^{\Delta t} a(t,x)
\end{equation}
We will suppose that this limit exists.

\begin{theorem}
Again suppose Conjecture \ref{conj:regcoercive}.  The limit $\bar H = \lim_{\Delta t\to \infty} H$ exists, and we have the bound
\begin{equation}
\|\bar H\|_{\cX}^4 \le FIXME \label{eq:longbound}
\end{equation}
\end{theorem}

\bibliographystyle{siamplain}
\bibliography{estimate}

\end{document}