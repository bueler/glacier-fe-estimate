\documentclass[hidelinks,onefignum,onetabnum,final]{siamart220329}  % for arxiv
%\documentclass[review,hidelinks,onefignum,onetabnum,final]{siamart220329}  % for submission

\usepackage{amsfonts}
\usepackage{graphicx}
\usepackage{epstopdf}
\ifpdf
  \DeclareGraphicsExtensions{.eps,.pdf,.png,.jpg}
\else
  \DeclareGraphicsExtensions{.eps}
\fi

% Used for creating new theorem and remark environments
\newsiamremark{remark}{Remark}
\newsiamremark{hypothesis}{Hypothesis}
\crefname{hypothesis}{Hypothesis}{Hypotheses}
\newsiamthm{claim}{Claim}
\newsiamremark{example}{Example}

\usepackage{amsopn}
\DeclareMathOperator{\diag}{diag}

\usepackage{bm,bbm,empheq,verbatim,fancyvrb,amssymb}
\usepackage{booktabs,multirow,xspace}
\usepackage{pifont}

\usepackage{tikz}
\usetikzlibrary{decorations.pathreplacing}
\usetikzlibrary{graphs,quotes}

\newcommand{\eps}{\epsilon}
\newcommand{\RR}{\mathbb{R}}

\newcommand{\grad}{\nabla}
\newcommand{\Div}{\nabla\cdot}

\newcommand{\bn}{\mathbf{n}}
\newcommand{\bw}{\mathbf{w}}
\newcommand{\bz}{\mathbf{z}}
\newcommand{\bX}{\mathbf{X}}

\newcommand{\cB}{\mathcal{B}}
\newcommand{\cH}{\mathcal{H}}
\newcommand{\cK}{\mathcal{K}}
\newcommand{\cV}{\mathcal{V}}

\newcommand{\ip}[2]{\left<#1,#2\right>}

\newcommand{\XX}{\ding{55}}

\newcommand{\dx}{\, \mathrm{d}x}

\DeclareMathOperator*{\argmin}{arg\,min}


% Sets running headers as well as PDF title and authors
\headers{Geometry errors in glacier simulations}{E. Bueler}

\title{A framework for estimating geometry errors \\ in glacier simulations}

\author{Ed Bueler\thanks{Department of Mathematics and Statistics, University of Alaska Fairbanks, USA
  (\email{elbueler@alaska.edu}).}}


\begin{document}

\maketitle

\begin{abstract}
The free-boundary problem of determining glacier geometry from the major data, namely bedrock elevation (topography) and the balance of snow accumulation minus melt (surface mass balance), can be posed as a variational inequality in terms of the surface elevation or thickness of the glacier.  This problem, over the closed and convex set of admissible surface elevations (thicknesses), is defined by the property that the ice surface elevation must be above the bed topography (thickness nonnegative).  We first an abstract estimate for finite element approximations of coercive variational inequalities over Banach spaces.  FIXME
\end{abstract}

% REQUIRED
\begin{keywords}
finite element methods, variational inequalities, ice flow, glaciers
\end{keywords}

% REQUIRED
\begin{MSCcodes}
FIXME
\end{MSCcodes}


\section{Introduction} \label{sec:intro}

FIXME


\section{Abstract error estimate} \label{sec:abstractestimate}

We first consider an abstract variational inequality (VI) problem as follows.  Let $\cV$ be a real reflexive Banach space with norm $\|\cdot\|$ and topological dual space $\cV'$.  Denote the dual pairing of $\phi \in \cV'$ and $v\in\cV$ by $\ip{\phi}{v} = \phi(v)$, and define the (Banach space) norm on $\cV'$ by $\|\phi\|' = \sup_{\|v\|=1} |\!\ip{\phi}{v}\!|$.

Let $\cK \subset \cV$ be a nonempty, closed, and convex subset, the constraint set; elements of $\cK$ are said to be admissible.  For a continuous, but generally nonlinear, operator $f:\cK \to \cV'$, and a linear source $\ell\in \cV'$, the problem is to find $u\in \cK$ such that
\begin{equation}
\ip{f(u)}{v-u} \ge \ip{\ell}{v-u} \quad \text{for all } v\in \cK. \label{eq:vi}
\end{equation}

The following definitions are standard \cite[for example]{KinderlehrerStampacchia1980}.

\begin{definition}  An operator $f:\cK \to \cV'$ is \emph{monotone} if
\begin{equation}
\ip{f(v)-f(w)}{v-w} \ge 0 \qquad \text{for all } v,w \in \cK, \label{eq:monotone}
\end{equation}
\emph{strictly monotone} if equality in \eqref{eq:monotone} implies $u=v$.
\end{definition}

\begin{definition}  Let $p>1$.  The operator $f:\cK \to \cV'$ is \emph{$p$-coercive} if there exists $\kappa>0$ such that
\begin{equation}
\ip{f(v)-f(w)}{v-w} \ge \kappa \|v-w\|^p \qquad \text{for all } v,w \in \cK. \label{eq:pcoercive}
\end{equation}
\end{definition}

\begin{definition}  For $\gamma>0$ let $B_\gamma = \{v\in \cV\,:\,\|v\|\le \gamma\}$.  We say $f:\cK \to \cV'$ is \emph{uniformly continuous on bounded sets of $\cK$} if for every $\gamma>0$ there is $C=C(\gamma)>0$ so that if $v,w \in B_\gamma \cap \cK$ and $z\in\cV$ then $|\ip{f(v)-f(w)}{z}| \le C(\gamma) \|z\|$.  Equivalently there is $C(\gamma)>0$ so that
\begin{equation}
\|f(v)-f(w)\|_{\cV'} \le C(\gamma) \quad \text{ for all } v,w \in B_\gamma \cap \cK.  \label{eq:continuousonbounded}
\end{equation}
\end{definition}

% \cite{Peral1997} uniform continuity over bounded sets for p-Laplacian: Thm A.0.6

It is well-known that if $f:\cK \to \cV'$ is continuous, monotone, and $p$-coercive then VI \eqref{eq:vi} has a solution \cite[Corollary III.1.8]{KinderlehrerStampacchia1980}; see also \cite{BuelerFarrell2024}.  It is easy to see that if $f$ is $p$-coercive then it is monotone and strictly monotone, thus the solution is unique.

Note that in all of the above we only assumed $f$ is defined on $\cK$.  This will be an issue to return to.

FIXME finite element space $\cV_h$, constraint set $\cK_h\subset \cV_h$, note generally $\cK_h \nsubseteq \cK$.  let $f_h:\cK_h\to\cV'$.  the FE VI problem is
\begin{equation}
\ip{f(u_h)}{v_h-u_h} \ge \ip{\ell}{v_h-u_h} \quad \text{for all } v_h\in \cK_h. \label{eq:fe:vi}
\end{equation}
we assume this problem is well-posed

Comparable to the Hilbert-space and bilinear version of the following theorem, i.e.~as in section 5.1 of Ciarlet \cite{Ciarlet2002}, and in \cite{Falk1974}, we assume that $\cV$ densely embeds in a larger Banach space $\cB$:
\begin{equation}
\cV \hookrightarrow \cB, \quad \overline{\cV} = \cB
\end{equation}
Observe that $\cB' \subset \cV'$.  The canonical example has $\cV=W^{1,p}(\Omega)$ and $\cB=L^p(\Omega)$.

A key concept here is that neither $\|f(u)-\ell\|_{\cB'}$, nor $\|f(u)-\ell\|_{\cV'}$, is generally zero when $u$ solves \eqref{eq:vi}.  Instead a nonlinear complementarity problem will hold, in sufficiently regular cases, as in Exercise 5.1.1 of \cite{Ciarlet2002}.  Only on the inactive set does the equation ``$f(u)=\ell$'' apply.

It is important to note that we assume $f$ is defined on all of $\cV$ in the following abstract error estimate.

\begin{theorem} \label{thm:abstractestimate}
Let $1<p<\infty$, and write $q$ for the conjugate exponent.  Suppose $f:\cV \to \cV'$ is $p$-coercive with coercivity constant $\alpha>0$.  Suppose $f$ is also uniformly continuous on bounded sets of $\cV$.  Suppose $u\in\cK$ solves \eqref{eq:vi} and $u_h\in\cK_h$ solves \eqref{eq:fe:vi}.  Let $R_h=\max\{\|u\|,\|u_h\|\}$.  Then there is a constant $c(R_h)>0$ so that
\begin{align}
\|u-u_h\|_{\cV} &\le \Big(\inf_{v_h\in\cK_h} \left\{c(R_h) \|u - v_h\|^q + \frac{2}{\alpha} \|f(u)-\ell\|_{\cB'} \|u-v_h\|_{\cB}\right\} \label{eq:abstractestimate} \\
   &\qquad + \inf_{v\in\cK} \frac{2}{\alpha} \|f(u)-\ell\|_{\cB'} \|u_h-v\|_{\cB}\Big)^{1/p} \notag
\end{align}
\end{theorem}

\begin{proof}  Rewrite VIs \eqref{eq:vi} and \eqref{eq:fe:vi} as follows:
\begin{align*}
\ip{f(u)}{u} &\le \ip{f(u)}{v} + \ell(v-u) \\
\ip{f(u_h)}{u_h} &\le \ip{f(u_h)}{v_h} + \ell(v_h-u_h)
\end{align*}
It follows from $p$-coercivity \eqref{eq:pcoercive} that for arbitrary $v\in\cK$ and $v_h\in\cK_h$,
\begin{align*}
\alpha \|u-u_h\|^p &\le \ip{f(u)-f(u_h)}{u-u_h} \\
  &= \ip{f(u)}{u} + \ip{f(u_h)}{u_h} - \ip{f(u)}{u_h} - \ip{f(u_h)}{u} \\
  &\le \ip{f(u)}{v} + \ell(v-u) + \ip{f(u_h)}{v_h} + \ell(v_h-u_h) \\
  &\qquad - \ip{f(u)}{u_h} - \ip{f(u_h)}{u} \\
  &= \ip{f(u)-\ell}{v-u_h} + \ip{f(u)-\ell}{v_h-u} \\
  &\qquad + \ip{f(u)-f(u_h)}{u-v_h}
\end{align*}
Since $u,u_h\in B_{R_h}$, by uniform continuity over bounded sets of $\cV$ there is $C(R_h)>0$ so that $\|f(u)-f(u_h)\|_{\cV'} \le C(R_h)$.  Observe that $\|f(u)-\ell\|_{\cB'} < \infty$ because $f:\cV\to\cV' \supset \cB'$.  It follows that
\begin{align*}
\alpha \|u-u_h\|^p &\le \|f(u)-\ell\|_{\cB'} \left(\|v-u_h\|_{\cB} + \|v_h-u\|_{\cB}\right) \\
  &\qquad + C(R_h) \|u-u_h\| \|u-v_h\|
\end{align*}
Now using Young's inequality with $\eps>0$ \cite[Appendix B.2]{Evans2010},
\begin{align*}
\alpha \|u-u_h\|^p &\le \|f(u)-\ell\|_{\cB'} \left(\|v-u_h\|_{\cB} + \|v_h-u\|_{\cB}\right) \\
  &\qquad + C(R_h) \left(\eps\|u-u_h\|^p + \tilde C(\eps) \|u-v_h\|^q\right)
\end{align*}
where $\tilde C(eps) = (\eps p)^{-q/p} q^{-1}$.  Choose $\eps$ so that $C(R_h) \eps \le \alpha/2$, and move terms to find
\begin{align*}
\frac{\alpha}{2} \|u-u_h\|^p &\le \|f(u)-\ell\|_{\cB'} \left(\|v-u_h\|_{\cB} + \|v_h-u\|_{\cB}\right) \\
  &\qquad + C(R_h) \tilde C(\eps) \|u-v_h\|^q
\end{align*}
Finally, multiply by $2/\alpha$ and take infimums to show \eqref{eq:abstractestimate}.
\end{proof}

This theorem reduces to the abstract error estimate for variational inequalities in \cite{Ciarlet2002} and \cite{Falk1974}.  Specifically, suppose $\ip{f(v)}{w}=a(v,w)$ is actually bilinear, coercive, and continuous on a Hilbert space $\cV$.  Define $A:\cV\to\cV'$, a bounded linear operator, by $Av(w) = a(v,w)$.  Suppose that $\cV\hookrightarrow \cH$ and $\overline{\cV} = \cH$ for another Hilbert space $\cH$, and that $\|Au-\ell\|_{\cH'} < \infty$ so, up to isomorphism, $Au-\ell \in\cH$.  Then Theorem \ref{thm:abstractestimate} reduces to Theorem 5.1.1 of \cite{Ciarlet2002}.


\section{Application to glaciers} \label{sec:application}

We now consider the particular variational equalities satisfied either by the glacier's surface elevation or by its thickness.  While essentially equivalent at the level of the continuum problem,\footnote{Equivalence holds for smooth bedrock elevation, but real bedrock topography is likely not $C^1$.} the problems have different character when FE approximations are applied and the above abstract estimate is considered.  Before addressing the two approaches we consider the aspects of the problems which are in common.

Let $\Omega \subset \RR^2$ be a fixed region of land, and suppose $b \in C^1(\Omega)$ is the bedrock elevation on $\Omega$.  (For simplicity we are assuming that $b=b(x)$ is time-independent.)  For $T>0$ let $a(t,x) \in C([0,T] \times \Omega)$ denote a given ``surface mass balance'' function \cite{GreveBlatter2009}, the annually-averaged rate of ice accumulation, as snow, minus melt and runoff.  Where $a(t,x)>0$ there is more snow in a year than can melt (accumulation), while if $a(t,x)<0$ the opposite is true (ablation).  Note that $a(t,x)$ is defined a.e.~$x\in \Omega$, regardless of whether a glacier is present or not.\footnote{Where there is no glacier, and thus necessarily $a(t,x) \le 0$, the value of $a(t,x)$ can be computed by a surface energy balance model \cite{GreveBlatter2009}, at that location, by hypothesizing a bare ice surface over an arbitrary depth of ice, and computing the total runoff from the available energy for melt.  This is balanced against snow accumulation, if any.  It is important that $a(t,x)$ represent the surface mass balance \emph{which a glacier would experience if present at that time and place}.}

Here are the two approaches when posed for the steady geometry of a glacier in a hypothesized steady climate.  FIXME show strong forms here

\begin{enumerate}
\item FIXME fluid layer thickness equation, also called mass continuity or Saint Venant equation \cite{JouvetBueler2012}, $\cK = \{H\ge 0\}$, $\cK_h \subset \cK$, apparently $\eta = H^{2p/(p-1)} \in W^{1,p}(\Omega)$, for e.g.~$p=4$, because of existence in SIA case \cite{JouvetBueler2012}, so we write in terms of $\eta$ even for other dynamical models
\item FIXME surface kinematical equation \cite{GreveBlatter2009}, $\cK = \{s\ge b\}$, $\cK_h \nsubseteq \cK$ in general, perhaps $\cV = W^{1,2}(\Omega)$
\end{enumerate}

FIXME show Greenland surface-vs-thickness figure

FIXME however, regularity arising from analyzing SIA not to be trusted because actual dynamics is not diffusive for short wavelength surface/thickness perturbations \cite{Pattynetal2008}, so for simplicity we also consider ``advective'' models with hypothesized transport velocities (details below)

FIXME time-dependent (implicit time step) case set up in \cite{Bueler2021conservation}

FIXME however, above is just strong forms; need to actually give operators to apply theorems.  we split case 1 into 1A (SIA) and 1B (ice flux is hypothesized ice velocity times thickness).  in case 2, somewhat like 1B, we assume surface ice velocity is given

\newcommand{\foneA}{f_{\,\text{1A}}}
\newcommand{\foneB}{f_{\,\text{1B}}}

FIXME write out operator $\ip{\foneA(\eta)}{w}$ (on $W^{1,p}(\Omega)$) or $\ip{\foneB(H)}{w}$ (on $W^{1,2}(\Omega)$) or $\ip{f_2(s)}{w}$  (on $W^{1,p}(\Omega)$)

FIXME apply Theorem \ref{thm:abstractestimate} and talk through what happens


\bibliographystyle{siamplain}
\bibliography{estimate}

\end{document}
