\documentclass[hidelinks,onefignum,onetabnum,final]{siamart220329}  % for arxiv
%\documentclass[review,hidelinks,onefignum,onetabnum,final]{siamart220329}  % for submission

\usepackage{amsfonts,yhmath}
\usepackage{graphicx}
\usepackage{epstopdf}
\ifpdf
  \DeclareGraphicsExtensions{.eps,.pdf,.png,.jpg}
\else
  \DeclareGraphicsExtensions{.eps}
\fi

% Used for creating new theorem and remark environments
\newsiamremark{remark}{Remark}
\newsiamthm{claim}{Claim}
\newsiamremark{example}{Example}

\newcommand{\newindthm}[2]{
  \theoremstyle{plain}
  \theoremheaderfont{\normalfont\sc}
  \theorembodyfont{\normalfont\itshape}
  \theoremseparator{.}
  \theoremsymbol{}
  \newtheorem{#1}{#2}
}
\newindthm{conjecture}{Conjecture}
\renewcommand{\theconjecture}{\Alph{conjecture}}

\newcommand{\newindthmstar}[2]{
  \theoremstyle{plain}
  \theoremheaderfont{\normalfont\sc}
  \theorembodyfont{\normalfont\itshape}
  \theoremseparator{.}
  \theoremsymbol{}
  \newtheorem*{#1}{#2}
}
\newindthmstar{assumptions}{Standard Assumptions}

\usepackage{amsopn}
\DeclareMathOperator{\diag}{diag}

\usepackage{bm,bbm,empheq,verbatim,fancyvrb,amssymb}
\usepackage{booktabs,multirow,xspace}
\usepackage{pifont}

\usepackage{tikz}
\usetikzlibrary{decorations.pathreplacing}
\usetikzlibrary{graphs,quotes}

\newcommand{\eps}{\epsilon}
\newcommand{\RR}{\mathbb{R}}

\newcommand{\grad}{\nabla}
\newcommand{\Div}{\nabla\cdot}

\newcommand{\bbf}{\mathbf{f}}
\newcommand{\bg}{\mathbf{g}}
\newcommand{\bn}{\mathbf{n}}
\newcommand{\bu}{\mathbf{u}}
\newcommand{\bv}{\mathbf{v}}
\newcommand{\bw}{\mathbf{w}}
\newcommand{\bx}{\mathbf{x}}
\newcommand{\bz}{\mathbf{z}}

\newcommand{\bU}{\mathbf{U}}
\newcommand{\bX}{\mathbf{X}}

\newcommand{\bzero}{\bm{0}}

\newcommand{\btau}{\bm{\tau}}

\newcommand{\cB}{\mathcal{B}}
\newcommand{\cH}{\mathcal{H}}
\newcommand{\cK}{\mathcal{K}}
\newcommand{\cQ}{\mathcal{Q}}
\newcommand{\cT}{\mathcal{T}}
\newcommand{\cV}{\mathcal{V}}
\newcommand{\cX}{\mathcal{X}}

\newcommand{\hcK}{\widehat{\cK}}

\newcommand{\nn}{{\text{\textnormal{n}}}}
\newcommand{\pp}{{\text{\textnormal{p}}}}
\newcommand{\qq}{{\text{\textnormal{q}}}}
\newcommand{\rr}{{\text{\textnormal{r}}}}

\newcommand{\ip}[2]{\left<#1,#2\right>}

\newcommand{\XX}{\ding{55}}

\newcommand{\dx}{\, \mathrm{d}x}

\newcommand{\rhoi}{\rho_{\text{i}}}

\DeclareMathOperator*{\argmin}{arg\,min}
\DeclareMathOperator*{\Hull}{Hull}

\newcommand{\Vdiv}{\cV_{\text{\textnormal{div}}}}

\newcommand{\CA}{C_\text{\textnormal{A}}}

% running headers and PDF metadata
\headers{Glacier geometry evolution under Stokes dynamics}{E. Bueler}

\title{Glacier geometry evolution under Stokes dynamics}

\author{Ed Bueler\thanks{Department of Mathematics and Statistics, University of Alaska Fairbanks, USA (\email{elbueler@alaska.edu}).}}


\begin{document}
\maketitle

\begin{abstract}
The primary data which determine the evolution of glaciation are the bedrock elevation and surface mass balance functions.  From this data the glacier's geometry solves a free-boundary problem over a set of admissible surface elevation functions; a surface is admissible if it is above the bedrock topography, equivalently if the ice thickness is nonnegative.  For an implicit time step, this free-boundary problem can be posed in weak form as a variational inequality over a fixed map-plane region.  After some preparatory theory, recalling and expanding upon what is known about the glaciological Stokes problem, we conjecture that the continuous-space problem at each implicit time step is well-posed.  This conjecture is supported by physical arguments and certain numerical evidence.  We then prove a general theorem which estimates the error made by a finite element approximation of a variational inequality problem over a Banach space.  The provided estimate is a sum of errors of different types, specific to variational inequalities.  When specialized to the glacier case the terms in the error estimate can be physically-interpreted.  Practical approaches to bound and/or reduce these errors are then proposed.
\end{abstract}

% REQUIRED
\begin{keywords}
error bounds, finite element methods, glaciers, ice flow, variational inequalities
\end{keywords}

% REQUIRED
%\begin{MSCcodes}
%FIXME
%\end{MSCcodes}


\section{Introduction} \label{sec:intro}

Glacier and ice sheet simulations typically model the ice as a free-surface layer of very-viscous, incompressible, and non-Newtonian fluid \cite{GreveBlatter2009,SchoofHewitt2013}.  For simplicity we will only consider such simulations of glaciers on land, without floating portions.  Note that an ``ice sheet'' is simply a continent-scale glacier.

The two essential types of input data into such simulations are the bedrock elevation, which is assumed here to be constant in time, and the time-dependent surface mass balance rate (SMB; it is also called the climatic mass balance rate \cite{Cogleyetal2011}).  By definition, the SMB is the difference between the vertically-accumulating snow and the loss of (liquid) water, through runoff, at the upper surface of the glacier \cite{Cogleyetal2011}.  It is positive in areas of snow accumulation, and it is often provided only through annual averages.  Note that elevations are measured here in meters, while SMB is measured in ice-equivalent units of meters per second.

Thus a glacier simulation takes, as inputs, at least these data: the bedrock topography, a (generally) time-dependent climate, and an initial geometry.  It produces the glacier's evolving geometry and flow velocity, the output fields of primary scientific value.  Note that additional complications are common in comprehensive ice sheet models \cite{SchoofHewitt2013,Winkelmannetal2011}.  For example, the internal energy \cite{Aschwandenetal2012} or temperature of the ice may be tracked, and/or any liquid water within the ice matrix and at ice surfaces.  However, here we only consider conservation of mass and momentum, but not of energy, and liquid water plays no role.  Furthermore we will assume zero velocity, i.e.~a non-sliding and non-penetrating condition, at the base of the ice.  On the other hand, we will not make the shallowness assumptions which are common in comprehensive models.

The glacier's geometry is parameterized by either the (upper) surface elevation or the ice thickness.  However, the computed flow velocity is only defined at those locations and times where ice is present, namely on an evolving 3D domain between the bedrock and surface elevations.  Furthermore, at a time and map-plane location where a glacier exists the surface elevation must exceed the bedrock elevation, equivalently the ice thickness must be positive.  That is, the glacier's geometry must satisfy an inequality to be admissible.

To expand upon this inequality we start with the notation sketched in Figure \ref{fig:stokesdomain}.  Let $\Omega \subset \RR^2$ be a fixed portion of land, with map-plane coordinates $x=(x_1,x_2)\in\Omega$.  All time-dependent quantities are assumed to be defined for $t\in [0,T]$ for some given $T>0$.  On $\Omega$ assume that we are given, as data, a real, continuous, and time-independent bed elevation function $b(x)$, and a real, continuous SMB function $a(t,x)$, generally taking both signs.  In areas where $a>0$ (accumulation; downward arrows in Figure \ref{fig:stokesdomain}), a glacier will exist.  If $a<0$ (upward arrows) then either a glacier exists with an ablating surface, or no glacier exists.  Determining which situation applies at a given point $t,x$ requires solving free-boundary problems like those considered in this paper.

\begin{figure}[ht]
\centering
\includegraphics[width=0.65\textwidth]{genfigs/stokesdomain.pdf}
\caption{Glacier notation used in this paper.}
\label{fig:stokesdomain}
\end{figure}

Let $s(t,x)$ be the (solution) ice surface elevation.  We will regard this as defined for all $x\in\Omega$, and subject to the constraint that the surface $z=s$ must be at or above the bedrock ($s \ge b$), and in regions with no ice we set $s=b$.  The solution ice velocity $\bu(t,x,z)$ and pressure $p(t,x,z)$ are then defined only on the 3D domain
\begin{equation}
\Lambda(t) = \left\{(x,z)\,:\,b(x) < z < s(t,x)\right\} \subset \Omega \times \RR. \label{eq:icydomain}
\end{equation}
This aspect of glacier modeling deserves emphasis:  The time-dependent 3D domain $\Lambda(t)$, on which the solution velocity and pressure are meaningful, is determined by the evolving surface elevation $s(t,x)$, which is itself part of the model solution.

The surface trace of the ice velocity will be of importance.  We extend it by zero so that it is defined everywhere in $\Omega$.  Let
\begin{equation}
\bu|_s(t,x) = \begin{cases} \bu(t,x,s(t,x)), & s(t,x)>b(t,x) \\
                            \bzero, & \text{otherwise} .\end{cases} \label{eq:defineus}
\end{equation}
Compare flux extension by zero in \cite{SchoofHewitt2013}.  Definition \eqref{eq:defineus} will be reconsidered later in a precise Sobolev space context.

Let $\bn_s = \left<-\grad s,1\right>$ be an un-normalized and upward surface normal vector.  (This is assumed well-defined for the purposes of this Introduction; compare Section \ref{sec:model}.)  Then an infinite-dimensional nonlinear complementarity problem (NCP) \cite{Bueler2021conservation,FacchineiPang2003,SchoofHewitt2013} applies almost everywhere in $[0,T]\times \Omega$:
\begin{subequations}
\label{eq:ncp}
\begin{align}
s - b &\ge 0 \\
\frac{\partial s}{\partial t} - \bu|_s \cdot \bn_s - a &\ge 0 \\
(s - b) \left(\frac{\partial s}{\partial t} - \bu|_s \cdot \bn_s - a\right) &= 0
\end{align}
\end{subequations}
This strong form NCP statement will be reformulated as a variational inequality (VI; \cite{KinderlehrerStampacchia1980}), a weak form, in Section \ref{sec:model}.  System \eqref{eq:ncp} says that either a location is ice free ($s-b=0$), where the climate is, necessarily, locally ablating ($a\le 0$), or that the surface kinematical equation (SKE) holds:
\begin{equation}
\frac{\partial s}{\partial t} - \bu|_s \cdot \bn_s - a = 0.  \label{eq:ske}
\end{equation}
SKE \eqref{eq:ske} says that the (non-material) surface of the ice moves vertically according to the sum of the SME and a component of the ice velocity at the surface \cite{SchoofHewitt2013}.  Equation \eqref{eq:ske} is a statement of mass conservation at a non-material surface \cite{Aschwandenetal2012}, sometimes called the free-surface equation \cite{LofgrenAhlkronaHelanow2022} or the kinematic boundary condition\footnote{Note that equation \eqref{eq:ske} is not a boundary condition of any identifiable PDE problem.} \cite{GreveBlatter2009}.

In the current paper the SMB $a$ is necessarily assumed to be defined everywhere in $\Omega$, regardless of whether a glacier is present or not.  At an ice-free location the SMB value can be modeled using precipitation and an energy balance \cite{GreveBlatter2009}, for instance by hypothesizing an ice or snow surface and then computing the balance of snow accumulation minus the total ablation from the energy available for melt.  Because a simulated glacier needs to be able to advance into unglaciated locations, the SMB there should have the value which a glacier surface would experience at that time and location.

The continuum models considered in this paper conserve mass and momentum.   The (standard) non-shallow ice dynamics model used here is a non-sliding (e.g.~frozen) base, isothermal, shear-thinning (non-Newtonian), and incompressible Stokes problem \cite{GreveBlatter2009,JouvetRappaz2011,SchoofHewitt2013}, applied over the domain $\Lambda(t)$ defined in \eqref{eq:icydomain}.  Let $\Gamma_s(t) \subset \partial \Lambda(t)$ be the upper surface $z=s$ and $\Gamma_b(t) \subset \partial \Lambda(t)$ be the base $z=b$.  The possibility of cliffs at the ice margin is neglected, so $\partial \Lambda(t) = \overline{\Gamma_s(t)} \cup \overline{\Gamma_b(t)}$ is assumed to hold at any time.

To state the shear-thinning (Glen's) flow law, let $D\bu=(\grad \bu + \grad \bu^{\top})/2$ denote the strain rate tensor, with Frobenius norm $|D\bu| = (D\bu:D\bu)^{1/2} = \left((D\bu)_{ij} (D\bu)_{ij}\right)^{1/2}$.  The effective ice (dynamic) viscosity \cite{GreveBlatter2009} is given by a regularized formula
\begin{equation}
\nu(D\bu) = \nu_\pp \left(\tfrac{1}{2} |D\bu|^2 + \eps\right)^{(\pp-2)/2} \label{eq:glen}
\end{equation}
The exponent $1 < \pp \le 2$, often written $\pp=(1/\nn)+1$, is approximately 4/3 in practice because $\nn\approx 3$ \cite{GreveBlatter2009}.  The coefficient $\nu_\pp>0$ has $\pp$-dependent units, but $\nu(D\bu)$ has SI units $\text{kg}\,\text{m}^{-1}\,\text{s}^{-1}$.  The values of $\nn$ and $\nu_\pp$ can be determined from measured properties of ice \cite{GoldsbyKohlstedt2001,GreveBlatter2009}, including temperature, but both are assumed to be constant (isothermal) here.  Note that $\pp=2$ yields a Newtonian fluid with constant viscosity, while for $\pp < 2$ the $\eps>0$ regularization implies that $\nu(D\bu)$ is bounded above.

Assume that the density of ice $\rhoi$ and the acceleration of gravity $\bg$ are constant.  At each time $t$ the modeled glacier has velocity and pressure solving the following 3D fluid equations:
\begin{subequations}
\label{eq:stokes}
\begin{align}
- \nabla \cdot \left(2 \nu(D\bu)\, D\bu\right) + \nabla p &= \rhoi \bg && \text{within $\Lambda(t)$} \\
\nabla \cdot \bu &= 0 && \qquad \text{''} \label{eq:stokes:incomp} \\
\left(2 \nu(D\bu) D\bu - pI\right) \bn_s &= \bzero && \text{on $\Gamma_s(t)$}\label{eq:stokes:stressfreesurface} \\
\bu  &= \bzero && \text{on $\Gamma_b(t)$} \label{eq:stokes:noslide}
\end{align}
\end{subequations}
Note that boundary condition \eqref{eq:stokes:stressfreesurface} says that the sub-aerial upper surface is stress free; this should not be confused with the SKE \eqref{eq:ske}.

In summary at this point, a glacier simulation is an evolving free-surface flow, subject to a signed climate that can add or remove ice, coupled to a nonlinear Stokes problem which must be solved within an evolving, 3D icy domain.  A well-posed initial/boundary value problem for \eqref{eq:icydomain}--\eqref{eq:stokes} will require data $b(x),a(t,x)$ plus an initial surface elevation $s(0,x)$.  The solution variables are $s(t,x)$, $\bu(t,x,z)$, and $p(t,x,z)$, with $s$ defined everywhere over $[0,T]\times \Omega$, but subject to $s \ge b$, and with $\bu,p$ defined on $\Lambda(t)$ for each $t$.  The surface elevation $s$ and surface velocity $\bu|_s$ are linked by the kinematical NCP \eqref{eq:ncp}, which is the only place that a time derivative appears in the model equations.  Because the flow is very viscous \cite{Acheson1990}, the Stokes sub-model \eqref{eq:glen}--\eqref{eq:stokes} acts as an instantaneous ``algebraic'' constraint on the evolution statement in \eqref{eq:ncp}.  The coupled, infinite-dimensional problem of determining the evolving geometry of a glacier, namely system \eqref{eq:icydomain}--\eqref{eq:stokes}, is therefore both a differential algebraic equation (DAE) system \cite{AscherPetzold1998,LofgrenAhlkronaHelanow2022} and an NCP.

As already noted, one may choose to parameterize the glacier geometry by either surface elevation or thickness.  While essentially equivalent in the (continuum) problem, formulations using these different functions have different character when the bedrock is realistically rough.  Specifically, when applying the abstract estimate of Section \ref{sec:abstractestimate}, surface elevation $s$ will be preferred because of the flow-caused smoothing effect illustrated in Figure \ref{fig:giscross}.  That is, we observe that for land-based glaciers $s(t,x)$ is smoother in $x$ than the thickness $H(t,x) = s(t,x)-b(x)$ because the latter ``inherits'' the lower regularity of the (typically) eroded and/or faulted bedrock topography $b(x)$.

\begin{figure}
\begin{minipage}[t]{0.85\textwidth}
\vspace{0pt}
\includegraphics[width=\textwidth]{genfigs/giscross.pdf}
\end{minipage}
\,
\begin{minipage}[t]{0.13\textwidth}
\vspace{10pt}
\includegraphics[width=\textwidth]{genfigs/gis/gris-profile-gray.png}
\end{minipage}
\caption{A cross-section of the Greenland ice sheet at $70^\circ$N latitude (see inset).  While the ice surface $s$ is relatively smooth because of ice flow (top), the bedrock elevation $b$ is much rougher.  The corresponding ice thickness $H = s-b$ (bottom), though a valid geometry parameterization, inherits the low regularity of $b$.  (Data from \cite{Morlighemetal2017} and A.~Aschwanden, personal communication.)}
\label{fig:giscross}
\end{figure}

Glacier simulations are commonly formulated using a finite element (FE) method for the Stokes sub-problem \cite{IsaacStadlerGhattas2015,Jouvetetal2008,Pattynetal2008}, or for a shallow approximation thereof.  However, to the author's knowledge all existing non-shallow (Stokes) evolution models use a explicit time-stepping scheme for the geometry, for example as in \cite{Jouvetetal2008} or \cite{LofgrenAhlkronaHelanow2022}, with the one exception of the exploratory model in reference \cite{WirbelJarosch2020}.  In the current paper we consider implicit time steps, for the theoretical reasons which are addressed here.  One can also make a case for implicit time-stepping based on simulation performance concerns; see \cite{Bueler2023}.

FIXME: state backward Euler step SKE \eqref{eq:be:ske} here, to commit to implicit

FIXME: WHAT WE ACCOMPLISH

This paper is organized as follows.  Section \ref{sec:stokes} recalls the known theory of the Glen-law Stokes problem on a fixed domain, but we add an apparently-new bound on the surface trace of the velocity solution (Corollary \ref{cor:surfacetracebound}).  In Section \ref{sec:model} we discretize the time derivative and then reformulate the coupled problem \eqref{eq:icydomain}--\eqref{eq:stokes} as a VI weak form for an implicit (backward Euler) time step.  The key coupling term in this problem is the surface motion ($\bu|_s\cdot \bn_s$) from the SKE \eqref{eq:ske}, for which we provide a quantitative bound over a Sobolev space of surface elevation functions (Lemma \ref{lem:philipschitz}).  However, this bound is subject to Conjecture \ref{conj:a}, namely that the surface trace of velocity is Lipschitz continuous with respect to surface elevation.  Well-posedness for each implicit time-step problem is considered in Section \ref{sec:theory}, based upon a second conjecture, Conjecture \ref{conj:b}, on the coercivity of the surface motion term.  Certain physical and modeling ideas are discussed here, as context needed to understand Conjecture \ref{conj:b}.  Section \ref{sec:numerical} provides basic numerical evidence for the validity of the Conjectures, using Stokes glacier simulations in one dimension.  At this point we have in hand a mathematically-precise, time-discretized continuum model, though one with only conjectural well-posedness; see Theorem \ref{thm:stepwellposed}.  This continuum model, precisely stated here for what we believe is the first time, can be approximated by a finite element (FE) method.  In Section \ref{sec:abstractestimate} we prove an abstract FE error estimate, Theorem \ref{thm:abstractestimate}, and its several Corollaries, for general VI problems involving nonlinear operators on Banach spaces.  This new estimate, which makes coercive and Lipshitz assumptions on the operator, extends the classical case by Falk \cite{Falk1974} for bilinear forms.  In Section \ref{sec:application} we apply the estimate to the glacier model to show Theorem \ref{thm:glacierapp}, our final result.  We explain the physical significance of each term in this error estimate, and implications for how choices are made in FE-based evolving glacier simulations using Stokes dynamics.


\section{The surface velocity from a glacier Stokes problem} \label{sec:stokes}

In this Section we consider the weak form of the non-sliding, isothermal, and Glen-law Stokes sub-model \eqref{eq:glen}--\eqref{eq:stokes}.  This sub-model is applied on a fixed 3D domain $\Lambda = \Lambda(t)$, defined by \eqref{eq:icydomain}, at a fixed time $t$, and it computes the surface velocity field $\bu|_s$ which appears in NCP \eqref{eq:ncp}.  Note that the weak form of \eqref{eq:ncp} will be presented in Section \ref{sec:model} below.

Suitable function spaces for Stokes problem \eqref{eq:glen}--\eqref{eq:stokes} are well-known so long as the domain $\Lambda$ is sufficiently regular.  We will assume that the ice base $\Gamma_b\subset\partial \Lambda$, on which a Dirichlet condition $\bu=\bzero$ holds, has positive measure, and that the remaining Neumann boundary $\Gamma_s = \partial \Lambda \setminus \overline{\Gamma_b}$ is sufficiently-smooth so that a zero normal stress condition can be applied.

Let $1 < \pp \le 2$.  (Recall that $\pp=(1/\nn)+1\approx 4/3$ in viscosity formula \eqref{eq:glen}.)  The Sobolev space \cite{Evans2010} of real-valued functions with $\pp$th-power integrable first derivatives is denoted $W^{1,\pp}(\Lambda)$.  Let
\begin{equation}
\cV = W_b^{1,\pp}(\Lambda; \RR^3) \label{eq:defineV}
\end{equation}
denote the corresponding space of vector-valued functions with value (trace) zero along $\Gamma_b$.  Let $[L]>0$ be a representative \emph{horizontal} glacier dimension.  We define the norm on $\cV$ by
\begin{equation}
\|\bv\|_{\cV} = \left(\int_\Lambda |\bv|^\pp\,dx\,dz + [L]^\pp \int_\Lambda |\grad\bv|^\pp\,dx\,dz\right)^{1/\pp}. \label{eq:vnorm}
\end{equation}
Here $dx\,dz = dx_1\,dx_2\,dz$ is the 3D volume element, which will be suppressed from now on in integrals over $\Lambda$.  Note that $|\bv|$ denotes the Euclidean norm of $\bv\in\RR^3$ and $|\grad\bv|=\left((\grad\bv)_{ij} (\grad\bv)_{ij}\right)^{1/2}$ is the Frobenius norm of $\grad\bv\in\RR^{3\times 3}$.  Remark 1.2.1 in \cite{BoffiBrezziFortin2013} explains the length scaling in \eqref{eq:vnorm}, such that $\|\bv\|_{\cV}$ has consistent units.

Let $\cQ=L^{\pp'}(\Lambda)$ where $\pp'=\pp/(\pp-1)\approx 4$ is the conjugate exponent.  Define
\begin{equation}
\mathcal{M} = \cV \times \cQ \label{eq:glenstokes:mixedspace}
\end{equation}
as the space of admissible velocity and pressure pairs.  For $(\bu,p) \in \mathcal{M}$ define
\begin{equation}
F_\Lambda(\bu,p)[\bv,q] = \int_\Lambda 2 \nu(D\bu) D\bu : D\bv - p \Div\bv - (\Div\bu) q - \rhoi \bg \cdot \bv. \label{eq:glenstokes:fcnl}
\end{equation}
The (mixed) weak form of the Stokes sub-model seeks the solution $(\bu,p)$ satisfying
\begin{equation}
F_\Lambda(\bu,p)[\bv,q] = 0 \qquad \text{for all } (\bv,q) \in \mathcal{M}. \label{eq:glenstokes:weak}
\end{equation}

Jouvet and Rappaz \cite{JouvetRappaz2011} have proven that problem \eqref{eq:glenstokes:weak} is well-posed if the Neumann portion of $\partial\Lambda$ is $C^1$.  Their proof uses the equivalence of \eqref{eq:glenstokes:weak} and the minimization of a convex and coercive functional over the divergence-free subspace $\Vdiv = \{\bv\in\cV\,:\,\Div\bv=0\}$.  Our regularization in Glen law \eqref{eq:glen} differs from that in \cite{JouvetRappaz2011}, but the necessary modifications are addressed in \cite{IsaacStadlerGhattas2015}.  Note that if the weak solution is sufficiently regular then the strong form \eqref{eq:stokes} is also satisfied.

\begin{theorem}[Theorem 3.10 in \cite{JouvetRappaz2011} and Appendix A of \cite{IsaacStadlerGhattas2015}] \label{thm:stokeswellposed}  Suppose $\Lambda$ is bounded, $\partial\Lambda$ is Lipschitz, $\Gamma_s$ is $C^1$, and $\Gamma_b$ has positive measure.  Let $1<\pp\le 2$ and $\eps>0$ in \eqref{eq:glen}.  Then there exists a unique pair $(\bu,p) \in \mathcal{M}$ solving \eqref{eq:glenstokes:weak}, and $\bu\in \Vdiv$.
\end{theorem}

Our primary purpose, resumed in the next Section, is to study the glacier geometry NCP \eqref{eq:ncp}, and its weak form.  For that analysis we need to bound the surface trace $\bu|_s$ in terms of certain geometric properties of $\Lambda$.  This uses several inequalities.

\begin{lemma}[Poincar\'e's inequality; (7.44) in \cite{GilbargTrudinger2001}] \label{lem:poincare}
Under the domain assumptions of Theorem \ref{thm:stokeswellposed}, there exists a dimensionless constant $c_{\pp}(\Lambda)>0$ so that
\begin{equation}
\int_\Lambda |\bv|^\pp \le c_{\pp}(\Lambda) [L]^\pp \int_\Lambda |\grad\bv|^\pp \qquad \text{for all } \bv \in \cV. \label{eq:poincare}
\end{equation}
\end{lemma}

Assuming $[L]\ge 1$, it  follows from Lemma \ref{lem:poincare} that $\|\bu\|_{\cV}^\pp \le (c_{\pp}(\Lambda) + 1) [L]^\pp \int_\Lambda |\grad\bv|^\pp$; this is used below.
 
\begin{lemma}[Korn's inequality; set $F(x)$ to the identity in Corollary 4.1 of \cite{Pompe2003}] \label{lem:korns}
Under the same assumptions, there exists a dimensionless constant $k_{\pp}(\Lambda)>0$ so that
\begin{equation}
\int_\Lambda |\grad\bv|^\pp \le k_{\pp}(\Lambda) \int_\Lambda |D\bv|^\pp \qquad \text{for all } \bv \in \cV. \label{eq:korns}
\end{equation}
\end{lemma}

A similar $\pp$-norm Korn's inequality was earlier proven by Ting \cite{Ting1972}---see Theorem 5.12 in \cite{KikuchiOden1988}---but for zero Dirichlet conditions over all of $\partial \Lambda$.

The main idea of the following \emph{a priori}  bound is that velocity is controlled by geometric properties of the domain $\Lambda$, including the constants in the above inequalities, along with certain physical constants.  We will denote the ice volume by $|\Lambda|$.

\begin{lemma} \label{lem:stokesapriori}
Suppose $\bu\in\cV$ is the Stokes velocity solution from Theorem \ref{thm:stokeswellposed}.  Then there is $C>0$ depending on $\pp$, $\rhoi |\bg|$, $\nu_\pp$, $\eps$, $[L]$, $|\Lambda|$, $c_\pp(\Lambda)$, and $k_\pp(\Lambda)$, but not on $\bu$, so that
\begin{equation}
\|\bu\|_{\cV} \le C. \label{eq:stokesapriori}
\end{equation}
\end{lemma}

\begin{proof}
From \eqref{eq:glenstokes:weak} and $\bu \in\Vdiv$ it follows that
\begin{equation}
0= F_\Lambda(\bu,p)[\bu,p] = \int_\Lambda 2 \nu(D\bu) D\bu : D\bu - \rhoi \bg \cdot \bu.  \label{eq:stokes:substituteu}
\end{equation}
Apply Korn's inequality, the facts that $\pp>0$ and $(\pp-2)/2 \le 0$, and equation \eqref{eq:glen}:
\begin{align}
\int_\Lambda |\grad\bu|^\pp &\le k_{\pp}(\Lambda) \int_\Lambda |D\bu|^\pp \le k_{\pp}(\Lambda) \int_\Lambda \left(|D\bu|^2 + \eps\right)^{(\pp-2)/2} \left(D\bu:D\bu + \eps\right) \label{eq:stokes:startapriori} \\
	&= k_{\pp}(\Lambda) \left[\eps^{\pp/2} |\Lambda| + (2 \nu_\pp)^{-1} \int_\Lambda \nu(D\bu) D\bu:D\bu\right]. \notag
\end{align}
Thus by equation \eqref{eq:stokes:substituteu} and H\"older's inequality we have
\begin{align}
\int_\Lambda |\grad\bu|^\pp &\le k_{\pp}(\Lambda) \left[\eps^{\pp/2} |\Lambda| + (2 \nu_\pp)^{-1} \int_\Lambda \rhoi \bg \cdot \bu\right] \label{eq:stokes:workapriori} \\
	&\le k_{\pp}(\Lambda) \left[\eps^{\pp/2} |\Lambda| + (2 \nu_\pp)^{-1} \rhoi |\bg| |\Lambda|^{1/\pp'} \|\bu\|_\cV\right]. \notag
\end{align}
(This assumes that $\rhoi$ and $|\bg|$ are constant.)  By the comment after Lemma \ref{lem:poincare},
\begin{equation}
\|\bu\|_{\cV}^\pp \le (c_{\pp}(\Lambda) + 1) [L]^\pp k_{\pp}(\Lambda) \left[\eps^{\pp/2} |\Lambda| + (2 \nu_\pp)^{-1} \rhoi |\bg| |\Lambda|^{1/\pp'} \|\bu\|_\cV\right].
\end{equation}

Let $z=\|\bu\|_\cV$.  We have proved that
\begin{equation}
z^\pp \le c_0 + c_1 z
\end{equation}
for $\pp>1$ and some constants $c_i>0$.  Note that $g(y) = y^\pp - c_1 y - c_0$ is smooth with $g(0)=-c_0<0$ and $g(y) \to \infty$ as $y \to \infty$, so there exists a right-most root $\tilde y>0$, with $\tilde y = f(\pp,c_0,c_1)$ in some manner, such that since $g(z)\le 0$ we have $z \le \tilde y$.  This proves inequality \eqref{eq:stokesapriori} with $C=\tilde y$.
\end{proof}

\begin{lemma}[Trace inequality] \label{lem:trace}
Under the domain assumptions of Theorem \ref{thm:stokeswellposed}, there exists a dimensionless constant $\gamma_{\pp}(\Lambda)>0$ so that for all $\bv \in \cV$,
\begin{equation}
\int_{\Gamma_s} |\bv|^\pp \,dS \le \frac{\gamma_{\pp}(\Lambda)}{[L]} \|\bv\|_{\cV}^\pp \label{eq:trace}
\end{equation}
where $\bv$ on the left is the trace on $\Gamma_s$.  (Here $dS$ is the area element over $\partial\Lambda$.)
\end{lemma}

\begin{proof}
Theorem 5.5.1 in \cite{Evans2010} defines a trace operator $T:\cV\to L^\pp(\partial\Lambda)$, and a constant $c>0$, dependent only on $\pp$ and $\Lambda$, so that
\begin{equation}
\int_{\partial\Lambda} |T\bv|^p\,dS \le c^\pp \int_{\Lambda} |\bv|^\pp + |\grad\bv|^\pp \label{eq:tracework}
\end{equation}
for $\bv\in\cV$.  Assuming that $[L] \ge 1$ then, because $\bv=\bzero$ along $\Gamma_b$,
\begin{equation}
\int_{\Gamma_s} |T\bv|^p\,dS = \int_{\partial\Lambda} |T\bv|^p\,dS \le c^\pp \int_{\Lambda} |\bv|^\pp + [L]^p |\grad\bv|^\pp = c^\pp \|\bv\|_{\cV}^\pp. \label{eq:traceworktwo}
\end{equation}
Define $\gamma_{\pp}(\Lambda)=[L] c^\pp$, which can be shown to be dimensionless.
\end{proof}

Combining Lemmata \ref{lem:stokesapriori} and \ref{lem:trace} yields the following bound.  In using this result, recall that $\Lambda$ is determined by $s$ and $b$ (definition \eqref{eq:icydomain}).

\begin{corollary}[Surface velocity bound] \label{cor:surfacetracebound}
Suppose $\bu\in\cV$ is the Stokes velocity solution from Theorem \ref{thm:stokeswellposed}.  The norm of its trace over $\Gamma_s$ is controlled, \emph{a priori}, by $[L]$, $\gamma_{\pp}(\Lambda)$ in \eqref{eq:trace}, and $C$ in \eqref{eq:stokesapriori}:
\begin{equation}
\int_{\Gamma_s} |\bu|^\pp \,dS \le \frac{\gamma_{\pp}(\Lambda)}{[L]} C^\pp. \label{eq:surfacetracebound}
\end{equation}
\end{corollary}


\section{The implicit time-step model} \label{sec:model}

Now we consider the simplest implicit time-stepping scheme based on \eqref{eq:icydomain}--\eqref{eq:stokes}.

\subsection{Backward Euler step equations and inequalities} \label{subsec:bestrong}  Let $\{t_n\}$ be any increasing sequence of times in $[0,T]$, with $t_0=0$.  Let $\Delta t = t_n-t_{n-1}$ denote the generic step length.  Let $a^n(x)$ be the (temporal) average of the data $a(t,x)$ over $[t_{n-1},t_n]$.

Suppose that $s(x)=s^n(x)\approx s(t_n,x)$ approximates the surface elevation at time $t_n$.  Using a backward Euler implicit step \cite{AscherPetzold1998}, SKE \eqref{eq:ske} becomes
\begin{equation}
\frac{s - s^{n-1}}{\Delta t} - \bu|_{s} \cdot \bn_{s} - a^n = 0. \label{eq:be:ske}
\end{equation}
Importantly, the unknown $s=s^n$ appears both in the surface velocity $\bu|_s$ and slope $\bn_s$.  For cleaner appearance, clear the denominator and define a source term:
\begin{equation}
\ell^n(x) = s^{n-1}(x)+\Delta t\,a^n(x) = s^{n-1}(x) + \int_{t_{n-1}}^{t_n} a(t,x)\,dt. \label{eq:be:source}
\end{equation}

As with \eqref{eq:ncp}, $s=s^n$ in \eqref{eq:be:ske} actually solves a problem of free-boundary type, including admissibility, which we again state as an NCP:
\begin{subequations}
\label{eq:be:ncp}
\begin{align}
s - b &\ge 0 \label{eq:be:ncp:constraint} \\
s - \Delta t\,\bu|_s \cdot \bn_s - \ell^n &\ge 0 \label{eq:be:ncp:residualpos} \\
(s - b) \left(s - \Delta t\,\bu|_s \cdot \bn_s - \ell^n\right) &= 0 \label{eq:be:ncp:complementarity}
\end{align}
\end{subequations}
Equation \eqref{eq:be:ncp:complementarity} says that, at the solution time $t=t_n$, and almost everywhere over $\Omega$, either there is no ice ($s=b$) or equation \eqref{eq:be:ske} holds.  Generally $s$ does not solve \eqref{eq:be:ske} over all of $\Omega \subset \RR^2$; it does not solve it on bare ground where $s=b$.

The strong form NCP \eqref{eq:be:ncp} has a weak-form variational inequality (VI; \cite{Evans2010,KinderlehrerStampacchia1980}) version, derived as follows via the argument in \cite{Bueler2021conservation}, which is suited to both well-posedness theory and finite element (FE) analysis.  The precise Banach space $\cX$ of surface elevations here is unknown, so for now this space is abstract.  However, the admissible surface elevations come from a convex and closed subset
\begin{equation}
\cK = \left\{r \in\cX\,:\,r|_{\partial\Omega}=b|_{\partial\Omega} \text{ and } r \ge b\right\}.  \label{eq:be:admissible}
\end{equation}
Noting that the fixed boundary condition is included in the definition of $\cK$, if the glacier does not reach $\partial\Omega$ then this condition is of little importance.

Suppose that $s \in \cK$ is a sufficiently-regular solution of NCP \eqref{eq:be:ncp}.  Let $\Omega_I$ be the (measurable) subset of $\Omega$ on which constraint \eqref{eq:be:ncp:constraint} is inactive, i.e.~where glacier ice is present: $\Omega_I = \{x\,:\,s(x)>b(x)\}$.  From \eqref{eq:be:ncp:complementarity}, integration over $\Omega_I$ shows that
\begin{equation}
\int_{\Omega_I} \left(s - \Delta t\,\bu|_s \cdot \bn_s - \ell^n\right)\,(r-s) = 0.  \label{eq:inactivetruth}
\end{equation}
for any $r\in\cK$.  On the other hand, suppose $\Omega_A = \{x \in \Omega \,:\,s(x)=b(x)\}$ is the active (ice-free) region for constraint \eqref{eq:be:ncp:constraint}.  Observe that \eqref{eq:be:ncp:residualpos} says that $b-\ell^n = s - \Delta t\,\bu|_s \cdot \bn_s - \ell^n \ge 0$ on $\Omega_A$.\footnote{Otherwise, physically speaking, and assuming continuity of $b$, $s^{n-1}$, and $a^n$, a glacier would still be present somewhere in $\Omega_A$, or would have appeared during the time step, a contradiction.  Note the role of extension by zero \eqref{eq:defineus} here.}  Note that $r-s=r-b\ge 0$ on $\Omega_A$ if $r\in\cK$.  Because $b-\ell^n \ge 0$ and $r-s\ge 0$ on $\Omega_A$, integration yields this inequality:
\begin{equation}
\int_{\Omega_A} \left(s - \Delta t\,\bu|_s \cdot \bn_s - \ell^n\right)\,(r-s) = \int_{\Omega_A} \left(b - \ell^n\right)\,(r-b) \ge 0.  \label{eq:activetruth}
\end{equation}
Almost everywhere on $\Omega$, either land is glacier covered (within $\Omega_I$) or it is ice-free ($\Omega_A$).  Addition of \eqref{eq:inactivetruth} and \eqref{eq:activetruth} therefore justifies the following VI for $s \in \cK$:
\begin{equation}
\int_\Omega \left(s - \Delta t\,\bu|_s \cdot \bn_s\right)\,(r-s) \ge \int_\Omega \ell^n \,(r-s) \quad \text{for all } r \in \cK. \label{eq:be:viearly}
\end{equation}
This integral inequality is known to be true in advance of any knowledge of which is the ice-covered part of $\Omega$.
	
\subsection{Surface motion map} \label{subsec:surfacemotion}  Now, the well-posedness of the weak-form Stokes problem \eqref{eq:glenstokes:weak}, over a 3D domain $\Lambda$, and the surface trace bound in Corollary \ref{cor:surfacetracebound}, allows us to create a well-defined map from an admissible surface elevation $s$ to the corresponding surface velocity solution $\bu|_s(x)$.  The map is defined via definition \eqref{eq:icydomain}, followed by the solution of \eqref{eq:glenstokes:weak} over $\Lambda$, evaluation of the trace of $\bu$ along $\Gamma_s$ (Corollary \ref{cor:surfacetracebound}), and then definition \eqref{eq:defineus}.  For this map to be well-defined, $s$ must admissible ($s\in\cK$) and it must be sufficiently regular so that these steps are justified.

We call $\Phi(s) = - \bu|_s\cdot \bn_s$ the \emph{surface motion map}.  It maps the scalar surface elevation $s$ to a scalar term in the SKE \eqref{eq:be:ske}.  Constructing a preliminary bound for $\Phi$ will help to identify a Banach space $\cX$ in which to seek admissible solutions $s$.

Recall from Section \ref{sec:stokes} that $\pp=1+1/\nn \approx 4/3$.  Let $\pp'=\pp/(\pp-1) \approx 4$ be the conjugate exponent.  Let $[L]>0$ be the horizontal scale used in the $\cV$ norm (see \eqref{eq:vnorm}).  For $\rr\ge 1$ and $q\in W^{1,\rr}(\Omega)$ we define
\begin{equation}
\|q\|_{W^{1,\rr}} = \left(\int_\Omega |q|^\rr\,dx + [L]^\rr \int_\Omega |\grad q|^\rr\,dx\right)^{1/\rr}. \label{eq:norm:Omega}
\end{equation}

\begin{lemma}[Preliminary bound on $\Phi(s)$] \label{lem:phibound:early}  Let $\Omega \subset \RR^2$ be a bounded domain.  Suppose $2 \le \rr \le \infty$, and assume $s\in W^{1,\rr}(\Omega)$ is admissible ($s\ge b$).  With $\Lambda$ defined by \eqref{eq:icydomain}, assume that the hypotheses of Theorem \ref{thm:stokeswellposed} and Corollary \ref{cor:surfacetracebound} apply, which also shows that $\Phi(s)$ is a well-defined measurable function.  Then there is a constant $C>0$, depending on $|\Omega|$, $[L]$, $\Lambda$, and $\|s\|_{W^{1,\rr}}$, so that
\begin{equation}
\left|\int_\Omega \Phi(s) q\,dx\right| = \left|\int_\Omega \bu|_s\cdot \bn_s q\,dx\right| \le C\, \|q\|_{W^{1,\rr}} \qquad \text{for all } q\in W^{1,\rr}(\Omega).  \label{eq:phibound:early}
\end{equation}
\end{lemma}

\begin{proof}  Observe that $dS = |\bn_s|\,dx = \sqrt{1+|\grad s|^2}\,dx$ is the surface area element for $\Gamma_s \subset \partial \Lambda$.  Apply the triangle inequality, and H\"older's inequality twice:
\begin{align}
\left|\int_\Omega \Phi(s) q\,dx\right| &\le \int_\Omega \big|\bu|_s\big| |\bn_s| |q|\,dx = \int_\Omega \big|\bu|_s\big| |\bn_s|^{1/\pp} |\bn_s|^{1/\pp'} |q|\,dx \label{eq:phibound:zero} \\
    &\le \left(\int_\Omega \big|\bu|_s\big|^\pp |\bn_s|\,dx\right)^{1/\pp} \left(\int_\Omega |\bn_s| |q|^{\pp'} \,dx\right)^{1/\pp'} \notag \\
    &\le \left(\int_{\Gamma_s} |\bu|^\pp \,dS\right)^{1/\pp} \left(\int_\Omega |\bn_s|^\rr \,dx\right)^{1/(\pp'\rr)} \left(\int_\Omega |q|^{\pp'\rr'} \,dx\right)^{1/(\pp'\rr')}. \notag
\end{align}
If $C_1$ is the bound from \eqref{eq:surfacetracebound} then we now have
\begin{equation}
\left|\int_\Omega \Phi(s) q\,dx\right| \le C_1^{1/\pp} \left(\int_\Omega \left(1+|\grad s|^2\right)^{\rr/2}\,dx\right)^{1/(\pp'\rr)} \|q\|_{L^{\pp'\rr'}}.
\end{equation}
Note that if $\alpha\ge 0$ then $(1+\alpha)^{\rr/2} \le 2^{(\rr-2)/2} (1+\alpha^{\rr/2})$, so
\begin{align}
\left|\int_\Omega \Phi(s) q\,dx\right| &\le C_1^{1/\pp} \left(2^{(\rr-2)/2} \int_\Omega 1 + |\grad s|^\rr\,dx\right)^{1/(\pp'\rr)} \|q\|_{L^{\pp'\rr'}} \label{eq:phibound:one} \\
  &\le C_2 \left(|\Omega| + [L]^{-\rr}\|s\|_{W^{1,\rr}}^\rr\right)^{1/(\pp'\rr)} \|q\|_{L^{\pp'\rr'}}. \notag
\end{align}
Since $\pp'\rr' < \infty$ and $\pp'\rr' \ge \pp' \ge 2$, by Sobolev's inequality\footnote{For example, apply Theorem 8.8 from \cite{LiebLoss1997} using $n=2$, $k=m=1$, $p=\rr$, and $q=\pp'\rr'$.} we also have $\|q\|_{L^{\pp'\rr'}} \le C_3 \|q\|_{W^{1,\rr}}$, which yields \eqref{eq:phibound:early}.
\end{proof}

The key assumption in Lemma \ref{lem:phibound:early} is that $s\in W^{1,\rr}(\Omega)$ implies that the domain $\Lambda$ is nice enough so that Corollary \ref{cor:surfacetracebound} gives a finite bound.  The key conclusion is that $\Phi(s) \in \left(W^{1,\rr}(\Omega)\right)'$ is in the dual space.  This conclusion will be critical in analyzing the weak form of NCP \eqref{eq:be:ncp}.

For $2 \le \rr \le \infty$ the conjugate exponent $\rr'=\rr/(\rr-1)$ satisfies $1 \le \rr' \le 2$.  Now we conjecture that for some $\rr >2$, i.e.~subject to a strict inequality,\footnote{The reason for requiring $\rr>2$ is seen in Lemma \ref{lem:philipschitz}.  It would seem to be only a technical requirement.  Section \ref{sec:application} demonstrates well-behaved results using the convenient exponent $\rr=2$.} the $L^{\rr'}$-norm of the surface trace $\bu|_s$ is Lipschitz as a function of $s \in W^{1,\rr}(\Omega)$.  Note that we do not assume that $\bu|_s$ is continuous, but it is a measurable function defined over all of $\Omega$.

\begin{conjecture} \label{conj:a}  There exists $2 < \rr \le \infty$ with the following two properties:  \emph{(i)} If $s\in W^{1,\rr}(\Omega)$ is admissible ($s\ge b$ and $s=b$ on $\partial\Omega$) then the conclusion of Theorem \ref{thm:stokeswellposed} applies, giving a well-defined surface velocity $\bu|_s$.  \emph{(ii)} If also $r\in W^{1,\rr}(\Omega)$ is admissible, yielding different values $\bu|_r$, then there exists $\CA>0$, independent of $s$ and $r$, such that
\begin{equation}
\big\|\bu|_r - \bu|_s\big\|_{L^{\rr'}} \le \CA \|r-s\|_{W^{1,\rr}}. \label{eq:ulipschitz}
\end{equation}
\end{conjecture}

From now on we will assume Conjecture \ref{conj:a} holds for some $2 < \rr \le \infty$.  We define
\begin{equation}
\cX = W^{1,\rr}(\Omega). \label{eq:defineX}
\end{equation}
Definition \eqref{eq:be:admissible} describes the closed and convex admissible subset $\cK \subset \cX$.  If $s\in\cK$ then, by Conjecture \ref{conj:a}, $\bu|_s\cdot\bn_s$ is a measurable, real-valued function on $\Omega$.  For $q\in\cX$ we will, from now on, write $\Phi(s)$ as a linear functional in $\cX'$:
\begin{equation}
\Phi(s)[q] = -\int_\Omega \bu|_s\cdot\bn_s\,q\,dx. \label{eq:definePhi}
\end{equation}
(Lemma \ref{lem:phibound:early} has already shown that $\Phi$ maps from $\cK$ to the (topological) dual space $\cX'$.)  Note that $\Phi(s)[q]$ is nonlinear in $s$ but linear in $q$.  The next Lemma simply proves that $\Phi$ is Lipschitz-continuous if we assume Conjecture \ref{conj:a}.

\begin{lemma} \label{lem:philipschitz}  Suppose that Conjecture \ref{conj:a} holds.  Fix $b \in \cX$ and use definition \eqref{eq:be:admissible} to define $\cK$.  The map $\Phi:\cK\to\cX'$ is Lipschitz on bounded subsets of $\cK$, that is, for each $R>0$ there is $C(R)>0$ so that if $r,s\in B_R \cap \cK = \{t\in \cK\,:\,\|t\|_{\cX} \le R\}$ and $q\in\cX$ then
\begin{equation}
\Big|\Phi(r)[q] - \Phi(s)[q]\Big| \le C(R)\, \|r-s\|_{\cX} \|q\|_{\cX}  \label{eq:philipschitz}
\end{equation}
\end{lemma}

\begin{proof}  Suppose $s,r\in\cK$.  Add and subtract $\bu|_s \cdot \bn_r$, and apply triangle inequalities, including $|\bn_r|=\left(1+|\grad r|^2\right)^{1/2} \le 1 + |\grad r|$:
\begin{align}
\big|\Phi(r)[q] - \Phi(s)[q]\big| &\le \int_\Omega \Big|\bu|_r\cdot \bn_r - \bu|_s \cdot \bn_s\Big| |q|\,dx \\
    &\le \int_\Omega \big|\bu|_r - \bu|_s\big| |\bn_r| |q|\,dx + \int_\Omega \big|\bu|_s\big| |\bn_r-\bn_s| |q|\,dx \notag \\
    &\le \int_\Omega \big|\bu|_r - \bu|_s\big| |q|\,dx + \int_\Omega \big|\bu|_r - \bu|_s\big| |\grad r| |q|\,dx \notag \\
    &\qquad\qquad + \int_\Omega \big|\bu|_s\big| |\grad r-\grad s| |q|\,dx \notag
\end{align}
Because $\rr>2$, Sobolev's inequality gives $\|q\|_{L^{\rr}} \le c \|q\|_\cX$ and $\|q\|_{L^\infty} \le c_\infty \|q\|_\cX$ for some $c,c_\infty>0$.  By applying H\"older's inequality to each integral we have
\begin{align}
\int_\Omega \big|\bu|_r - \bu|_s\big| |q|\,dx &\le \big\|\bu|_r - \bu|_s\big\|_{L^{\rr'}} \|q\|_{L^{\rr}}, \label{eq:philipschitz:1} \\
\int_\Omega \big|\bu|_r - \bu|_s\big| |\grad r| |q|\,dx &\le \left(\int_\Omega \big|\bu|_r - \bu|_s\big|^{\rr'} |q|^{\rr'}\, dx\right)^{1/\rr'} \|\grad r\|_{L^\rr} \label{eq:philipschitz:2} \\
    &\le [L]^{-1} \big\|\bu|_r - \bu|_s\big\|_{L^{\rr'}} \|r\|_{\cX} \|q\|_{L^\infty}, \notag \\
\int_\Omega \big|\bu|_s\big| |\grad r-\grad s| |q|\,dx &\le \left(\int_\Omega \big|\bu|_s\big|^{\rr'} |q|^{\rr'}\, dx\right)^{1/\rr'} \|\grad r- \grad s\|_{L^\rr}  \label{eq:philipschitz:3} \\
    &\le [L]^{-1} \big\|\bu|_s - \bzero\big\|_{L^{\rr'}} \|r-s\|_{\cX} \|q\|_{L^\infty}. \notag
\end{align}
Note that $\bu|_b=\bzero$, that is, when there is no glacier.  Now apply Conjecture \ref{conj:a} to \eqref{eq:philipschitz:1}, \eqref{eq:philipschitz:2}, and \eqref{eq:philipschitz:3}:
\begin{equation}
\big|\Phi(r)[q] - \Phi(s)[q]\big| \le \CA \left[c + c_\infty [L]^{-1} \left(\|r\|_{\cX} + \|s - b\|_{\cX}\right)\right] \|r-s\|_{\cX} \|q\|_{\cX}.
\end{equation}
Assume $s,r,b\in B_R\cap \cK$.\footnote{This requires $R \ge \|b\|_\cX$, in particular.}  Then, by the triangle inequality, \eqref{eq:philipschitz} follows with $C(R) = \CA \left(c + 3 c_\infty [L]^{-1} R\right)$.
\end{proof}

\subsection{Weak form} \label{subsec:weakform} With the above tools we can now define an operator which puts the backward Euler time step VI \eqref{eq:be:viearly} into a mathematically-precise weak form.  If $s\in\cK$ and $q\in\cX$ then we define $F_{\Delta t}:\cK\to\cX'$:
\begin{equation}
F_{\Delta t}(s)[q] = \Delta t\,\Phi(s)[q] + \int_\Omega s q = \int_\Omega \left(s - \Delta t\, \bu|_s \cdot \bn_s\right) q.  \label{eq:be:Fdefine}
\end{equation}
If Conjecture \ref{conj:a} holds then Lemma \ref{lem:philipschitz} says that this operator is well-defined and Lipschitz on bounded subsets, with a constant which depends linearly on $\Delta t$.  We also assume that the source term $\ell^n$, defined in \eqref{eq:be:source}, is in $\cX'$.  Then we will seek $s = s^n \in \cK$ so that
\begin{equation}
\boxed{F_{\Delta t}(s)[r-s] \ge \ell^n[r-s] \quad \text{for all } r \in \cK.} \label{eq:be:vi}
\end{equation}
This VI, which rewrites \eqref{eq:be:viearly}, is the final weak form of the implicit time-step problem.  The reader should keep in mind its strong-form NCP \eqref{eq:be:ncp} as well.

Instead of the specific Lipschitz statement \eqref{eq:philipschitz}, it would suffice for well-posedness purposes to know  that $\Big|\Phi(r)[q] - \Phi(s)[q]\Big| \le C(R)\, (\|r-s\|_{\cX})^\omega \|q\|_{\cX}$ for any $\omega>0$.  This is sufficient continuity for the well-posedness theorem in Section \ref{sec:theory}, but the finite element error theorem in Section \ref{sec:abstractestimate} needs \eqref{eq:philipschitz} with $\omega=1$.

If the horizontal components of the surface velocity are differentiable then one may revise definition \eqref{eq:be:Fdefine}.  First write $\bu=(u,v,w)$ in cartesian coordinates, and define $\bU=(u,v)$.  Assuming $\bU|_s=\bzero$ along $\partial\Omega$, e.g.~supposing the glacier is strictly inside the open domain $\Omega$, integrating \eqref{eq:be:Fdefine} by parts gives
\begin{equation}
F_{\Delta t}(s)[q] = \int_\Omega \left(s - \Delta t\, w|_s\right) q - \Delta t\, \Div\left(\bU|_s\, q\right).  \label{eq:be:Fdefine:alt}
\end{equation}
However, this seems not to be a better trade-off of regularity between $\bu|_s$ and $s$.  We have proven in Section \ref{sec:stokes} that $\bu|_s$ is an $L^\pp(\Gamma_s)$ function (Corollary \ref{cor:surfacetracebound}), but we know of no proof that it is more regular than that.  On the other hand we are hypothesizing that $\grad s$ is a well-defined function in $L^\rr$ because $s\in\cX=W^{1,\rr}(\Omega)$.  We will use definition \eqref{eq:be:Fdefine} from now on.

\subsection{Local consequences of the weak form} \label{subsec:viconsequences}  Since $\rr>2$, the Sobolev inequality shows $b\in\cX$ and $s\in\cK$ are continuous over $\bar\Omega$.  This fact allows us to derive certain consequences using admissible functions of the form $r=s+\phi$, for $\phi$ supported in portions of $\Omega$.  We show three illustrations.

\smallskip
\emph{Interior condition.}  Suppose $\phi(x) \in C(\Omega)$ is supported in the open (inactive) set $\Omega_I=\{x\in\Omega\,:\,s(x)>b(x)\}$, and let $r=s+\eps\phi$ for $\eps$ real.  By the compactness of the support of $\phi$, and the continuity of $b$ and $s$, we have $r\in\cK$ for $\eps$ sufficiently close to zero.  Substitution into \eqref{eq:be:vi} gives
\begin{equation}
\eps\left(F_{\Delta t}(s) - \ell^n\right)[\phi] \ge 0 \qquad \text{for $\phi$ supported in $\Omega_I$.}
\end{equation}
Because this holds for all $\eps$ close to zero, of either sign, it follows by considering all such $\phi$ that
\begin{equation}
s - \Delta t\, \bu|_s \cdot \bn_s - s^{n-1} - \Delta t\,a^n = 0 \qquad \text{within } \Omega_I.
\end{equation}
That is, the backward Euler step SKE \eqref{eq:be:ske} holds where there is ice.

\smallskip
\emph{Ice-free condition.}  Suppose $\phi(x) \in C(\Omega)$ is nonnegative and supported in the interior of the active set $\Omega_A=\{x\in\Omega\,:\,s(x)=b(x)\}$.  Let $r=s+\phi$, so $r\in\cK$.  Substitution into \eqref{eq:be:vi} gives
\begin{equation}
\left(F_{\Delta t}(s) - \ell^n\right)[\phi] \ge 0 \qquad \text{for $\phi\ge 0$ in the interior of $\Omega_A$.}
\end{equation}
By considering all such $\phi$, and assuming that $s^{n-1}$ and $a^n$ are also continuous, and assuming that $\Omega_A$ is the closure of its interior, then
\begin{equation}
b-\ell^n = b - s^{n-1} - \Delta t\,a^n \ge 0 \qquad \text{in } \Omega_A.
\end{equation}

The fact that $b-\ell^n\ge 0$ in ice-free areas will be important in Section \ref{sec:application}.  The same fact is also an upper bound on the SMB:
\begin{equation}
a^n \le \frac{b - s^{n-1}}{\Delta t} \qquad \text{in } \Omega_A. \label{eq:viconsequence:icefree}
\end{equation}
This upper bound is nonpositive because $s^{n-1}\in\cK$.  Thus ablation must occur in ice-free areas.  In fact, inequality \eqref{eq:viconsequence:icefree} shows that where $s$ indicates no ice the SMB is sufficiently negative to remove any ice present at the start of the time step \cite{Bueler2021conservation}.

\smallskip
\emph{Margin slope in 1D.}  The third illustration yields an apparently novel observation.  Assume $\Omega \subset \RR^1$ is a one-dimensional interval, and consider a glacier margin at $x=x_m$ in its interior.  Suppose ice is present on $x<x_m$ and absent where $x>x_m$ (as in Figure \ref{fig:margins}).  As before, $s,b,s^{n-1}$ are continuous on $\Omega$.  Assume that $a^n$, thus $\ell^n$, is also continuous.  Assume that the surface velocity components $u|_s$ and $w|_s$ are continuous on $x<x_m$, with finite limits $u_-$ and $w_-$ as $x\to x_m$.

Construct a hat function $\phi\in\cX$ which is nonnegative, piecewise-linear, and supported in $[x_m-\delta,x_m+f\delta]$ for $\delta>0$ and $0<f<1$.  Specifically, set $\phi(x_m-\delta)=\phi(x_m+f\delta)=0$ and $\phi(x_m)=1$, and linear between.  Then $r=s+\phi\in\cK$, and substitution into \eqref{eq:be:vi} gives
\begin{equation}
\int_{x_m-\delta}^{x_m} \left(s + \Delta t\, (u s' - w) - \ell^n\right)\phi + \int_{x_m}^{x_m+f\delta} \left(b - \ell^n\right)\phi \ge 0
\end{equation}
The second integral is $O(f\delta)$, and it plays no further role.  Applying the mean value theorem to the first integral,
\begin{equation}
\frac{\delta}{2} \big[s(\xi_1) - \Delta t\,w(\xi_1) - \ell^n(\xi_1)\big] + \Delta t\,u(\xi_2) \int_{x_m-\delta}^{x_m} s'\phi + O(f\delta) \ge 0 \label{eq:viconsequence:margin:early}
\end{equation}
for $\xi_i \in (x_m-\delta,x_m)$.  Now, flow toward the margin ($u_->0$) or stagnant ice there ($u_-=0$) would seem to be the reasonable cases, even at a margin undergoing retreat.  If $u_->0$ then, taking the $f\to 0$ limit and then the $\delta \to 0$ limit, and noting $s(x_m)=b(x_m)$, yields a lower bound:
\begin{equation}
s'_- \ge \frac{(s^{n-1}(x_m)-b(x_m))/\Delta t + w_- + a^n(x_m)}{u_-}. \label{eq:viconsequence:margin}
\end{equation}
Here $s'_- = \liminf_{\delta \to 0^+} \frac{2}{\delta} \int_{x_m-\delta}^{x_m} s'\phi$ by definition; this is a kind of marginal trace of the surface slope.

By inequality \eqref{eq:viconsequence:margin}, the surface slope at the margin cannot be unbounded below.  That is, if in 1D the velocity components have finite limits at the margin, and if the flow is toward the margin, then the surface must have finite slope; the margin must be wedge-shaped.  (Compare Subsection \ref{subsec:margin}.)  On the other hand, it is not known that the velocity limits exist at the marginal point, and numerical results (Section \ref{sec:numerical}) do not give much confidence that the vertical velocity limit $w_-$, in particular, exists.

Consider inequality \eqref{eq:viconsequence:margin:early} in the other case $u_-=0$.  Here one can show that the SMB at the margin solves a similar inequality to \eqref{eq:viconsequence:icefree}, but with a contribution from the limiting vertical velocity; the details are left to the reader.


\section{Theoretical considerations for VI problem~(\ref{eq:be:vi})} \label{sec:theory}

There is general agreement among glaciologists that a non-sliding and isothermal glacier, modeled as a very-viscous and incompressible fluid, satisfies the conditions of NCP \eqref{eq:ncp}, or, specifically for a backward Euler time step, NCP \eqref{eq:be:ncp}.  For example, numerical ice sheet models \cite{IsaacStadlerGhattas2015,Pattynetal2008,WirbelJarosch2020} are based upon the Stokes model \eqref{eq:stokes}, and the SKE \eqref{eq:ske} is a standard way for surface geometry to evolve \cite{GreveBlatter2009,SchoofHewitt2013}.  The idea that positive SMB at a given location implies the existence of glacier ice is not controversial.  Likewise, ice-free areas are understood to correspond to negative SMB, at least supposing that SMB is both continuous in time and space, and that any initial accumulation (positive SMB) immediately becomes glacier ice by definition.

However, the numerical error bounds proven later in Sections \ref{sec:abstractestimate} and \ref{sec:application} need to compare a finite element (FE) surface elevation with the unique solution of the continuum problem, and therefore VI problem \eqref{eq:be:vi} must be well-posed.  Despite the theoretical progress made in Sections \ref{sec:stokes} and \ref{sec:model}, no results known to the author prove well-posedness for \eqref{eq:be:vi}, nor for any similar glacier geometry problem based on Stokes dynamics, so we make Conjecture \ref{conj:b} in subsection \ref{subsec:conjecture} below.  We build up to this Conjecture using comparative cases and physical reasoning.

\subsection{The no-flow problem is well-posed over $L^2$} \label{subsec:noflow}   Consider a glacier that does not flow.  The backward Euler time-step VI problem \eqref{eq:be:vi} then reduces to determining the geometry according only to the SMB and the prior geometry, a well-posed problem over $L^2(\Omega)$.  To see this precisely, let $F^{\bzero}_{\Delta t}(s)[q] = \int_\Omega sq$, which sets $\bu|_s=\bzero$ in \eqref{eq:be:Fdefine}.  Assuming that definition \eqref{eq:be:source} yields $\ell^n \in L^2(\Omega)$, there exists a unique solution $s \in \cK_{L^2} = \left\{r\in L^2(\Omega)\,:\,r \ge b\right\}$ of the no-flow VI problem
\begin{equation}
F^{\bzero}_{\Delta t}(s)[r-s] \ge \ell^n[r-s] \quad \text{for all } r \in \cK_{L^2}.
\end{equation}
The solution is by truncation \cite[section II.3]{KinderlehrerStampacchia1980}:
\begin{equation}
s = \max\{b, \ell^n\} = \max\{b, s^{n-1} + \Delta t\,a^n\} \qquad (\text{\emph{no flow}}).
\end{equation}
Thus, in the absence of flow, the new surface is raised or lowered according to the (pointwise) integral of the SMB rate, subject to the restriction that it not go below the bed.

\subsection{The explicit problem may not be well-posed after the first time step} \label{subsec:explicit}   The explicit time-step VI problem corresponding to \eqref{eq:be:vi} has the same mathematical character as no flow at all.  Suppose $s^{n-1}$ is admissible and sufficiently regular so that $\bn_{s^{n-1}}$ is well-defined, and so that the weak-form Stokes problem \eqref{eq:glenstokes:weak} is well-posed over the domain $\Lambda^{n-1}$.  The explicit operator
\begin{equation}
F^{\text{e}}_{\Delta t}(s)[q] = \int_\Omega \left(s - \Delta t\, \bu|_{s^{n-1}} \cdot \bn_{s^{n-1}}\right) q  \label{eq:explicitFdefine}
\end{equation}
then arises by applying forward Euler to SKE \eqref{eq:be:ske}.  The corresponding VI problem
\begin{equation}
F^{\text{e}}_{\Delta t}(s)[r-s] \ge \ell^n[r-s]
\end{equation}
is well-posed in $L^2(\Omega)$, and again it is solved for $s \in \cK_{L^2}$ by truncation:
\begin{equation}
s = \max\{b, s^{n-1} + \Delta t\, \bu|_{s^{n-1}} \cdot \bn_{s^{n-1}} + \Delta t\,a^n\} \qquad (\text{\emph{explicit step}}). \label{eq:explicits}
\end{equation}

However, formula \eqref{eq:explicits} would seem to leave no regularity for the next step.  The derivatives in $\bn_{s}$, the trace evaluation $\bu|_{s}$, and the truncation itself all (generally) reduce regularity of $s$ relative to $s^{n-1}$.  It would seem from what we know about well-posed Stokes problems that the function $s$ defined by \eqref{eq:explicits} is not regular enough, i.e.~not sufficiently differentiable in space, so as to serve as the surface elevation at the start of the next time step.  That is, it is not clear that $s$ from \eqref{eq:explicits} defines a sufficiently-smooth domain $\Lambda$ so that the (weak) Stokes problem \eqref{eq:glenstokes:weak} is well-posed.

Re-stating this situation generously is worthwhile because explicit time-stepping simulations using Stokes dynamics are common in glacier science:  There is no known mathematical reason why explicit schemes should converge under temporal refinement.  This is because the continuum limit of the time-discretized solution, resulting from taking multiple explicit time steps and applying truncation \eqref{eq:explicits}, is not even conjecturally clear.  Convergence and stability being intimately related, this situation aligns with the very incomplete understanding, in the literature, of which explicit refinement paths are conditionally stable \cite[and references therein]{Bueler2023,Chengetal2017,LofgrenAhlkronaHelanow2022}.  If it were shown that the parabolic  VI problem \cite{Glowinski1984}, corresponding to all of the strong form conditions \eqref{eq:icydomain}--\eqref{eq:stokes}, were well-posed, then convergence of both implicit and explicit schemes would become generally clearer.

The regularity of the surface elevation solution might be improved by use of semi-implicit Euler time-stepping  \cite{LofgrenAhlkronaHelanow2022}, which uses $s$ in the surface normal in \eqref{eq:explicitFdefine}: $\bn_{s^{n-1}} \to \bn_s$.  However, \cite{LofgrenAhlkronaHelanow2022} demonstrate that this change, by itself, has small effect on stability, and we see no reason why it would address the well-posedness concern.

\subsection{The problem is not of advection type} \label{subsec:notadv}  It is common in the literature to regard the SKE \eqref{eq:ske} as an advection, based on its appearance,\footnote{Often written $\frac{\partial s}{\partial t} + u \frac{\partial s}{\partial x} + v \frac{\partial s}{\partial y} = a + w$, where $(u,v,w)$ denotes the surface velocity \cite{GreveBlatter2009,SchoofHewitt2013}.} but this is far from the whole truth.  Mathematically, it is not an advection because the surface velocity is not determined externally to \eqref{eq:ske}.  Instead, coupled stress balance equations, over the domain determined by the solution $s$, supply the ``advecting velocity'' $\bu|_s$.  Physically, glacier geometry solves a gravity-driven, free-surface, and viscous flow problem, so ice flows predominantly downhill.  The surface therefore typically responds to local surface perturbations with negative feedback; the flow response to a raised surface bump tends to remove the bump, and likewise for an indentation.  Thus the response is diffusive, not advective, in the large.  Such diffusive response explains the relatively smooth large-scale appearance of actual surface elevations (Figure \ref{fig:giscross}).

In the shallow ice approximation (SIA) this diffusive character is precise.  For the non-sliding and isothermal SIA model \cite{GreveBlatter2009,JouvetBueler2012}, SKE \eqref{eq:ske} is seen to be a nonlinear diffusion:
\begin{equation}
\frac{\partial s}{\partial t} - \Gamma (s-b)^{\nn+1} |\grad s|^{\nn+1} - \Div \left(\frac{\nn+1}{\nn+2} \Gamma (s-b)^{\nn+1} |\grad s|^{\nn-1} \grad s\right) - a = 0  \label{eq:ske:sia}
\end{equation}
% \Gamma = (2/(n+1)) A (\rho g)^\nn
Here $\nn\approx 3$ is Glen's exponent \cite{GreveBlatter2009}, and the constant $\Gamma>0$ can be computed from $\nu_\pp$ in \eqref{eq:glen}.  The highest-order divergence term in \eqref{eq:ske:sia}, arising from the ``$w$'' vertical velocity term in the SKE, acts as negative feedback on surface bumps.

Well-posedness results are partially known for SIA models, though with ice geometry usually parameterized by the ice thickness.  For $H=s-b$, let $\tilde{\cK} = \{H\ge 0\}$ be the thickness constraint set.   For steady SIA models with smooth bed elevations, existence is known with $\eta = H^{2\qq/(\qq-1)} \in W^{1,\qq}(\Omega)$ for $\qq=\nn+1$ \cite{JouvetBueler2012}.  (However, the set of functions whose given power is in a Sobolev space is not generally even a vector space.)  In time-dependent cases both existence and uniqueness is known, but only when the bedrock is flat \cite{Calvoetal2003,PiersantiTemam2023}.

The regularity exhibited by solutions to \eqref{eq:ske:sia} will not fully persist in solutions to the Stokes-based VI problem \eqref{eq:be:vi}.  The surface response in a Stokes model is known to have a significantly different small-wavelength limit \cite{Pattynetal2008}, though longer wavelengths are handled correctly by the SIA.

\subsection{The problem is not of optimization type} \label{subsec:notopt}  FIXME porous medium character, i.e.~coercivity blocked by thickness (power) scaling of operator; calculation shows weak form op $f(u)[v] = \int u \grad u\cdot \grad v$ does not have symmetry $f'(u)[v,w] = f'(u)[w,v]$ which would arise if $f(u)=j'(u)$ for objective $j(u) \in \RR$

FIXME physically, thin ice does not flow much regardless of surface slope, which is reflected in $(s-b)^4$ coefficient in the SIA surface velocity formula

\subsection{Margin/terminus shape obscures the choice of Sobolev space} \label{subsec:margin}  The theoretically-predicted shape of the grounded margin of a glacier is not so clear (Figure \ref{fig:margins}), and this makes it difficult to speculate on which Sobolev spaces might allow well-posedness.  The SIA theory suggests root-type (fractional-power) shapes \cite{Bueleretal2005} for the marginal surface elevation, with unbounded gradients.  By contrast, a ``wedge'' margin shape with a bounded gradient has been hypothesized \cite[for example]{EchelmeyerKamb1986}, which would allow $s\in W^{1,\rr}(\Omega)$.  However, reality is more complicated.  In the vicinity of any steep ice margin, especially on steep bedrock features, real glacier ice can generate overhangs which violate the assumption that a surface elevation function is meaningful, and similarly the bedrock can overhang.  Fractures, crevasses, and cliffs are commonly found in glacier margins, but modeling such features escapes from the viscous fluid paradigm considered here.

\begin{figure}[ht]
\begin{center}
\includegraphics[width=0.8\textwidth]{figs/margins.jpg}
\end{center}
\caption{FIXME REDRAW.  In which Sobolev space should we seek the surface elevation function?  This question relates to the theoretically-expected shape of the ice margin.  The shallow ice theory yields a fractional power shape (left).  Other glacier models hypothesize a ``wedge'' shape (center).  On the other hand, actual glacier margins often have overhangs, crevasses, and cliffs (right).}
\label{fig:margins}
\end{figure}

Because margins are small features compared to the overall scale of glaciers and ice sheets, almost all modeling literature ignores overhangs and assumes instead that surface and bed elevation functions are well-defined; see \cite{IsaacStadlerGhattas2015,Jouvetetal2008,LofgrenAhlkronaHelanow2022,WirbelJarosch2020} among many examples.  Extending Stokes-based viscous models by allowing overhangs, and supplementing the momentum conservation model with an additional advected damage variable \cite{PralongFunk2005}, so that ice-cliff calving can occur via a stress-failure criterion, might provide a mechanism to explain how (nearly) well-defined surface elevation and thickness functions arise as solutions to the kind of models considered in this paper.

\subsection{Conjectural well-posedness for the continuum problem} \label{subsec:conjecture} Subsections \ref{subsec:noflow}--\ref{subsec:margin} have deployed various imperfect arguments to explain why the backward Euler VI problem \eqref{eq:be:vi} could be well-posed, or at least why other approaches are less promising.  We now state a mathematically-precise conjectural framework for well-posedness of this problem based upon the idea that the surface motion map $\Phi(s) = -\bu|_s\cdot \bn_s$ assigns significantly different results to inputs which are different in norm.

\begin{conjecture} \label{conj:b}  For $\rr>2$ such that Conjecture \ref{conj:a} holds, let $\cX = W^{1,\rr}(\Omega)$.  Fix $b\in\cX$ and let $\cK=\{r\in\cX\,:\,r|_{\partial\Omega}=b|_{\partial\Omega} \text{ and } r\ge b\}$.  Recall $\Phi:\cK\to\cX'$ is then well-defined by Lemma \ref{lem:philipschitz}.  Then there are constants $\alpha>0$ and $\qq>1$ so that
\begin{equation}
\left(\Phi(r) - \Phi(s)\right)[r-s] \ge \alpha \|r-s\|_{\cX}^\qq \qquad \text{for all } r,s\in\cK. \label{eq:conj:b}
\end{equation}
\end{conjecture}

In Section \ref{sec:abstractestimate}, inequality \eqref{eq:conj:b} will be called $\qq$-\emph{coercivity} of $\Phi$ over $\cK$.  In the abstract VI context of that Section, if $F_{\Delta t}$ is both Lipschitz continuous (Lemma \ref{lem:philipschitz}) and $\qq$-coercive, over a closed and convex subset $\cK$ of a Banach space $\cX$, then \eqref{eq:be:vi} is well-posed.  This is what we prove next, namely that Conjectures \ref{conj:a} and \ref{conj:b} are sufficient for well-posedness of each implicit time step VI problem \eqref{eq:be:vi}.

\begin{theorem} \label{thm:stepwellposed}  Assume Conjectures \ref{conj:a} and \ref{conj:b}, and fix $b \in \cX$ to define $\cK$.  Suppose that $s^{n-1}\in\cK$ and that the SMB function $a(t,x)$ is in $C([0,T]; L^{\rr'}(\Omega))$.  Then there exists a unique surface elevation $s\in\cK$ satisfying backwards Euler VI problem \eqref{eq:be:vi}. \end{theorem}

\begin{proof}  Let $R>0$.  By Lemma \ref{lem:philipschitz} there is $C(R)>0$ so that if $r,s\in B_R\cap\cK$ then $\big|\Phi(r)[q] - \Phi(s)[q]\big| \le C(R) \|r-s\|_{\cX} \|q\|_{\cX}$.  Then by definition \eqref{eq:be:Fdefine}, H\"older's inequality, and Sobolev's inequality we have
\begin{align}
\left|F_{\Delta t}(r)[q] - F_{\Delta t}(s)[q]\right| &\le \int_\Omega |r-s||q| + \Delta t\, \big|\Phi(r)[q] - \Phi(s)[q]\big| \\
    &\le C \|r-s\|_{\cX} \|q\|_\cX \notag
\end{align}
for some $C>0$ which depends on $R$ and $\Delta t$.  Thus $F_{\Delta t}$ is (Lipschitz) continuous on bounded subsets of $\cK$.  If Conjecture \ref{conj:b} holds then
\begin{align}
\left(F_{\Delta t}(r) - F_{\Delta t}(s)\right)[r-s] &= \int_\Omega (r-s)^2 + \Delta t\, \left(\Phi(r) - \Phi(s)\right)[r-s] \\
    &\ge \alpha\Delta t \|r-s\|_{\cX}^\qq, \notag
\end{align}
so $F_{\Delta t}$ is $\qq$-coercive over $\cX$.  From definition \eqref{eq:be:source}, the hypothesis on $a$, and H\"older's inequality,
\begin{equation}
\big|\ell^n[q]\big| \le \left(\|s^{n-1}\|_{L^{\rr'}} + \Delta t \max_{t\in[t_{n-1},t_n]} \|a(t,\cdot)\|_{L^{\rr'}}\right) \|q\|_{L^\rr}
\label{eq:sourcetermbound}
\end{equation}
for all $q \in \cX$.  Because $\|s^{n-1}\|_{L^{\rr'}}<\infty$ by Sobolev's inequality, and $\|q\|_{L^\rr} \le \|q\|_{\cX}$, it follows that $\ell^n \in \cX'$.  Because $F_{\Delta t}$ is $\qq$-coercive it is also coercive and strictly-monotone (Definition \ref{def:monotonecoercive}).  Now Corollary III.1.8 of \cite{KinderlehrerStampacchia1980} shows existence of a solution to \eqref{eq:be:vi}, which is unique by strict monotonicity.
\end{proof}

Theorem \ref{thm:stepwellposed} addresses the well-posedness of a single time-step problem over $[t_{n-1},t_n]$.  Its conclusion is not sufficient to show well-posedness of the time-dependent problem parabolic VI problem, over $[0,T]$, corresponding to NCP \eqref{eq:ncp}.  If this problem were known to be well-posed then one might analyze whether implicit steps converge in the $\Delta t\to 0$ limit.

It would seem that those modeling practitioners who use Stokes dynamics expect Conjectures \ref{conj:a} and \ref{conj:b}, or equivalents, to hold, but they may be difficult to prove despite some progress in Sections \ref{sec:stokes} and \ref{sec:model}.  Greater progress has been made in the SIA case \cite{Calvoetal2003,JouvetBueler2012,PiersantiTemam2023}, which might be helpful.

The computation of $\bu|_s$ and $\Phi(s)=-\bu|_s\cdot \bn_s$, or equivalent expressions, would seem to be necessary in any evolving-geometry Stokes framework, and only these expressions are addressed in the Conjectures.  By contrast, the backward Euler scheme used here is merely the simplest A-stable \cite{AscherPetzold1998} scheme which can be applied.  (Note that for finite dimensional DAE problems, implicit and A-stable schemes are standard \cite{AscherPetzold1998}.)  In any case, the geometry error bounds and results in the next two Sections only require the conclusion of Theorem \ref{thm:stepwellposed}.


\section{Numerical evaluation of coercivity} \label{sec:numerical}

We may explore the validity of Conjecture \ref{conj:b} by sampling from numerical simulations.  The experiment here,\footnote{The Python source code is at \href{https://github.com/bueler/glacier-fe-estimate}{\texttt{github.com/bueler/glacier-fe-estimate}}, and it should be consulted for all details not addressed here.} performed using the Firedrake FE library \cite{Hametal2023}, is not intended to demonstrate implicit time-stepping, but only to generate admissible surface elevation pairs $r,s\in\cK$ to use as samples.  For a given pair we evaluate the coercivity ratio
\begin{equation}
\rho(r,s) = \frac{\left(\Phi(r) - \Phi(s)\right)[r-s]}{\|r-s\|_{\cX}^2}. \label{eq:Phiratio}
\end{equation}
If, for all $r,s$ in some $\cK$, the set of ratios $\{\rho(r,s)\}$ were bounded below by a positive constant $\alpha>0$, then this would confirm the $\qq=2$ case of coercivity inequality \eqref{eq:conj:b} for that $\cK$.  Of course, a numerical experiment allows only finite sampling, and furthermore a finite discretization must be used.

The domain for our experiment is the 1D interval $\Omega=(-L,L)$, $L=100$ km, with $\cX = W_b^{1,2}(\Omega)$.  Three bed profiles  (Figure \ref{fig:cases}) were considered, \emph{flat} with $b=0$, \emph{smooth} with a superposition of several wavelengths down to 10 km, and \emph{rough} with an additional 4 km wavelength mode.  The beds generate corresponding constraint sets $\cK_i \subset \cX$, $i=1,2,3$.

\begin{figure}[ht]
\mbox{\includegraphics[width=0.31\textwidth]{figs/snapsflat.png} \, \includegraphics[width=0.335\textwidth]{figs/snapssmooth.png} \, \includegraphics[width=0.335\textwidth]{figs/snapsrough.png}}

\caption{Three bed cases (left-to-right: flat, smooth, rough) define constraint sets $\cK_i\subset\cX$ in the numerical experiment.  Three time-dependent runs of $T=200$ years, starting from the same initial state (dotted), but using different constant values of the SMB (see text), generated many admissible states in a given $\cK_i$.  Later states ($t=170$ years) in these runs are shown (solid).  Ratios \eqref{eq:Phiratio} were then computed for 1000 sample pairs $r,s$ from each set $\cK_i$.}
\label{fig:cases}
\end{figure}

The interval $\Omega$ was uniformly-meshed into equal intervals, at resolutions which are addressed below.  A $P_1$ piecewise-linear FE space $\cX_h\subset \cX$ was used for the bed $b$ and the surface $s$, giving polygonal domains $\Lambda$ defined by $b,s$ (equation \eqref{eq:icydomain}).  The Stokes problem \eqref{eq:glenstokes:weak}, with viscosity regularization value $\eps=10^{-19}\, \text{s}^{-2}$ in \eqref{eq:glen}, was solved on each domain $\Lambda$ using a vertically-extruded mesh (Figure \ref{fig:fe:operatorvisualization}).  The mixed FE method used the $P_2\times P_1$ (Taylor-Hood) stable pair \cite{Elmanetal2014} and a Newton solver, with direct solution of the linear step equations.

For each bed and constraint set $\cK_i$, three constant SMB values were considered: $a=0,-2.5\times 10^{-7},10^{-7}$ $\text{m}\,\text{s}^{-1}$.  For each value a time-dependent run of duration $T=200$ years started from the same initial surface elevation, a Halfar\footnote{In the case of a flat bed and $a=0$, the exact time-dependent solution under SIA dynamics is known by Halfar's result \cite{Halfar1981}.  The initial state used here has characteristic time $t_0=29$ years.  The final ($t=T$) surface elevation, computed using Stokes dynamics in the experiment, agrees closely with the SIA exact solution; compare comments in \cite{LofgrenAhlkronaHelanow2022}.} profile \cite{Halfar1981}.  Time-stepping of the surface was based on the ``FSSA'' stabilization technique from \cite[equation (23)]{LofgrenAhlkronaHelanow2022}.  Each time-step VI \eqref{eq:be:vi} was semi-implicit, modifying definition \eqref{eq:be:Fdefine} by using the prior surface velocity $\bu|_{s^{n-1}}$, and numerically solved by a reduced-space Newton method with line search \cite{BensonMunson2006}.  The resulting time-dependent numerical method is only conditionally stable, but adequate for the experimental purposes.

FIXME Table \ref{tab:histograms}

FIXME also compute ratios
\begin{equation}
\frac{\big\|\bu|_r - \bu|_s\big\|_{L^{\rr'}}}{\|r-s\|_{W^{1,\rr}}}
\end{equation}
to evaluate Conjecture \ref{conj:a}


\section{Abstract error estimate for a finite element approximation} \label{sec:abstractestimate}

In this Section we consider the finite element (FE) approximation of an abstract variational inequality (VI) problem.  We will return to glaciological problem \eqref{eq:be:vi} in Section \ref{sec:application}.

Let $\cX$ be a real reflexive Banach space with norm $\|\cdot\|$ and topological dual (Banach) space $\cX'$.  Denote the dual pairing of $\ell \in \cX'$ and $v\in\cX$ by $\ell[v]$, and define $\|\ell\|_{\cX'} = \sup_{\|v\|=1} \big|\ell[v]\big|$.  Let $\cK \subset \cX$ be a nonempty, closed, and convex subset, called the constraint set, whose elements are called admissible.  For a continuous, but generally nonlinear, operator $f:\cK \to \cX'$, and a source $\ell\in \cX'$, the VI problem is to find $u\in \cK$ such that
\begin{equation}
f(u)[v-u] \ge \ell[v-u] \quad \text{for all } v\in \cK. \label{eq:vi}
\end{equation}
The best known example of such a problem is the obstacle problem for the Laplacian operator---see \cite{Ciarlet2002,Evans2010,KinderlehrerStampacchia1980} for theory and FE analysis, and VI \eqref{eq:be:vi} is in this form.  A key observation is that $f(u)-\ell \in \cX'$ is generally nonzero when $u$ solves \eqref{eq:vi}, though if $u$ is in the interior of $\cK$ then $f(u)=\ell$.  Under sufficient regularity assumptions an NCP like \eqref{eq:ncp} or \eqref{eq:be:ncp} follows from \eqref{eq:vi}.

The following definitions are standard \cite[Chapter III]{KinderlehrerStampacchia1980}.  The definition of $\qq$-coercive appeared in Conjecture \ref{conj:b}.

\begin{definition} \label{def:monotonecoercive}
An operator $f:\cK \to \cX'$ is said to be \emph{monotone} if
\begin{equation}
\left(f(v)-f(w)\right)[v-w] \ge 0 \qquad \text{for all } v,w \in \cK \label{eq:monotone}
\end{equation}
and \emph{strictly monotone} if equality in \eqref{eq:monotone} implies $v=w$.  It is \emph{coercive} if there is $w\in \cK$ so that $\left(f(v)-f(w)\right)[v-w]/\|v-w\| \to +\infty$ for $v \in \cK$ as $\|v\| \to +\infty$.  It is \emph{$\qq$-coercive} \cite{Bueler2021conservation}, for some $\qq>1$, if there exists $\alpha>0$ such that
\begin{equation}
\left(f(v)-f(w)\right)[v-w] \ge \alpha \|v-w\|^\qq \qquad \text{for all } v,w \in \cK. \label{eq:qcoercive}
\end{equation}
\end{definition}

If $f:\cK \to \cX'$ is monotone and coercive, and also continuous on finite-dimensional subspaces, then VI \eqref{eq:vi} has a solution \cite[Corollary III.1.8]{KinderlehrerStampacchia1980}.  If $f$ is strictly monotone then the solution is unique.  If $f$ is $\qq$-coercive then it coercive and strictly monotone, so $\qq$-coercivity and continuity yield well-posedness for \eqref{eq:vi}.  Note that Definitions \ref{def:monotonecoercive} and \ref{def:lipshitz} do not require $f$ to be defined on all of $\cX$, but only on $\cK$.  
  
The following definition appeared in Lemma \ref{lem:philipschitz}.  If it holds then $f$ is continuous.

\begin{definition} \label{def:lipshitz}
For $R>0$ let $B_R = \{v\in \cX\,:\,\|v\|\le R\}$.  We say $f:\cK \to \cX'$ is \emph{Lipshitz on bounded subsets of $\cK$} if for every $R>0$ there is $C(R)>0$ so that if $v,w \in B_R \cap \cK$ and $z\in\cX$ then $|\left(f(v)-f(w)\right)[z]| \le C(R) \|v-w\| \|z\|$, equivalently
\begin{equation}
\|f(v)-f(w)\|_{\cX'} \le C(R) \|v-w\| \quad \text{ for all } v,w \in B_R \cap \cK.  \label{eq:liponbounded}
\end{equation}
\end{definition}

An FE method for \eqref{eq:vi} is a finite-dimensional VI problem.  Suppose $\cX_h \subset \cX$ is a finite-dimensional subspace, typically some space of continuous, piecewise-polynomial functions defined on a mesh.  The FE constraint set $\cK_h\subset \cX_h$ is assumed to be closed and convex, but generally $\cK_h \nsubseteq \cK$.  Let $f_h:\cK_h\to\cX'$, and note that generally $f_h\ne f$ because of quadrature and other approximations.  (Looking ahead, both $\cK_h \nsubseteq \cK$ and $f_h\ne f$ occur naturally in the glacier geometry problem; see Section \ref{sec:application}.)  The FE VI problem is
\begin{equation}
f_h(u_h)[v_h-u_h] \ge \ell[v_h-u_h] \quad \text{for all } v_h\in \cK_h. \label{eq:fe:vi}
\end{equation}
We will assume that \eqref{eq:fe:vi} has a solution $u_h\in\cK_h$.

The following abstract error estimation theorem extends the well-known result by Falk \cite{Falk1974}; see also Theorem 5.1.1 in \cite{Ciarlet2002}.  We will \emph{not} assume and of the following: $f$ is linear, $\cK_h \subset \cK$, $f_h=f$, $f_h$ is $\qq$-coercive, $f$ is Lipschitz, or $f_h$ is continuous.  However, we must assume that the domain of $f$ includes the FE solution, which is achieved here by defining a convex superset of $\cK$ and $\cK_h$.  This technical assumption permits a clean and general estimation theorem, but the choice of $\cK_h$ made in Section \ref{sec:application} means that $\hcK$ is not needed in our glacier application; see Corollary \ref{cor:abstractestimate:nohull}.

\begin{theorem} \label{thm:abstractestimate}  Define $\hcK = \overline{\Hull{(\cK \cup \cK_h)}}$ as the closure in $\cX$ of the convex hull of $\cK \cup \cK_h$, and suppose that $f:\hcK \to \cX'$.  For $\qq>1$, with conjugate exponent $\qq'=\qq/(\qq-1)$, assume that $f$ is $\qq$-coercive over $\hcK$ with constant $\alpha>0$, and Lipshitz on bounded sets of $\hcK$.  Suppose $u\in\cK$ solves \eqref{eq:vi} and $u_h\in\cK_h$ solves \eqref{eq:fe:vi}, and let $R_h=\max\{\|u\|,\|u_h\|\}$.  Then there is a constant $c=c(R_h)>0$, not otherwise depending on $u$ or $u_h$, so that
\begin{align}
\|u-u_h\|^\qq &\le \quad \frac{2}{\alpha} \inf_{v\in\cK} \left(f(u)-\ell\right)[v-u_h] \label{eq:abstractestimate} \\
   &\quad\, + \frac{2}{\alpha} \inf_{v_h\in\cK_h} \left(f(u)-\ell\right)[v_h-u] \notag \\
   &\quad\, + \frac{2}{\alpha} \left(f(u_h)-f_h(u_h)\right)[u_h] \notag \\
   &\quad\, + \inf_{v_h\in\cK_h} c \|v_h - u\|^{\qq'}. \notag
\end{align}
\end{theorem}

\begin{proof}  For arbitrary $v\in\cK$ and $v_h\in\cK_h$, rewrite \eqref{eq:vi} and \eqref{eq:fe:vi} as follows:
\begin{align}
f(u)[u]     &\le f(u)[v] + \ell[u-v], \label{eq:abstract:one}  \\
f_h(u_h)[u_h] &\le f_h(u_h)[v_h] + \ell[u_h-v_h]. \notag
\end{align}
It follows from \eqref{eq:abstract:one} and the $\qq$-coercivity of $f$ that
\begin{align}
\alpha \|u-u_h\|^\qq &\le \left(f(u)-f(u_h)\right)[u-u_h] \label{eq:abstract:two} \\
  &= f(u)[u] + f(u_h)[u_h] - f(u)[u_h] - f(u_h)[u] \notag \\
  &= f(u)[u] + f_h(u_h)[u_h] \notag \\
  &\qquad - f(u)[u_h] - f(u_h)[u] + \left(f(u_h)-f_h(u_h)\right)[u_h] \notag \\
  &\le f(u)[v] + \ell[u-v] + f(u_h)[v_h] + \ell[u_h-v_h] \notag \\
  &\qquad - f(u)[u_h] - f(u_h)[u] + \left(f(u_h)-f_h(u_h)\right)[u_h] \notag \\
  &= f(u)[v-u_h] - \ell[v-u_h] + f(u_h)[v_h-u] - \ell[v_h-u] \notag \\
  &\qquad + \left(f(u_h)-f_h(u_h)\right)[u_h] \notag \\
  &= \left(f(u)-\ell\right)[v-u_h] + \left(f(u)-\ell\right)[v_h-u] \notag \\
  &\qquad + \left(f(u)-f(u_h)\right)[u-v_h] + \left(f(u_h)-f_h(u_h)\right)[u_h] \notag
\end{align}
Since $u,u_h\in B_{R_h}$, by the Lipshitz assumption over $\hcK$ there is $C(R_h)>0$ so that
\begin{equation}
\left(f(u)-f(u_h)\right)[u-v_h] \le C(R_h) \|u-u_h\|\|u-v_h\|. \label{eq:abstract:three}
\end{equation}
Noting $1<\qq<\infty$, now use Young's inequality with $\eps>0$ \cite[Appendix B.2]{Evans2010}:
\begin{align}
\alpha \|u-u_h\|^\qq &\le \left(f(u)-\ell\right)[v-u_h] + \left(f(u)-\ell\right)[v_h-u]  \label{eq:abstract:four} \\
  &\qquad + C(R_h) \left(\eps\|u-u_h\|^\qq + \tilde C(\eps) \|u-v_h\|^{\qq'}\right) \notag \\
  &\qquad + \left(f(u_h)-f_h(u_h)\right)[u_h], \notag
\end{align}
where $\tilde C(\eps) = (\eps \qq)^{-\qq'/\qq} {\qq'}^{-1}$.  Choose $\eps>0$ so that $C(R_h) \eps \le \alpha/2$, and subtract:
\begin{align}
\frac{\alpha}{2} \|u-u_h\|^\qq &\le \left(f(u)-\ell\right)[v-u_h] + \left(f(u)-\ell\right)[v_h-u]  \label{eq:abstract:five} \\
  &\qquad + C(R_h) \tilde C(\eps) \|u-v_h\|^{\qq'} + \left(f(u_h)-f_h(u_h)\right)[u_h] \notag
\end{align}
Take infimums to show \eqref{eq:abstractestimate}.
\end{proof}

The next Corollary addresses two important cases where the convex hull operation is not needed.  We will see in Section \ref{sec:application} that case \emph{i)} can be imposed in glacier simulations.

\begin{corollary}  \label{cor:abstractestimate:nohull}  In addition to the assumptions of Theorem \ref{thm:abstractestimate}, suppose that one of the following situations apply:
\renewcommand{\labelenumi}{\roman{enumi})}
\begin{enumerate}
\item $\cK_h \subset \cK$, with $f$ $\qq$-coercive over $\cK$, or
\item $f$ is defined on, and $\qq$-coercive over, all of $\cX$.
\end{enumerate}
Then, without the construction of the convex hull $\hcK$, conclusion \eqref{eq:abstractestimate} applies as stated, and in case i) the ``\,$\inf_{v\in\cK}$'' term in \eqref{eq:abstractestimate} is zero.
\end{corollary}

Consider $f(u)-\ell\in \cX'$.  It might be a measure or a measurable function, and the first two terms in estimate \eqref{eq:abstractestimate} preserve information about its support.  (This plays a role in the glacier application of Section \ref{sec:application}.)  By contrast, the Hilbert space result in \cite{Falk1974} computes norms and loses this information.  The following Corollary, with easy proof, takes such a norm-based approach.  We suppose that $\cX$ continuously and densely embeds into a larger Banach space $\cB$:
\begin{equation}
\cX \hookrightarrow \cB, \quad \bar{\cX} = \cB \label{eq:VembedsinB}
\end{equation}
Observe that $\cB' \subset \cX'$.  A standard example is $\cX=W^{1,\rr}(\Omega)$ and $\cB=L^\rr(\Omega)$.

\begin{corollary}  \label{cor:abstractestimate:Bnorm}  In addition to the assumptions of Theorem \ref{thm:abstractestimate}, suppose \eqref{eq:VembedsinB} holds, and that $\|f(u)-\ell\|_{\cB'} < \infty$.  Then
\begin{align}
\|u-u_h\|^\qq &\le \frac{2}{\alpha} \|f(u)-\ell\|_{\cB'} \left( \inf_{v\in\cK} \|v-u_h\|_{\cB} +   \inf_{v_h\in\cK_h} \|v_h-u\|_{\cB} \right) \label{eq:abstractestimate:Bnorm} \\
   &\qquad + \frac{2}{\alpha} \left(f(u_h)-f_h(u_h)\right)[u_h] + \inf_{v_h\in\cK_h} c \|v_h - u\|^{\qq'} \notag
\end{align}
\end{corollary}

The result by Falk \cite{Falk1974} combines the above two Corollaries, under the further assumption that $f$ is linear.  To say this precisely, suppose $f(v)[w]=a(v,w)$ is bilinear, uniformly elliptic, and continuous on a Hilbert space $\cX$.  (The definition of uniformly elliptic coincides with definition \eqref{eq:qcoercive} of $2$-coercive, and continuity of $a(v,w)$ implies \eqref{eq:liponbounded}.)  Suppose that $\cX\hookrightarrow \cH$ and $\bar{\cX} = \cH$ for some Hilbert space $\cH$, and that $\|f(u)-\ell\|_{\cH'} < \infty$ so that, up to isomorphism, $f(u)-\ell \in \cH$.  Finally, suppose that $f(u_h)=f_h(u_h)$.  Then case \emph{ii)} of Corollary \ref{cor:abstractestimate:nohull} combines with Corollary \ref{cor:abstractestimate:Bnorm} to yield Theorem 1 in \cite{Falk1974}.

The $\inf_{v\in\cK}$ term in estimates \eqref{eq:abstractestimate} and \eqref{eq:abstractestimate:Bnorm} is generally nonzero in obstacle problems where $\cK_h \nsubseteq \cK$ \cite{Ciarlet2002}.  In fact, consider a unilateral obstacle problem where $\cK=\{v \in \cX\,:\,v\ge \psi\}$.  Suppose $\psi_h$ is an FE interpolant of $\psi$, and define $\cK_h=\{v_h \in \cX_h\,:\,v_h\ge \psi_h\}$.  While $\psi_h(x_j)=\psi(x_j)$ for interpolation nodes $x_j$, generally $\psi_h(x) \ge \psi(x)$ does not hold for all $x\in\Omega$ even if $\psi$ is arbitrarily smooth.  Thus nodal admissiblity does not generally imply admissibility if interpolation is applied to the obstacle (Figure \ref{fig:nonadmissible}).  While this concern is relevant for glacier models which track the surface elevation, in Section \ref{sec:application} we bypass it using a one-sided interpolation process.  Models which solve for ice thickness avoid it intrinsically because for $\cK = \{v\in\cX\,:\,v\ge 0\}$ we have $\cK_h=\cK\cap\cX_h \subset \cK$, at least for $P_1$ elements.

\begin{figure}[ht]
\begin{center}
FIXME figure like Ciarlet Figure 5.1.3 %\includegraphics[width=0.8\textwidth]{figs/xxx.jpg}
\end{center}
\caption{Nodal admissibility generally does not imply admissibility when $\psi_h$ is the FE interpolant of the obstacle $\psi$.}
\label{fig:nonadmissible}
\end{figure}

The following Corollary collects various conclusions one might draw from assuming $\cK_h \subset \cK$ and then making stronger assumptions, especially that $f_h=f$.

\begin{corollary}  \label{cor:abstractestimate:various}  Make the assumptions of case i) of Corollary \ref{cor:abstractestimate:nohull}.  Also assume that $f_h(u_h)[u_h] = f(u_h)[u_h]$.  Then
\begin{equation}
\|u-u_h\|^\qq \le  \inf_{v_h\in\cK_h} \left\{\frac{2}{\alpha} \left(f(u)-\ell\right)[v_h-u] + c \|v_h - u\|^{\qq'}\right\}. \label{eq:abstractestimate:subset}
\end{equation}
If also the assumptions of Corollary \ref{cor:abstractestimate:Bnorm} hold then
\begin{equation}
\|u-u_h\|^\qq \le \inf_{v_h\in\cK_h} \left\{\frac{2}{\alpha} \|f(u)-\ell\|_{\cB'} \|v_h-u\|_{\cB} + c \|v_h-u\|^{\qq'}\right\} \label{eq:abstractestimate:subset:Bnorm}
\end{equation}
If $f(u)=\ell$, for example if $u$ is in the interior of $\cK$, then
\begin{equation}
\|u-u_h\|^\qq \le c \inf_{v_h\in\cK_h} \|v_h-u\|^{\qq'} \label{eq:abstractestimate:subset:Cea}
\end{equation}
\end{corollary}

Note that \eqref{eq:abstractestimate:subset:Cea} is Cea's lemma \cite[Theorem 2.4.1]{Ciarlet2002} in a Banach space, though with a constant which depends on $R_h=\max\{\|u\|,\|u_h\|\}$.  This case is a reduction to the PDE case with no active set or free boundary.  In the glacier context it applies when the entire domain $\Omega$ is covered in ice.


\section{Application of the theory to numerical glacier models} \label{sec:application}

Now we can synthesize the theory and apply it to an implicit time step of a glacier simulation which uses Stokes dynamics.  This will give ``conforming FE method'' a precise meaning for evolving-geometry glacier simulations.  We will combine three types of results: \emph{i)} well-posedness theory and \emph{a priori} bounds for the glaciological Stokes problem on a fixed domain (Section \ref{sec:stokes}), \emph{ii)} conjectural well-posedness theory of the surface elevation VI problem (Sections \ref{sec:model} and \ref{sec:theory}), and \emph{iii)} an abstract error estimate for FE solutions of VIs (Section \ref{sec:abstractestimate}).

Consider an FE method for VI problem \eqref{eq:be:vi}.  For a finite-dimensional subspace $\cX_h\subset \cX$, with a constraint set $\cK_h\subset \cX_h$, we seek $s_h\in\cK_h$ solving
\begin{equation}
F^h_{\Delta t}(s_h)[r_h-s_h] \ge \ell^n[r_h-s_h] \quad \text{for all } r_h \in \cK_h. \label{eq:fe:be:vi}
\end{equation}
The operator $F^h_{\Delta t}$ denotes an FE approximation to the operator $F_{\Delta t}$ defined in \eqref{eq:be:Fdefine}, while $\ell^n = s^{n-1} + \Delta t\,a^n$ is defined exactly as before, by equation \eqref{eq:be:source}.  We assume that $s^{n-1} \in \cK$ is general, and not necessarily that it is in the FE space $\cK_h$.

Evaluation of $F^h_{\Delta t}(s_h)$ in the FE VI problem \eqref{eq:fe:be:vi} requires the nontrivial numerical solution of a glaciological Stokes problem \eqref{eq:stokes} over a 3D mesh of the domain between $z=b_h$ and $z=s_h$ (Figure \ref{fig:fe:operatorvisualization}).  Such a mesh need not be extruded vertically as shown, nor must $s_h$ necessarily be piecewise-linear, but the upper and lower surfaces, where boundary conditions \eqref{eq:stokes:stressfreesurface} and \eqref{eq:stokes:noslide} are applied, must be FE-space functions, i.e.~$s_h,b_h\in\cX_h$ with $s_h\ge b_h$.  The numerical velocity from solving the Stokes problem, over the domain geometry defined by $s_h$, is denoted $\bu_h$, and its surface trace is denoted $\bu_h|_{s_h}$.  Observe that $\bu_h|_{s_h}$ will generally be different from the surface trace of the exact solution of the same Stokes boundary value problem for the same ($s_h$) geometry, denoted by $\bu|_{s_h}$.  Although $F^h_{\Delta t}(s_h)$ is defined by formula \eqref{eq:be:Fdefine}, it is a different operator from $F_{\Delta t}(s_h)$ because it uses the numerical solution velocity and not the exact solution velocity (of the \emph{same} Stokes problem).

\begin{figure}[ht]
\begin{center}
\includegraphics[width=0.75\textwidth]{genfigs/extruded.pdf}
\end{center}
\caption{Evaluating $F^h_{\Delta t}(s_h)$ in \eqref{eq:fe:be:vi} requires numerically solving a Stokes problem for velocity $\bu_h$, on a mesh between $b_h$ and $s_h$, and then evaluating its upper surface trace.}
\label{fig:fe:operatorvisualization}
\end{figure}

A key concern in applying abstract Theorem \ref{thm:abstractestimate} or its Corollaries to a glaciological context is the choice of the numerical bed elevation $b_h \approx b$, which defines the constraint set $\cK_h$.  We will assume $b$ is continuous on the closed domain $\bar\Omega$.  An abstract $b\in C(\bar\Omega) \cap \cX$ can then be considered, but in practice $b$ is provided via a high resolution map derived from ice-penetrating radar \cite{Morlighemetal2017}.  Thus $b$ may in fact be in an FE space, but often on a (much) finer mesh.  We assert that, because of what we prove in this Section, it is better to choose $b_h \in \cX_h$ to satisfy $b_h\ge b$.  Maximum monotone restriction \cite{BuelerFarrell2024} or similar can be applied to achieve this, i.e.~$b_h = R^\oplus b$ if $\cX_h$ is generated from a submesh of the fine mesh on which $b$ is provided.  Also, as one would in ``conforming'' FE methods for classical PDE problems \cite{Elmanetal2014}, we assume $b_h=b$ along $\partial\Omega$.

We define an interpolation and truncation operation $\Pi_h : \cX \to \cK_h$ as follows.  For $r\in\cX$ this gives the unique FE function $\Pi_h(r) \in \cX_h$ so that
\begin{equation}
\Pi_h(r)(x_j) = \max \,\{b_h(x_j), r(x_j)\} \label{eq:definePi}
\end{equation}
for every interior node $x_j \in \cT_h$, with $\Pi_h(x_j)=b(x_j)$ if $x_j\in\partial\Omega$.  Observe that definition \eqref{eq:definePi} only yields nodal admissibility.  The FE space must be such that this implies admissibility \emph{per se}, namely that $\Pi_h(r)(x) \ge b_h(x)$ for all $x \in \Omega$, so that $\Pi_h(r) \in \cK_h$.  This condition is satisfied by the continuous and piecewise-linear FE space $P_1$, but not, for example, by $P_2$ \cite{BuelerFarrell2024}.

To summarize, from now on we make certain standard assumptions for solving VI problem \eqref{eq:be:vi} using numerical scheme \eqref{eq:fe:be:vi}.

\smallskip
\begin{assumptions}
The following data are given:
\begin{enumerate}
\item A bounded, convex polygon $\Omega\subset\RR^2$.
\item An exponent $\rr > 2$, with conjugate exponent $\rr' = \rr/(\rr-1)$. \label{item:rr}
\item A time-dependent SMB function $a\in C([0,T]; L^{\rr'}(\Omega))$.
\item A bed topography function $b \in C(\bar\Omega) \cap W^{1,\rr}(\Omega)$, with piecewise-linear boundary values $b|_{\partial\Omega}$.
\end{enumerate}
We make these definitions:
\begin{enumerate}
\setcounter{enumi}{4}
\item $\cX = W^{1,\rr}(\Omega)$.
\item $\cK = \{r\in\cX\,:\,r|_{\partial \Omega} = b|_{\partial \Omega} \text{ and } r \ge b\}$.
\item $\cX_h \subset \cX$ denotes a finite-dimensional and conforming FE space, from a mesh $\cT_h$ exactly tiling $\bar\Omega$.
\item The boundary values $b|_{\partial\Omega}$ are exactly representable in the FE space.
\end{enumerate}
The following are assumed to hold:
\begin{enumerate}
\setcounter{enumi}{8}
\item Conjecture \ref{conj:a} holds with Lipschitz constant $\CA > 0$. \label{item:conj:a}
\item Conjecture \ref{conj:b} holds with exponent $\qq>1$ and coercivity constant $\alpha > 0$.\label{item:conj:b}
\end{enumerate}
We also assume and define:
\begin{enumerate}
\setcounter{enumi}{10}
\item $b_h\in\cX_h$ is given, with $b_h\ge b$ on $\bar\Omega$ and $b_h=b$ along $\partial \Omega$. \label{item:goodbh}
\item $\cK_h = \{r_h\in\cX_h\,:\,r_h|_{\partial \Omega} = b_h|_{\partial \Omega} \text{ and } r_h \ge b_h\}$. \label{item:defineKh}
\item Interpolation/truncation $\Pi_h$ yields admissible elements in $\cK_h$.  \label{item:Pi}
\end{enumerate}
\end{assumptions}

\medskip
The conforming condition $\cK_h\subset \cK$ follows from assumptions \ref{item:goodbh} and \ref{item:defineKh}, with advantages to be exposed.  As seen in the proof of Theorem \ref{thm:stepwellposed}, assumptions \ref{item:conj:a} and \ref{item:conj:b} show that $F_{\Delta t}$ is $\qq$-coercive and Lipschitz on bounded subsets of $\cK$.  By Theorem \ref{thm:stepwellposed} and case \emph{i)} of Corollary \ref{cor:abstractestimate:nohull} we have the following Lemma.

\begin{lemma} \label{lem:preglacierapp}  Make the Standard Assumptions.  Suppose that $s^{n-1}\in\cK$ and define $\ell^n \in \cX'$ by \eqref{eq:be:source}.  Let $s\in\cK$ be the unique surface elevation satisfying the implicit time-step VI problem \eqref{eq:be:vi}.  Assume that $F^h_{\Delta t}$ represents a numerical scheme for, and that $s_h\in\cK_h$ is a solution of, problem \eqref{eq:fe:be:vi}.  Let $R_h=\max\{\|s\|_\cX,\|s_h\|_\cX\}$.  Then there is a constant $c_0>0$, depending on $R_h$ (and not otherwise depending on $s$ or $s_h$), and independent of $\Delta t>0$, so that % FIXME is c_0 indep of Delta t?
\begin{align}
\|s-s_h\|_\cX^\rr &\le \quad \frac{2}{\alpha} \inf_{r_h\in\cK_h} \left(F_{\Delta t}(s)-\ell^n\right)[r_h-s] \label{eq:preglacierestimate} \\
   &\quad\, + \frac{2}{\alpha} \left(F_{\Delta t}(s_h)-F^h_{\Delta t}(s_h)\right)[s_h] \notag \\
   &\quad\, + c_0 \inf_{r_h\in\cK_h} \|r_h - s\|_{\cX}^\qq. \notag
\end{align}
\end{lemma}

Each term in estimate \eqref{eq:preglacierestimate} turns out to have a clear glaciological meaning, which we expose in the following Theorem.  For the statement recall that $h$ denotes the maximum diameter of cells in $\cT_h$, $\Lambda_{s_h}$ denotes the 3D domain defined by $s_h$, $\gamma_\pp(\Lambda_{s_h})$ denotes the trace constant of that domain (Lemma \ref{lem:trace}), and $\cV=W_b^{1,\pp}(\Lambda_{s_h}; \RR^3)$ is the velocity space for the Stokes problem \eqref{eq:glenstokes:weak}.

\begin{theorem} \label{thm:glacierapp} Make the same assumptions as in Lemma \ref{lem:preglacierapp}.  Define $\Omega_A(s) = \left\{x\in\Omega\,:\,s(x)=b(x)\right\}$, the active set for $s$.  Then
\begin{align}
\phantom{dfkljsd} \|s_h-s\|_\cX^\rr &\le \quad \frac{2}{\alpha} \int_{\Omega_A(s)} (b - \ell^n) (b_h - b) &&\text{\textnormal{[term 1]}} \label{eq:glacierestimate} \\
   &\quad\, + \frac{\Delta t}{\alpha} \, \Gamma(s_h) \big\|\bu_h - \bu\big\|_{\cV} &&\text{\textnormal{[term 2]}} \notag \\
   &\quad\, + c_0 \|\Pi_h(s) - s\|_\cX^\qq. &&\text{\textnormal{[term 3]}} \notag
\end{align}
The constant $c_0>0$ is from Lemma \ref{lem:preglacierapp}.  The coefficient in term 2, namely
\begin{equation}
\Gamma(s_h) = c_1 \left(\frac{\gamma_\pp(\Lambda_{s_h})}{[L]}\right)^{1/\pp} \left(|\Omega| + [L]^{-\rr}\|s_h\|_{\cX}^\rr\right)^{1/(\pp'\rr)} \|s_h\|_{L^{\pp'\rr'}},
\end{equation}
depends on nontrivially on $s_h$, but $c_1>0$ depends only on the exponents $\rr$, $\pp$.
\end{theorem}

Before proving the Theorem, consider the meaning of each term.

\medskip
\begin{itemize}
\item[term 1:]  This error term comes from approximation of the bed in the ice-free area defined by the exact solution $s$.  If the bed were exactly representable ($b_h=b$) then this term would be zero.  Note that $s_h \ge b_h \ge b = s$ in the ice-free area $\Omega_A(s)$, so the factor $b_h-b$ in the integrand reflects the smallest possible difference $s_h - s$.  Also $b-\ell^n\ge 0$ (Section \ref{sec:model}) so the integrand is nonnegative.

\item[term 2:]  This shows how numerical errors in solving the Stokes problem over the domain $\Lambda_{s_h}$ will affect the geometrical error in $s_h$.  For an FE method applied to the Stokes problem \eqref{eq:stokes} with some convergence rate $\|\bu_h - \bu\|_{\cV} = O(h^k)$, this term would reflect that rate.

\item[term 3:]  An interpolation error term like this arises in the classical Cea's lemma argument for quasi-optimality, thus convergence, of FE methods \cite{Ciarlet2002}.  However, here there are the added complications that the interpolant of $s$ must be truncated into $\cK_h$, using \eqref{eq:definePi}, and that nodal admissibility must imply admissibility.
\end{itemize}

\begin{proof}  Apply Lemma \ref{lem:preglacierapp}.  Because $s$ solves \eqref{eq:be:vi}, the residual $\Psi = F_{\Delta t}(s)-\ell^n \in \cX'$, while generally nonzero, is non-negative.  In particular, if $\phi\in C_c^\infty(\Omega)$ is nonnegative then $r=s+\phi \in \cK$ and $\Psi[r-s] = \Psi[\phi] \ge 0$.  Thus $\Psi\in\cX'$ is a non-negative distribution, and it is represented by a positive Borel measure $\mu$, $\Psi[\phi] = \int_\Omega \phi\,d\mu$ \cite[Theorem 6.22]{LiebLoss1997}.  However, by the proof of Theorem II.6.9 in \cite{KinderlehrerStampacchia1980} this measure is supported in $\Omega_A(s)$.  In fact, recall from Section \ref{sec:model} that $b-\ell^n\ge 0$ on $\Omega_A(s)$---this gives the density of $\mu$---and note also that $\bu|_{s}=\bzero$ and $s=b$ on $\Omega_A(s)$.  Let $r_h = b_h \in \cK_h$.  From the first term in \eqref{eq:preglacierestimate}, and definition \eqref{eq:be:Fdefine}, we get term 1:
\begin{equation}
\left(F_{\Delta t}(s)-\ell^n\right)[r_h-s] = \int_{\Omega_A(s)} \left(b - \ell^n\right) (b_h - b).
\end{equation}

Consider the second term in \eqref{eq:preglacierestimate}.  Recall that $dS = |\bn_{s_h}|\,dx$ is the surface area element for the surface $\Gamma_{s_h} \subset \partial \Lambda_{s_h}$.  After expanding definition \eqref{eq:be:Fdefine}, apply the triangle and H\"older inequalities:\footnote{A similar argument was used in the proof of Lemma \ref{lem:phibound:early}.}
\begin{align}
\left(F_{\Delta t}(s_h)-F^h_{\Delta t}(s_h)\right)&[s_h] = - \Delta t \int_\Omega \left(\bu|_{s_h} - \bu_h|_{s_h}\right)\cdot \bn_{s_h} s_h  \\
  &\le \Delta t \int_\Omega \Big|\bu|_{s_h} - \bu_h|_{s_h}\Big| |\bn_{s_h}|^{1/\pp} |\bn_{s_h}|^{1/\pp'} |s_h| \notag \\
  &\le \Delta t \left(\int_\Omega \Big|\bu|_{s_h} - \bu_h|_{s_h}\Big|^\pp |\bn_{s_h}|\right)^{1/\pp} \left(\int_\Omega |\bn_{s_h}| |s_h|^{\pp'}\right)^{1/\pp'} \notag \\
  &\le \Delta t \left(\int_{\Gamma_{s_h}} \big|\bu - \bu_h\big|^\pp dS\right)^{1/\pp} \left(\int_\Omega |\bn_{s_h}|^\rr\right)^{1/(\pp'\rr)} \|s_h\|_{L^{\pp'\rr'}} \notag
\end{align}
Now apply the trace inequality, Lemma \ref{lem:trace}, and use the fact that $(1+\alpha)^{\rr/2} \le 2^{(\rr-2)/2} (1+\alpha^{\rr/2})$ on $|\bn_{s_h}|^\rr$:
\begin{align}
\left(F_{\Delta t}(s_h)-F^h_{\Delta t}(s_h)\right)[s_h] &\le \Delta t \left(\frac{\gamma_\pp(\Lambda_{s_h})}{[L]}\right)^{1/\pp} \|\bu - \bu_h\|_{\cV} \\
  &\qquad \cdot \left(2^{(\rr-2)/2} \int_\Omega 1 + |\grad s_h|^\rr\right)^{1/(\pp'\rr)} \|s_h\|_{L^{\pp'\rr'}}.  \notag
\end{align}
Recalling equation \eqref{eq:norm:Omega}, we have term 2.

Term 3 follows by substituting $r_h=\Pi_h(s)$ into the third term in \eqref{eq:preglacierestimate}.
\end{proof}

FIXME FROM BELIEVABLE ASSUMPTIONS ON $\bu$, namely error estimates for FE Stokes velocity solutions \cite{Elmanetal2014}, WE CAN get a more specific form for term 2

FIXME FROM A SUSPICIOUSLY STRONG REGULARITY ASSUMPTIONS ON $s$, namely $s\in W^{2,\bar \rr}(\Omega)$ for $\bar \rr \in [\rr,+\infty]$, WE CAN get a more specific form for terms 3; USE RESULT IN \cite{JouvetBueler2012}, namely \cite[Theorem 3.1.6]{Ciarlet2002}


\section{Discussion and conclusion} \label{sec:conclusion}

FIXME one might simply declare that the ``fake ice'' (where $s=b$ and $s_h=b_h$ and $b_h>b$) does not exist, which reduces term 1, but generally $s$ is unknown so this is not precisely possible

FIXME the barrier theory from \cite{Bueler2021conservation} is written in terms of the thickness but it applies to all models considered here; term 1 in Theorem \ref{thm:glacierapp} is novel relative to \cite{Bueler2021conservation}

FIXME a weaker version of Conjecture \ref{conj:b} comes from setting $r=s + \phi$ for $\phi$ supported where $s>b$; this may be easier to prove;  even if this weaker version is true it is not sufficient for Theorem \ref{thm:stepwellposed}; it does not address the marginal overhang issue


\bibliographystyle{siamplain}
\bibliography{estimate}

\end{document}
