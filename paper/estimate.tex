\documentclass[final,hidelinks,onefignum,onetabnum]{siamart250211}
%\documentclass[review,hidelinks,onefignum,onetabnum]{siamart250211}

\usepackage{amsfonts,amssymb}
\usepackage{graphicx}
\ifpdf
  \DeclareGraphicsExtensions{.eps,.pdf,.png,.jpg}
\else
  \DeclareGraphicsExtensions{.eps}
\fi
\usepackage{bm}
\usepackage{tikz}
\usetikzlibrary{math,positioning}

% Add a serial/Oxford comma by default.
\newcommand{\creflastconjunction}{, and~}

% Used for creating new theorem and remark environments
\newsiamremark{remark}{Remark}
\newsiamthm{stdass}{Standard Assumptions}
\newsiamthm{conjecture}{Conjecture}

\DeclareMathOperator{\diag}{diag}

\newcommand{\eps}{\epsilon}
\newcommand{\RR}{\mathbb{R}}

\newcommand{\grad}{\nabla}
\newcommand{\Div}{\nabla\cdot}

\newcommand{\bbf}{\mathbf{f}}
\newcommand{\bg}{\mathbf{g}}
\newcommand{\bn}{\mathbf{n}}
\newcommand{\bq}{\mathbf{q}}
\newcommand{\bu}{\mathbf{u}}
\newcommand{\bv}{\mathbf{v}}
\newcommand{\bw}{\mathbf{w}}
\newcommand{\bx}{\mathbf{x}}
\newcommand{\bz}{\mathbf{z}}

\newcommand{\bU}{\mathbf{U}}
\newcommand{\bX}{\mathbf{X}}

\newcommand{\bzero}{\bm{0}}

\newcommand{\btau}{\bm{\tau}}

\newcommand{\cB}{\mathcal{B}}
\newcommand{\cH}{\mathcal{H}}
\newcommand{\cK}{\mathcal{K}}
\newcommand{\cQ}{\mathcal{Q}}
\newcommand{\cT}{\mathcal{T}}
\newcommand{\cV}{\mathcal{V}}
\newcommand{\cX}{\mathcal{X}}

\newcommand{\hcK}{\widehat{\cK}}

\newcommand{\nn}{{\text{\textnormal{n}}}}
\newcommand{\pp}{{\text{\textnormal{p}}}}
\newcommand{\qq}{{\text{\textnormal{q}}}}
\newcommand{\rr}{{\text{\textnormal{r}}}}

\newcommand{\ip}[2]{\left<#1,#2\right>}

\newcommand{\XX}{\ding{55}}

\newcommand{\dx}{\, \mathrm{d}x}

\newcommand{\rhoi}{\rho_{\text{i}}}

\newcommand{\nsubset}{\not\subset}
\DeclareMathOperator*{\argmin}{arg\,min}
\DeclareMathOperator*{\Hull}{Hull}

\newcommand{\Vdiv}{\cV_{\text{\textnormal{div}}}}

\newcommand{\CA}{C_\text{\textnormal{A}}}


% Optional PDF information
\ifpdf
\hypersetup{
  pdftitle={Surface elevation errors in finite element Stokes models for glacier evolution},
  pdfauthor={E. Bueler}
}
\fi

% Sets running headers as well as PDF title and authors
\headers{Surface elevation errors in glacier models}{E. Bueler}

% Title.
\title{Surface elevation errors in finite element Stokes models for glacier evolution\thanks{Submitted to the editors DATE (draft \today).}}

% Authors: full names plus addresses.
\author{Ed Bueler\thanks{Dept.~Mathematics \& Statistics, University of Alaska Fairbanks, USA (\email{elbueler@alaska.edu}, \href{https://bueler.github.io/}{\texttt{bueler.github.io}}).}}

% FundRef data to be entered by SIAM
%<funding-group specific-use="FundRef">
%<award-group>
%<funding-source>
%<named-content content-type="funder-name">
%</named-content>
%<named-content content-type="funder-identifier">
%</named-content>
%</funding-source>
%<award-id> </award-id>
%</award-group>
%</funding-group>


\begin{document}

\maketitle

\begin{abstract}
The primary data which determine the evolution of glaciation are the bedrock elevation and the surface mass balance.  From this data, which we assume is defined over a fixed land region, the glacier's geometry solves a free boundary problem which balances the time derivative of the surface elevation, the surface velocity from the Stokes flow of the ice, and the surface balance rate.  This problem can be posed in weak form as a variational inequality over a cone of admissible surface elevation functions, those which are above the bedrock topography.  After some preparatory theory for the Stokes problem, we conjecture that the corresponding continuous-space, implicit time-step variational inequality problem is well-posed if the surface kinematical equation is appropriately regularized.  This conjecture is supported by physical arguments and numerical evidence.  We then prove a general theorem which bounds the numerical error made by finite element approximations of nonlinear-operator variational inequalities in Banach spaces.  This bound is a sum of error terms of different types, special to variational inequalities.  When it is applied to the implicit time-step glacier problem there are three terms in the bound: an error from discretizing the bed elevation, an error from numerically solving for the Stokes velocity, and finally an expected error which is quasi-optimal in the finite element space representation of the surface elevation.  The design of glacier models is then reconsidered based on this \emph{a priori} error analysis.
\end{abstract}

\begin{keywords}
error bounds, finite element methods, glaciers, ice flow, variational inequalities
\end{keywords}

\begin{MSCcodes}
76D27, 76D07, 49J40, 65N30, 65N15
\end{MSCcodes}


\section{Introduction} \label{sec:intro}

Glacier and ice sheet simulations model the flowing ice as a free-surface layer of very-viscous, incompressible, and non-Newtonian fluid \cite{GreveBlatter2009,SchoofHewitt2013}.  The two essential input data into such simulations are the bedrock elevation, which is assumed for simplicity to be independent of time, and the time- and space-dependent surface mass balance rate (SMB), the climate.  By definition, the SMB is the balance between accumulating snow and the loss of melt water, through runoff, at the upper surface of the glacier \cite{Cogleyetal2011}.  Here we will only consider simulations of land-based glaciers and ice sheets (continent-scale glaciers), without floating portions.  Note that elevations are measured in meters, and SMB is in (ice-equivalent) meters per second.

All time-dependent glacier simulation produce the glacier's evolving geometry and flow velocity.  Comprehensive models  \cite{SchoofHewitt2013,Winkelmannetal2011} generally have additional physics and outputs.  For example, they might track the enthalpy/temperature \cite{Aschwandenetal2012} of the ice, or liquid water along glacier surfaces.  However, the current work only considers conservation of mass and momentum, but not energy, and liquid water plays no role.  Furthermore we will assume zero velocity at the base of the ice, a non-sliding and non-penetrating condition.  On the other hand, we will not make any of the shallowness assumptions which are common in present-day comprehensive models, and furthermore we will address the mathematical model rigorously.

One may parameterize the glacier's geometry using either the (upper) surface elevation or the ice thickness.  At any time and map-plane location where a glacier exists the surface elevation exceeds the elevation of the bedrock, equivalently the ice thickness is positive.  The computed flow velocity is only defined at those locations and times where ice is present, on an evolving 3D domain between the bedrock and surface elevations.  In other words, admissibility of the surface elevation is required to make the velocity and pressure Stokes problem meaningful.

Our kinematical notation is sketched in Figure \ref{fig:stokesdomain}.  Let $\Omega \subset \RR^2$ be a fixed domain of land with map-plane coordinates $x=(x_1,x_2)\in\Omega$.  We are given, as data, a continuous bedrock elevation function $b(x)$ on $\Omega$, and a continuous SMB function $a(t,x)$ on $[0,T]\times \Omega$, for some $T>0$.  Where $a>0$ (accumulation; downward arrows in Figure \ref{fig:stokesdomain}), a glacier will exist.  However, where $a<0$ (ablation; upward arrows) then either ice exists with an ablating surface, supported by flow from accumulation areas, or no glacier exists.  Determining which situation applies at a given location and time, and the surface elevation there, requires solving a free-boundary problem.

\begin{figure}[ht]
\centering
\includegraphics[width=0.6\textwidth]{genfigs/stokesdomain.pdf}
\caption{Glacier notation: $x\in\Omega\subset\RR^2$ and $(x,z) \in \Lambda(t)\subset\RR^3$.}
\label{fig:stokesdomain}
\end{figure}

Let $s(t,x)$ be the (solution) ice surface elevation.  This is defined for all $x\in\Omega$, but subject to the constraint that the surface $z=s$ must be at or above the bedrock ($s \ge b$).  In regions with no ice, $s=b$ holds.  The solution ice velocity $\bu(t,x,z)$ and pressure $p(t,x,z)$ are defined only on the open 3D domain
\begin{equation}
\Lambda(t) = \left\{(x,z)\,:\,b(x) < z < s(t,x)\right\} \subset \Omega \times \RR. \label{eq:icydomain}
\end{equation}
This aspect of glacier modeling deserves emphasis:  The time-dependent 3D domain $\Lambda(t)$, on which the velocity and pressure are meaningful, is determined by the evolving surface elevation $s$, which is itself part of the coupled model solution.

The surface trace of the ice velocity will be of importance, and a precise Sobolev space for the velocity will be given in Section \ref{sec:stokes}.  We extend it by zero so that it is defined for all $t,x \in [0,T]\times\Omega$; compare flux extension by zero in \cite{SchoofHewitt2013}:
\begin{equation}
\bu|_s(t,x) = \begin{cases} \bu(t,x,s(t,x)), & s(t,x)>b(t,x) \\
                            \bzero, & \text{otherwise} .\end{cases} \label{eq:defineus}
\end{equation}
Also let $\bn_s = \left<-\partial s/\partial x_1,-\partial s/\partial x_2,1\right>$ be an upward surface normal vector.

The glacier geometry evolution model says that an infinite-dimensional nonlinear complementarity problem (NCP) \cite{FacchineiPang2003}, an obstacle problem, applies in $[0,T]\times \Omega$:
\begin{subequations}
\label{eq:ncp}
\begin{align}
s - b &\ge 0 \label{eq:ncp:constraint} \\
\frac{\partial s}{\partial t} - \bu|_s \cdot \bn_s - a &\ge 0 \label{eq:ncp:residualpos} \\
(s - b) \left(\frac{\partial s}{\partial t} - \bu|_s \cdot \bn_s - a\right) &= 0
\end{align}
\end{subequations}
This system implies that either a location is ice free ($s=b$), where the climate is locally ablating ($a\le 0$), or that the surface kinematical equation (SKE) holds:
\begin{equation}
\frac{\partial s}{\partial t} - \bu|_s \cdot \bn_s - a = 0.  \label{eq:ske}
\end{equation}
This equation says that the (non-material) ice surface moves vertically according to the sum of an ice velocity component at the surface and the SMB.  This statement of mass conservation \cite{Aschwandenetal2012}, also called the free-surface equation \cite{LofgrenAhlkronaHelanow2022}, is a standard description of glacier geometry evolution in numerical models \cite{GreveBlatter2009,SchoofHewitt2013}.  Though it is not yet common to state the system completely, as here, glaciologists would agree with the conditions of NCP \eqref{eq:ncp}.  While condition \eqref{eq:ncp:constraint} is sometimes stated in glacier literature \cite{Durandetal2009,Halfar1981,JouvetBueler2012,PiersantiTemam2023,WirbelJarosch2020}, the complementary fact \eqref{eq:ncp:residualpos}, that the residual is everywhere nonnegative, though observed in \cite{Calvoetal2003}, is rarely written \cite{SchoofHewitt2013}.

Because a simulated glacier needs to be able to advance into unglaciated locations, we assume that the implicit SMB is defined everywhere in $\Omega$, regardless of whether a glacier is present or not.  In ice-free areas the SMB should have the value which a glacier surface at elevation $b$ would experience.  It can be computed, from precipitation and an energy balance model \cite{GreveBlatter2009}, by balancing snow accumulation minus the ablation which would follow from using the available energy for melt (if ice were present).

The non-shallow ice dynamics model used in this paper is the non-sliding (e.g.~frozen base), isothermal, non-Newtonian, and incompressible Stokes system \cite{GreveBlatter2009,JouvetRappaz2011,SchoofHewitt2013}, applied over the domain $\Lambda(t)$ defined in \eqref{eq:icydomain}.  Let $\Gamma_s(t)$, $\Gamma_b(t)$ denote the upper and lower surfaces at $z=s$ and $z=b$, respectively.  The possibility of cliffs at the ice margin will be neglected, so $\partial \Lambda(t) = \overline{\Gamma_s(t)} \cup \overline{\Gamma_b(t)}$.  Let $D\bu=(\grad \bu + \grad \bu^{\top})/2$ denote the strain rate tensor, with Frobenius norm $|D\bu| = \left((D\bu)_{ij} (D\bu)_{ij}\right)^{1/2}$.  A shear-thinning constitutive relation, namely a regularized Glen's flow law \cite{GreveBlatter2009}, computes the effective dynamic ice viscosity by this formula:
\begin{equation}
\nu(D\bu) = \nu_\pp \left(|D\bu|^2 + \mu_0\right)^{(\pp-2)/2}. \label{eq:glen}
\end{equation}
The exponent $1 < \pp \le 2$, written $\pp=(1/\nn)+1$ in terms of Glen's exponent $\nn\ge 1$, is commonly taken to be $\pp=4/3$ \cite{GreveBlatter2009}.  The coefficient $\nu_\pp>0$ necessarily has $\pp$-dependent units, but $\nu(D\bu)$ has SI units $\text{kg}\,\text{m}^{-1}\,\text{s}^{-1}$.  The values of $\nn$ and $\nu_\pp$, assumed to be constant here, are constrained by laboratory experiments \cite{GoldsbyKohlstedt2001,GreveBlatter2009}.  While $\pp=2$ yields a Newtonian fluid with constant viscosity, when $\pp < 2$ the $\mu_0>0$ regularization implies that $\nu(D\bu)$ is bounded.

Now the velocity and pressure solve the following 3D Stokes equations:%
\begin{subequations}
\label{eq:stokes}
\begin{align}
- \nabla \cdot \left(2 \nu(D\bu)\, D\bu\right) + \nabla p &= \rhoi \bg && \text{within $\Lambda(t)$} \\
\nabla \cdot \bu &= 0 && \qquad \text{''} \label{eq:stokes:incomp} \\
\left(2 \nu(D\bu) D\bu - pI\right) \bn_s &= \bzero && \text{on $\Gamma_s(t)$}\label{eq:stokes:stressfreesurface} \\
\bu  &= \bzero && \text{on $\Gamma_b(t)$} \label{eq:stokes:noslide}
\end{align}
\end{subequations}
The density of ice $\rhoi$ and the acceleration of gravity $\bg$ are here assumed constant.  Boundary condition \eqref{eq:stokes:stressfreesurface} says that the sub-aerial upper surface is stress free.  The Stokes problem \eqref{eq:glen}, \eqref{eq:stokes} is commonly discretized using finite elements (FEs) \cite{IsaacStadlerGhattas2015,Jouvetetal2008,Pattynetal2008}; we will return to its FE approximation in Chapter \ref{sec:application}.

In summary at this point, this work considers an evolving free-surface flow for a glacier, subject to a signed climate which can add or remove ice, simultaneously with a non-Newtonian Stokes problem which must be solved within the evolving 3D domain of ice.  This coupled initial-boundary value problem, in strong form as \eqref{eq:icydomain}--\eqref{eq:stokes}, requires data $b(x)$ and $a(t,x)$, plus an initial surface elevation $s(0,x)$.  The solution variables are $s(t,x)$, defined everywhere over $[0,T]\times \Omega$ and subject to $s \ge b$, and $\bu(t,x,z)$, and $p(t,x,z)$ defined only on $\Lambda(t)$ given by \eqref{eq:icydomain}.

A formulation using ice thickness to parameterize glacier geometry would have a different character from ours using surface elevation.  The latter is subject to a flow-caused smoothing effect, illustrated for a real ice sheet in Figure \ref{fig:giscross}.  For land-based glaciers $s(t,x)$ is spatially smoother than the thickness $H(t,x) = s(t,x)-b(x)$, which ``inherits'' the lower regularity of the often steep and eroded bedrock topography.

\begin{figure}
\begin{minipage}[t]{0.8\textwidth}
\vspace{0pt}
\includegraphics[width=\textwidth]{genfigs/giscross.pdf}
\end{minipage}
\,
\begin{minipage}[t]{0.15\textwidth}
\vspace{10pt}
\includegraphics[width=\textwidth]{genfigs/gis/gris-profile-gray.png}
\end{minipage}
\caption{A cross-section of the Greenland ice sheet at $70^\circ$N latitude \cite{Morlighemetal2017}; see inset.  Top: While the ice surface $s$ is smoothed because of ice flow, the bedrock elevation $b$ is much rougher.  Bottom: The corresponding ice thickness $H = s-b$ ``inherits'' the low regularity of $b$.}
\label{fig:giscross}
\end{figure}

Observe that NCP \eqref{eq:ncp} is the only part of the model where a time derivative appears.  Indeed, because the flow is very-viscous \cite{Acheson1990}, the Stokes sub-model acts as an instantaneous ``algebraic'' constraint on the evolution.  Therefore the coupled and infinite-dimensional glacier evolution problem is simultaneously a differential algebraic equation (DAE) system \cite{AscherPetzold1998,LofgrenAhlkronaHelanow2022} and a free-boundary, obstacle-type NCP.  While this is a standard physical model for an evolving glacier, as a mathematical model it is poorly understood.  No well-posedness theory is known to this author, although existence is known for a shallow limit \cite{JouvetBueler2012,PiersantiTemam2023}.  The current paper seems to be the first time that the non-shallow model is stated with (proposed) function spaces for all variables, including the surface elevation.

Furthermore, to the author's knowledge all existing evolution models using Stokes dynamics, or with any other form of a longitudinal stress balance \cite{Bueler2023}, use an explicit or semi-implicit time-stepping scheme for the geometry \cite[for examples]{Brinkerhoff2023,Durandetal2009,Jouvetetal2008,LofgrenAhlkronaHelanow2022,WirbelJarosch2020}, or are fully-implicit on a fixed map-plane region \cite{AhlkronaLofgrenHenry2025}.  Specifically, when implicitness is applied in these implementations the portion of $\RR^2$ covered by (solution) ice is not general.  Many schemes limit margin advance or retreat to one cell from the previous time step.  The current paper considers the full solution of a single time step of glacier evolution from the data, with no restrictions on the new ice-covered area.

For finite-dimensional DAE systems, implicit schemes are the standard choice, that is, to handle unbounded stiffness \cite{AscherPetzold1998}.  In the infinite-dimensional case here, where NCP \eqref{eq:ncp} is constrained by the algebraic Stokes problem \eqref{eq:glen}--\eqref{eq:stokes}, implicit schemes are certainly the natural choice.  In this work we consider a backward Euler semi-discretization.  For a time step $\Delta t > 0$, and times $\{t_n\}$ with $t_n-t_{n-1}=\Delta t$, applied to \eqref{eq:ncp} this scheme yields a new NCP for the updated surface elevation $s^n \approx s(t_n,x)$:
\begin{subequations}
\label{eq:be:ncp}
\begin{align}
s^n - b &\ge 0 \label{eq:be:ncp:constraint} \\
s^n - \Delta t\,\bu|_{s^n} \cdot \bn_{s^n} - \ell^n &\ge 0 \label{eq:be:ncp:residualpos} \\
(s^n - b) \left(s - \Delta t\,\bu|_{s^n} \cdot \bn_{s^n} - \ell^n\right) &= 0 \label{eq:be:ncp:complementarity}
\end{align}
\end{subequations}
For clarity we have collected a source term $\ell^n(x) = s^{n-1}(x) + \int_{t_{n-1}}^{t_n} a(t,x)\,dt$.

The backward Euler scheme \eqref{eq:be:ncp} is merely the simplest A-stable choice.  Extension to higher-order A-stable and stiff decay schemes \cite{AscherPetzold1998} is natural and straightforward.  On the other hand, a (fully) explicit version of \eqref{eq:be:ncp} would replace $s^n$ in the surface motion term by the previous surface elevation: $\bu|_{s^n} \cdot \bn_{s^n} \to \bu|_{s^{n-1}} \cdot \bn_{s^{n-1}}$ \cite{Lengetal2012}.  One possible semi-implicit scheme uses the surface velocity from the old time, but with the updated value for the surface slope part: $\bu|_{s^n} \cdot \bn_{s^n} \to \bu|_{s^{n-1}} \cdot \bn_{s^n}$ \cite{Durandetal2009}.  A more stable form of semi-implicitness comes from modifying the body force in the coupled Stokes problem using an explicit surface estimate \cite{LofgrenAhlkronaHelanow2022}.

In Section \ref{sec:model} we will re-write \eqref{eq:glen}--\eqref{eq:be:ncp} as a weak-form variational inequality (VI) for $s$ in an admissible subset of a proposed Banach space.  Based on conjectured well-posedness for this problem (Section \ref{sec:conjectural}), our main result follows, namely a new \emph{a priori} estimate on the numerical error in an FE approximation of this model (Sections \ref{sec:abstractestimate}, \ref{sec:application}).

Note that one can make a case for implicit time-stepping based on simulation performance, that is, by applying computational complexity scaling arguments which compare to various conditionally-stable explicit alternatives \cite{Bueler2023}.  However, actual performance advantages will depend on positive answers to the key questions:  Is the implicit-step VI problem well-posed?  Can it be solved accurately and efficiently?  Does the (coupled) implicit scheme have unconditional stability?  In the current paper we conjecture well-posedness and we then address FE accuracy.  Solver efficiency and time-stepping stability are topics for further research.

This paper is organized as follows.  Section \ref{sec:stokes} recalls the theory of the Stokes problem on a fixed 3D domain.  For this sub-model we give a new bound on the surface trace of the velocity solution (Lemma \ref{lem:surfacetracebound}).  In Section \ref{sec:model} we re-formulate the coupled, implicit-step NCP problem \eqref{eq:glen}--\eqref{eq:be:ncp} as a VI weak form.  The key coupling is the implicit surface motion term $\bu|_{s^n}\cdot \bn_{s^n}$ in \eqref{eq:be:ncp}.  For this term we conjecture a Lipschitz continuity bound over a Sobolev space of surface elevation functions.  Well-posedness for each implicit step VI problem, and the operator coercivity needed to apply the later \emph{a priori} bound on the FE solution, is then considered in Section \ref{sec:conjectural}.  Physical and modeling ideas are given as the context needed to understand Conjecture \ref{conj:regcoercive} which hypotheses the coercivity of a regularization of this surface motion term.  Numerical results support this major conjecture.  While well-posedness here is conjectural, no prior attempt to formulate a mathematically-precise coupled, non-shallow, free-boundary glacier model is known to the author.

With a temporally-discretized and spatially-continuous model in hand, we turn to its finite element (FE) approximation.  In Section \ref{sec:abstractestimate} we prove an error estimate, Theorem \ref{thm:abstractestimate}, apparently new for VI problems involving nonlinear operators on Banach spaces.  This estimate, which makes coercivity and Lipshitz assumptions on the operator, significantly extends a classical bilinear argument by Falk \cite{Falk1974}.  In Section \ref{sec:application} we apply the abstract estimate to the glacier problem, yielding our main result (Theorem \ref{thm:glacierapp}).  The significance of each term in the bound, and how associated FE method and glacier modeling choices can be made, is addressed in Sections \ref{sec:application} and \ref{sec:conclusion}.

We will use only the following few abbreviations which are standard in their respective fields: DAE (differential-algebraic equations), FE (finite element), NCP (nonlinear complementarity problem), PDE (partial differential equation), SKE (surface kinematical equation), SMB (surface mass balance), and VI (variational inequality).


\section{Surface velocity from the Stokes sub-model} \label{sec:stokes}

In this Section we address only the Stokes sub-model \eqref{eq:glen}--\eqref{eq:stokes}, applied on a domain $\Lambda \subset \RR^3$, defined by \eqref{eq:icydomain} at some time $t$.  This sub-model will compute the surface velocity $\bu|_s$ which appears in NCPs \eqref{eq:be:ncp}, but we defer discussion of the coupling to later sections.

Suitable function spaces for well-posedness of this sub-model are known.  Let $1 < \pp \le 2$.  Denote the Sobolev space \cite{Evans2010} of real-valued functions on $\Lambda$ with $\pp$th-power integrable first derivatives by $W^{1,\pp}(\Lambda)$.  Let $\cV = W_0^{1,\pp}(\Lambda; \RR^3)$ be the corresponding vector-valued functions with trace zero along the ice base $\Gamma_b\subset\partial \Lambda$, which is assumed to have positive measure.  For $[H]\ge 1$ a representative vertical glacier scale in meters, we define the norm
\begin{equation}
\|\bv\|_{\cV} = \left(\int_\Lambda |\bv|^\pp\,dx\,dz + [H]^\pp \int_\Lambda |\grad\bv|^\pp\,dx\,dz\right)^{1/\pp}. \label{eq:vnorm}
\end{equation}
Here $|\grad\bv|=\left(\grad\bv : \grad\bv\right)^{1/2}$ is the Frobenius norm on $\RR^{3\times 3}$, for $A:B=a_{ij}b_{ij}$.  The scale $[H]$ gives $\|\bv\|_{\cV}$ consistent units; compare Remark 1.2.1 in \cite{BoffiBrezziFortin2013}.  The volume element $dx\,dz = dx_1\,dx_2\,dz$ will be suppressed in integrals from now on.

Let $\cQ=L^{\pp'}(\Lambda)$ where $\pp'=\pp/(\pp-1)$.  For $(\bu,p), (\bv,q) \in \mathcal{M} = \cV \times \cQ$, the mixed velocity-pressure space, define
\begin{equation}
F_\Lambda(\bu,p)[\bv,q] = \int_\Lambda 2 \nu(D\bu) D\bu : D\bv - p \Div\bv - (\Div\bu) q - \rhoi \bg \cdot \bv. \label{eq:glenstokes:fcnl}
\end{equation}
The weak form of the Stokes sub-model \eqref{eq:glen}--\eqref{eq:stokes} seeks $(\bu,p) \in \mathcal{M}$ satisfying
\begin{equation}
F_\Lambda(\bu,p)[\bv,q] = 0 \qquad \text{for all } (\bv,q) \in \mathcal{M}. \label{eq:glenstokes:weak}
\end{equation}

Problem \eqref{eq:glenstokes:weak} is well-posed if $\partial\Lambda$ is appropriately regular, e.g.~piecewise $C^1$ \cite{JouvetRappaz2011} or polygonal \cite{Belenkietal2012}.  The well-posedness proofs in those references use specific constitutive relations, and our regularization \eqref{eq:glen} of the Glen law  matches that in \cite{Belenkietal2012,IsaacStadlerGhattas2015}.  Thus there exists a unique pair $(\bu,p) \in \mathcal{M}$ solving \eqref{eq:glenstokes:weak}, with $\bu\in \cV_0 = \{\bv\in\cV\,:\, \Div\bv=0\}$.  The following \emph{a priori} bound is proven, for completeness, in Appendix \ref{app:provestokesapriori}.

\begin{lemma} \label{lem:stokesapriori}
There is $C>0$ depending continuously on $\pp$, $\rhoi |\bg|$, $\nu_\pp$, $\mu_0$, $[H]$, and $\Lambda$ so that if $\bu\in\cV_0$ is the solution from \eqref{eq:glenstokes:weak} then
\begin{equation}
\|\bu\|_{\cV} \le C. \label{eq:stokesapriori}
\end{equation}
\end{lemma}

Our primary purpose is to study NCP \eqref{eq:be:ncp} for glacier geometry, specifically its time-discretized weak form, and for that we need control on the surface trace $\bu|_s$.  Theorem 5.5.1 in \cite{Evans2010} gives the following Lemma.

\begin{lemma}[Trace inequality] \label{lem:trace}
There exists a constant $C$, depending on $\pp$, $[H]$, and $\Lambda$, so that for all $\bv \in \cV$,
\begin{equation}
\int_{\Gamma_s} \big|\bv|_s\big|^\pp \,dS \le C \|\bv\|_{\cV}^\pp. \label{eq:trace}
\end{equation}
Here $dS$ denotes the area element on $\partial\Lambda$.
\end{lemma}

Combining Lemmas \ref{lem:stokesapriori} and \ref{lem:trace} yields the following bound on the surface trace of the velocity.  When applying this result in the next Section, recall that $\Lambda=\Lambda(t)$ and $\Gamma_s=\Gamma_s(t)$ are defined via \eqref{eq:icydomain}, in terms of $s(t,x)$ and $b(x)$.

\begin{lemma}[Surface velocity bound] \label{lem:surfacetracebound}
There is a constant $C>0$, computable from physical constants and the geometry of $\Lambda$, so that for $\bu\in\cV_0$ solving \eqref{eq:glenstokes:weak},
\begin{equation}
\int_{\Gamma_s} \big|\bu|_s\big|^\pp \,dS \le C. \label{eq:surfacetracebound}
\end{equation}
\end{lemma}


\section{The weak form of an implicit time step} \label{sec:model}

Consider an implicit time step for NCP \eqref{eq:be:ncp}.  Suppose $(t_n)$ are increasing times in $[0,T]$, with $t_0=0$, and write $\Delta t = t_n-t_{n-1}$ for a generic step.  Let $a^n(x)$ be the temporal average of the climatic data $a(t,x)$ over $[t_{n-1},t_n]$.  Suppose that $s^n(x)\approx s(t_n,x)$ approximates the surface elevation at time $t_n$.  The backward Euler scheme \cite{AscherPetzold1998} for SKE \eqref{eq:ske} is
\begin{equation}
\frac{s^n - s^{n-1}}{\Delta t} - \bu|_{s^n} \cdot \bn_{s^n} - a^n = 0. \label{eq:be:ske}
\end{equation}
Note how $s^n$ appears in both the surface velocity $\bu|_{s^n}$ and the slope $\bn_{s^n}$.  For cleaner appearance in what follows we will write $s=s^n$ for the unknown surface elevation, and we will collect a source term which is defined over all of $\Omega$:
\begin{equation}
\boxed{\ell^n(x) = s^{n-1}(x)+\Delta t\,a^n(x) = s^{n-1}(x) + \int_{t_{n-1}}^{t_n} a(t,x)\,dt.} \label{eq:be:source}
\end{equation}

As noted in the Introduction, $s$ solves a problem of free-boundary type, one which says that either there is bare ground ($s=b$) or equation \eqref{eq:be:ske} holds.  Leaving the precise Banach space $\cX$ to be determined, admissible surface elevations for this problem form a convex and closed cone:
\begin{equation}
\boxed{\cK = \left\{\sigma \in\cX\,:\,\sigma|_{\partial\Omega}=b|_{\partial\Omega} \text{ and } \sigma \ge b\right\}.}  \label{eq:be:admissible}
\end{equation}
(The Dirichlet boundary condition is included in the definition of $\cK$.)  We assume throughout that $b\in C^1(\bar\Omega)$ is smooth, with bounded gradient.

The weak form of the free-boundary problem is derived by assuming that $s \in \cK$ is a sufficiently-regular solution of \eqref{eq:be:ncp}.  Let $\Omega_I = \{x\in\Omega\,:\,s(x)>b(x)\}$ be the (measurable) subset on which the constraint \eqref{eq:be:ncp:constraint} is inactive, i.e.~where glacier ice is present.  By \eqref{eq:be:ncp:complementarity}, integration over $\Omega_I$ gives%
\begin{equation}
\int_{\Omega_I} \left(s - \Delta t\,\bu|_s \cdot \bn_s - \ell^n\right)\,(\sigma-s) = 0  \label{eq:inactivetruth}
\end{equation}
for any $\sigma\in\cK$.  On the other hand, let $\Omega_A = \Omega \setminus \Omega_I$ be the active (ice-free) region.  Using extension by zero \eqref{eq:defineus}, inequality \eqref{eq:be:ncp:residualpos} over $\Omega_A$ implies $b-\ell^n = s - \Delta t\,\bu|_s \cdot \bn_s - \ell^n \ge 0$.  Since also $\sigma-s=\sigma-b\ge 0$ on $\Omega_A$, integration yields an inequality:
\begin{equation}
\int_{\Omega_A} \left(s - \Delta t\,\bu|_s \cdot \bn_s - \ell^n\right)\,(\sigma-s) \ge 0.  \label{eq:activetruth}
\end{equation}
Adding \eqref{eq:inactivetruth} and \eqref{eq:activetruth} gives a variational inequality (VI; \cite{KinderlehrerStampacchia1980})
\begin{equation}
\int_\Omega \left(s - \Delta t\,\bu|_s \cdot \bn_s - \ell^n\right)\,(\sigma-s) \ge 0 \quad \text{for all } \sigma \in \cK. \label{eq:be:viearly}
\end{equation}
This VI holds true for $s\in\cK$, in advance of knowing which part of $\Omega$ is ice-covered.

It remains to identify a Sobolev space $\cX$ suitable for surface elevations.  However, within a physical Stokes-based theory, the shape that should be predicted for a glacier's grounded margin is not clear (Figure \ref{fig:margins}).  ``Wedge'' shapes with bounded gradients have been hypothesized from observations \cite{EchelmeyerKamb1986}, while shallow-ice theory suggests root-type (fractional-power) shapes, with different powers for advance and retreat \cite{Bueleretal2005,JouvetBueler2012}.  In real glacier margins the ice can overhang, especially on steep bedrock features, violating our assumption of a single-valued surface elevation, and fractures, crevasses, and cliffs in the ice are common.  Even the bedrock can overhang.  However, all major numerical models ignore overhangs \cite{IsaacStadlerGhattas2015,Jouvetetal2008,LofgrenAhlkronaHelanow2022,WirbelJarosch2020},
and fractures do not occur within the viscous-fluid physics of Stokes models.  The current work follows these patterns.  Future models might allow fractures, supplementing momentum conservation with an advected damage variable and a stress-failure criterion \cite{PralongFunk2005}, and the emergent macroscopic margin shapes might suggest a regularity assumption and Sobolev space for surface elevations in the viscous theory.

\begin{figure}[ht]
\begin{center}
\begin{tikzpicture}[scale=1.0]
  \node (wedge) {\includegraphics[width=25mm]{figs/wedge.png}} node[xshift=-4mm, yshift=1mm] at (wedge.center) {{\small \emph{ice}}};
  \node[right=of wedge] (unbounded) {\includegraphics[width=26mm]{figs/unbounded.png}} node[xshift=-2mm, yshift=1mm] at (unbounded.center) {{\small \emph{ice}}};
  \node[right=of unbounded] (realistic) {\includegraphics[width=26mm]{figs/realistic.png}} node[xshift=-6mm, yshift=4mm] at (realistic.center) {{\small \emph{ice}}};
\end{tikzpicture}
\end{center}

\vspace{-2mm}

\caption{Glacier margins with finite-slope ``wedge'' shapes (left) are possible, while shallow theory has fractional powers (center).  Actual glacier margins might have overhangs and fractures (right).}
\label{fig:margins}
\end{figure}

We now attempt to choose $\cX$ based on mathematical considerations within the viscous theory.  Well-posedness of the fixed-domain Stokes problem \eqref{eq:glenstokes:weak}, and the surface trace bound in Lemma  \ref{lem:surfacetracebound}, generates a well-defined map from an admissible surface elevation $s\in \cK$ to the surface motion term $\bu|_s\cdot \bn_s$ in \eqref{eq:be:viearly}.  To clarify this map we use domain notation which emphasizes dependence on the surface elevation:
\begin{equation}
\boxed{\Lambda(s) = \left\{(x,z)\,:\,b(x) < z < s(x)\right\} \subset \Omega \times \RR.} \label{eq:domainfroms}
\end{equation}
(This replaces the $t$-dependent notation in \eqref{eq:icydomain}.)  For a given input surface $s$, the proposed map first solves the Stokes model \eqref{eq:glenstokes:weak} over $\Lambda(s)$, then evaluates the trace of $\bu$ along $\Gamma_s$, which is extended by zero \eqref{eq:defineus}, and finally it multiplies by $\bn_s$.

For this to be well-defined, $s\in\cK\subset\cX$ must be sufficiently-regular.  For $\cX$ we propose a Sobolev space $W^{1,\rr}(\Omega)$ because of the next Lemma.  Let $[L]>0$ be a representative \emph{horizontal} scale for $\Omega\subset\RR^2$, and for $\omega\in W^{1,\rr}(\Omega)$ define the norm
\begin{equation}
\|\omega\|_{W^{1,\rr}} = \left(\int_\Omega |\omega|^\rr\,dx + [L]^\rr \int_\Omega |\grad \omega|^\rr\,dx\right)^{1/\rr}. \label{eq:norm:Omega}
\end{equation}
We now show that if $r\ge 2$ and $s\in W^{1,\rr}(\Omega)$ then the measurable function $\bu|_s\cdot \bn_s$ is in the dual space $(W^{1,\rr}(\Omega))'$.

\begin{lemma}[Bound on surface motion] \label{lem:phibound}  Suppose $2 \le \rr \le \infty$ and that $s\in W^{1,\rr}(\Omega)$ satisfies $s\ge b$ and $s=b$ on $\partial\Omega$.  Then there is a constant $C(\|s\|_{W^{1,\rr}})>0$ so that for all  $\omega\in W^{1,\rr}(\Omega)$,
\begin{equation}
\left|\int_\Omega \bu|_s\cdot \bn_s \,\omega\,dx\right| \le C(\|s\|_{W^{1,\rr}})\, \|\omega\|_{W^{1,\rr}}. \label{eq:phibound:early}
\end{equation}
\end{lemma}

\begin{proof}  Recall that $dS = |\bn_s|\,dx = \sqrt{1+|\grad s|^2}\,dx$ is the surface area element for $\Gamma_s \subset \partial \Lambda$.  Apply H\"older's inequalities:
\begin{align}
\Big|\int_\Omega \bu|_s\cdot \bn_s \, &\omega\,dx\Big| \le \int_\Omega \big|\bu|_s\big| |\bn_s|^{1/\pp} |\bn_s|^{1/\pp'} |\omega|\,dx \label{eq:phibound:zero} \\
    &\le \left(\int_\Omega \big|\bu|_s\big|^\pp |\bn_s|\,dx\right)^{1/\pp} \left(\int_\Omega |\bn_s| |\omega|^{\pp'} \,dx\right)^{1/\pp'} \notag \\
    &\le \left(\int_{\Gamma_s} |\bu|^\pp \,dS\right)^{1/\pp} \left(\int_\Omega |\bn_s|^\rr \,dx\right)^{1/(\pp'\rr)} \left(\int_\Omega |\omega|^{\pp'\rr'} \,dx\right)^{1/(\pp'\rr')}. \notag
\end{align}
From bound \eqref{eq:surfacetracebound} we may write
\begin{equation}
\left|\int_\Omega \bu|_s\cdot \bn_s \, \omega\,dx\right| \le C \left(\int_\Omega \left(1+|\grad s|^2\right)^{\rr/2}\,dx\right)^{1/(\pp'\rr)} \|\omega\|_{L^{\pp'\rr'}}.
\end{equation}
Note that if $\alpha\ge 0$ then $(1+\alpha)^{\rr/2} \le 2^{(\rr-2)/2} (1+\alpha^{\rr/2})$, so, using generic constants,
\begin{align}
\left|\int_\Omega \bu|_s\cdot \bn_s \, \omega\,dx\right| &\le C \left(2^{(\rr-2)/2} \int_\Omega 1 + |\grad s|^\rr\,dx\right)^{1/(\pp'\rr)} \|\omega\|_{L^{\pp'\rr'}} \label{eq:phibound:one} \\
  &\le C \left(|\Omega| + [L]^{-\rr}\|s\|_{W^{1,\rr}}^\rr\right)^{1/(\pp'\rr)} \|\omega\|_{L^{\pp'\rr'}}. \notag
\end{align}
Since $2 \le \pp' \le \pp'\rr' < \infty$, by Sobolev's inequality, e.g.~Theorem 8.8 from \cite{LiebLoss1997} using $n=2$, $k=m=1$, $p=\rr$, and $\omega=\pp'\rr'$, we have $\|\omega\|_{L^{\pp'\rr'}} \le C \|\omega\|_{W^{1,\rr}}$, thus \eqref{eq:phibound:early}.
\end{proof}

Because $\Omega\subset \RR^2$, if additionally $\rr>2$ then $W^{1,\rr}(\Omega) \hookrightarrow C(\bar\Omega)$.  Note that in Section \ref{sec:application} the FE approximation $s_h\approx s$ will be continuous and piecewise-linear.  Thus for $2 < \rr \le \infty$ we define from now on
\begin{equation}
\boxed{\cX = W^{1,\rr}(\Omega).} \label{eq:defineX}
\end{equation}
The admissible subset $\cK \subset \cX$ for surface elevations was already defined by \eqref{eq:be:admissible}.  By Lemma \ref{lem:surfacetracebound} the surface velocity $\bu|_s$ is well-defined as a measurable function, and its $L^\pp(\Gamma_s)$ norm is bounded.  By Lemma \ref{lem:phibound} the surface motion term $\bu|_s\cdot\bn_s$ is in $\cX'$.  Thus we define the nonlinear \emph{surface motion map} $\Phi:\cK \to \cX'$ by
\begin{equation}
\boxed{\Phi(s)[\omega] = -\int_\Omega \bu|_s\cdot\bn_s\,\omega\,dx.} \label{eq:definePhi}
\end{equation}

However, for well-posedness of VI \eqref{eq:be:viearly} we must conjecture that a norm of $\bu|_s$ is Lipschitz as a function of $s$.  Proving this apparently requires additional, detailed knowledge about Stokes problem \eqref{eq:glenstokes:weak}.  The dependence of scalar Laplacian-type equations on the domain is addressed in some literature, for example \cite{Daners2003}, but such results for Stokes problems are not known to this author.

\begin{conjecture}[Surface velocity is Lipschitz in the surface elevation] \label{conj:lipschitz}
There exists $2 < \rr \le \infty$ so that if $R>0$ then there is $C(R)>0$ so that for $\sigma,s\in B_R \cap \cK = \{t\in \cK\,:\,\|t\|_{\cX} \le R\}$ we have
\begin{equation}
\big\|\bu|_\sigma - \bu|_s\big\|_{L^{\rr'}} \le C(R) \|\sigma-s\|_{\cX} \label{eq:ulipschitz}
\end{equation}
\end{conjecture}

This Conjecture implies that $\Phi$ is Lipschitz, as needed for well-posedness of \eqref{eq:be:viearly}.

\begin{definition} \label{def:lipshitz}
For $R>0$ let $B_R = \{v\in \cX\,:\,\|v\|\le R\}$.  We say that a map $f:\cK \to \cX'$ is \emph{Lipshitz on bounded subsets of $\cK$} if for every $R>0$ there is $C(R)>0$ so that
\begin{equation}
\|f(v)-f(w)\|_{\cX'} \le C(R) \|v-w\|_{\cX} \quad \text{ for all } v,w \in B_R \cap \cK.  \label{eq:liponbounded}
\end{equation}
\end{definition}

\begin{lemma} \label{lem:philipschitz}  The map $\Phi$ is Lipschitz on bounded subsets of $\cK$.
\end{lemma}

\begin{proof}  Assume without loss of generality that $R\ge \|b\|_\cX$, so $b\in B_R \cap \cK$.  Suppose $\sigma,s\in B_R \cap \cK$.  For any $\omega\in\cX$, in the following we add and subtract $\bu|_s \cdot \bn_\sigma$, and use the triangle inequality $|\bn_\sigma|=\left(1+|\grad \sigma|^2\right)^{1/2} \le 1 + |\grad \sigma|$:
\begin{align}
\Big|\Phi(\sigma)[\omega] - \Phi(s)[\omega]\Big| &\le \int_\Omega \Big|\bu|_\sigma - \bu|_s\Big| |\bn_\sigma| |\omega|\,dx + \int_\Omega \big|\bu|_s\big| |\bn_\sigma-\bn_s| |\omega|\,dx \\
    &\le \int_\Omega \Big|\bu|_\sigma - \bu|_s\Big| |\omega|\,dx + \int_\Omega \Big|\bu|_\sigma - \bu|_s\Big| |\grad \sigma| |\omega|\,dx \notag \\
    &\qquad\qquad + \int_\Omega \big|\bu|_s\big| |\grad \sigma-\grad s| |\omega|\,dx \notag
\end{align}
By applying H\"older's inequality to each of these integrals we have
\begin{align}
\int_\Omega \Big|\bu|_\sigma - \bu|_s\Big| |\omega|\,dx &\le \big\|\bu|_\sigma - \bu|_s\big\|_{L^{\rr'}} \|\omega\|_{L^{\rr}} \le \|\bu|_\sigma - \bu|_s\|_{L^{\rr'}} \|\omega\|_{\cX}, \label{eq:philipschitz:1} \\
\int_\Omega \Big|\bu|_\sigma - \bu|_s\Big| |\grad \sigma| |\omega|\,dx &\le \left(\int_\Omega \Big|\bu|_\sigma - \bu|_s\Big|^{\rr'} |\omega|^{\rr'}\, dx\right)^{1/\rr'} \|\grad \sigma\|_{L^\rr} \label{eq:philipschitz:2} \\
    &\le [L]^{-1} \big\|\bu|_\sigma - \bu|_s\big\|_{L^{\rr'}} \|\sigma\|_{\cX} \|\omega\|_{L^\infty}, \notag \\
\int_\Omega \big|\bu|_s\big| |\grad \sigma-\grad s| |\omega|\,dx &\le \left(\int_\Omega \big|\bu|_s\big|^{\rr'} |\omega|^{\rr'}\, dx\right)^{1/\rr'} \|\grad \sigma- \grad s\|_{L^\rr}  \label{eq:philipschitz:3} \\
    &\le [L]^{-1} \big\|\bu|_s - \bzero\big\|_{L^{\rr'}} \|\sigma-s\|_{\cX} \|\omega\|_{L^\infty}. \notag
\end{align}
Note that $\bu|_b=\bzero$; there is no flow when there is no glacier.  Because $\rr>2$, Sobolev's inequality gives $\|\omega\|_{L^\infty} \le c_\infty \|\omega\|_\cX$ for some $c_\infty>0$.  Now apply Conjecture \ref{conj:lipschitz} to each of \eqref{eq:philipschitz:1}--\eqref{eq:philipschitz:3}:
\begin{equation}
\big|\Phi(\sigma)[\omega] - \Phi(s)[\omega]\big| \le \tilde C(R) \left(1 + c_\infty [L]^{-1} \left(\|\sigma\|_{\cX} + \|s - b\|_{\cX}\right)\right) \|\sigma-s\|_{\cX} \|\omega\|_{\cX}.
\end{equation}
Recalling that $b\in B_R\cap \cK$, use the triangle inequality again to conclude with $C(R) = \tilde C(R) \left(1 + 3 c_\infty [L]^{-1} R\right)$.
\end{proof}

At this point we have the tools needed to make the continuum free-boundary model for a backward-Euler time-step \eqref{eq:be:ske} mathematically precise.

\begin{definition} For $\Delta t>0$, $s\in\cK$, and $\omega\in\cX$ define the nonlinear (Stokes) \emph{geometry update operator} $F_{\Delta t}:\cK\to\cX'$ as
\begin{equation}
\boxed{F_{\Delta t}(s)[\omega] = \Delta t\,\Phi(s)[\omega] + \int_\Omega s \omega.}  \label{eq:be:Fdefine}
\end{equation}
Assume that $\ell^n$, defined from data by \eqref{eq:be:source}, is in $\cX'$.  We say that the surface elevation $s=s^n \in \cK$ solves the (weak-form) \emph{implicit time-step problem} if
\begin{equation}
\boxed{F_{\Delta t}(s)[\sigma-s] \ge \ell^n[\sigma-s] \quad \text{for all } \sigma \in \cK.} \label{eq:be:vi}
\end{equation}
\end{definition}

If Conjecture \ref{conj:lipschitz} holds then $\Phi$ and $F_{\Delta t}$ are well-defined and Lipschitz on bounded subsets.  The reader should confirm that the weak-form \eqref{eq:be:vi} then merely rewrites \eqref{eq:be:viearly}.  This VI gives a precise meaning to strong-form NCP \eqref{eq:be:ncp}.

The seven boxed definitions and inequalities above, namely \eqref{eq:be:source}, \eqref{eq:be:admissible}, \eqref{eq:domainfroms}, \eqref{eq:defineX}, \eqref{eq:definePhi}, \eqref{eq:be:Fdefine}, and \eqref{eq:be:vi}, form a continuum model for a single time-step of glacier evolution based on Stokes dynamics.  Although Conjecture \ref{conj:lipschitz} is used in the construction of the operator $F_{\Delta t}$, the whole model is precise enough to be mathematically analyzed, and a future existence and/or uniqueness result for \eqref{eq:be:vi} is possible.  However, we will regularize the model in the next Section.  Based on numerical evidence, well-posedness for this regularization is credible.  Sections \ref{sec:abstractestimate} and \ref{sec:application} then prove an error estimation theorem for an FE approximation of the regularized model.


\section{Conjectural well-posedness for the regularized problem} \label{sec:conjectural}

Theorem \ref{thm:glacierapp}, our main result to come, is numerical.  It bounds the FE surface elevation error by extending a technique from Falk \cite{Falk1974} to a large class of nonlinear operators.  This is tied to the concepts of monotonicity \cite{Minty1963} and coercivity \cite[Chapter III]{KinderlehrerStampacchia1980}.

\begin{definition} \label{def:monotonecoercive}
Suppose $\cX$ is a Banach space, with norm $\|\cdot\|$, and $\cK\subset \cX$ is closed and convex.  An operator $f:\cK \to \cX'$ is said to be \emph{monotone} if
\begin{equation}
\left(f(v)-f(w)\right)[v-w] \ge 0 \qquad \text{for all } v,w \in \cK, \label{eq:monotone}
\end{equation}
and \emph{strictly monotone} if equality implies $v=w$.  It is \emph{coercive} if there is $w\in \cK$ so that $\left(f(v)-f(w)\right)[v-w]/\|v-w\| \to +\infty$ for $v \in \cK$ as $\|v\| \to +\infty$.  It is \emph{$\qq$-coercive} for $\qq>1$ if there exists $\alpha>0$ such that
\begin{equation}
\left(f(v)-f(w)\right)[v-w] \ge \alpha \|v-w\|^\qq \qquad \text{for all } v,w \in \cK. \label{eq:qcoercive}
\end{equation}
\end{definition}

Note that $\qq$-coercivity \cite{Bueler2021conservation} implies coercivity and strict monotonicity, and that $2$-coercivity is sometimes called \emph{strong monotonicity} \cite{Chow1989}.

Theorem \ref{thm:abstractestimate} below applies to VIs for operators $f:\cK\to\cX'$ which are $\qq$-coercive.  We would like to apply that Theorem to $F_{\Delta t}$ \eqref{eq:be:Fdefine} acting on an admissible subset $\cK$ of $W^{1,\rr}(\Omega)$.  However, examples\footnote{These apparently-new examples also negatively resolve a previously open question of uniqueness, for elevation-independent SMB, in the shallow ice approximation.} in Appendix \ref{app:noncoercive} show that the surface motion map $\Phi$ \eqref{eq:definePhi} is not $\qq$-coercive, nor strictly monotone, over $W^{1,\rr}(\Omega)$ for any $\rr$, at least for any bedrock $b$ possessing local minima.  Since $F_{\Delta t}$ adds the integral $\int_\Omega s\omega$ to the surface motion term $\Delta t\,\Phi(s)[\omega]$, lack of coercivity for $\Phi$ does not dis-prove it for $F_{\Delta t}$.

However, the numerical experiments in Appendix \ref{app:numerical} show that the surface motion operator $\Phi$ is \emph{nearly} $\qq$-coercive.  In these experiments thousands of high-resolution surface pairs $\sigma,s\in \cK\subset W^{1,\rr}(\Omega)$ were generated from a 1D glacier simulation.  Coercivity ratios for $\Phi$ were computed with exponents $\rr=\qq=4$,
\begin{equation}
\frac{(\Phi(\sigma) - \Phi(s))[\sigma - s]}{\|\sigma-s\|_{W^{1,4}}^{4}}, \label{eq:ratios}
\end{equation}
using surface velocities from a numerical Stokes model (when evaluating $\Phi(s)$ and $\Phi(\sigma)$).  Recalling definition \eqref{eq:qcoercive}, if $\Phi$ were $4$-coercive over $\cX=W^{1,4}(\Omega)$, and if computations were exact, then these ratios would all exceed a constant $\alpha>0$.  Appendix \ref{app:numerical} shows that under mesh refinement the pairs yielding negative ratios become quite rare.  Standard presentations of Stokes models for glaciers give no reason to expect these ratios should prefer positive values, but the mean and median ratios are decidely positive, and under refinement they stabilize away from zero.

In our particular Stokes model we can express the surface motion $\Phi(s)$ in a different manner which suggests coercivity.  Extend $\omega\in \cX$ to $\Lambda(s)$ as $\tilde\omega(x,z)=\omega(x)$, and define the glaciological ice flux $\bq(s) = \int_b^s \bU\,dz$ \cite{SchoofHewitt2013}.  (Here $\bU=(u_1,u_2)$ is the horizontal velocity, from $\bu=(u_1,u_2,w)$.)  Using definition \eqref{eq:definePhi}, the non-sliding assumption ($\bu|_{\Gamma_b}=\bzero$), the divergence theorem, and incompressibility yields
\begin{align}
\Phi(s)[\omega] &= -\int_{\Omega} \bu|_s \cdot \bn_s\,\omega\,dx = -\int_{\partial\Lambda(s)} \tilde\omega \bu\cdot {\hat\bn}\,dS = - \int_{\Lambda(s)} \Div\left(\tilde\omega \bu\right)  \label{eq:fluxformPhi} \\
  &= - \int_{\Lambda(s)} \bu\cdot \grad \tilde\omega = -\int_\Omega \bq(s) \cdot \grad \omega, \notag
\end{align}
where $\hat\bn$ is the outward unit normal and $dS$ is the area element along $\partial\Lambda(s)$.  Now, the natural glaciological belief is that ice flows downhill, which is the informal association $\bq(s) \sim - \grad s$.  Under that heuristic, formula \eqref{eq:fluxformPhi} implies $(\Phi(\sigma) - \Phi(s))[\sigma - s] \sim \int_\Omega |\grad s - \grad \sigma|^2$.  On the other hand, the shallow ice existence theory for shear-thinning ice in \cite{JouvetBueler2012} suggests that $W^{1,2}(\Omega)$ is not the space in which to seek the surface elevation solution.

\newcommand{\wSIA}{\tilde w}
We will instead regularize the surface motion term as follows.  The isothermal shallow ice approximation \cite{GreveBlatter2009} yields an expression for the surface vertical motion:
% Gamma = \frac{\nn+1}{\nn+2} 2 (rho g)^2 / (\nn+2),  i.e. (n+1)/(n+2) times the usual Gamma
\begin{equation} \label{eq:verticalvelocitysia}
\wSIA|_s = \Div \left(\Gamma (s-b)^{\nn+1} |\grad s|^{\nn-1} \grad s\right).
\end{equation}
Here $\nn$ is Glen's exponent and $\Gamma>0$ is an ice softness constant computable from the viscosity scale $\nu_\pp$ in \eqref{eq:glen}.  Formula \eqref{eq:verticalvelocitysia} degenerates at the glacier margin, where the thickness $s-b$ goes to zero, so we regularize using a fixed ice thickness $\eta_0>0$.  Let $\nn=3$ and $\eps>0$.  Writing the Stokes velocity in components $\bu = (u_1,u_2,w)$, for $\omega \in \cX$ we define this regularized weak formula to replace \eqref{eq:definePhi}:
\begin{equation} \label{eq:defineregularizedPhi}
\Phi^\eps(s)[\omega] = \int_\Omega \Big(\big(u_1|_s,u_2|_s\big) \cdot \grad s - (1-\eps) w|_s\Big) \omega + \eps\, \Gamma \eta_0^4 |\grad s|^2 \grad s \cdot \grad \omega.
\end{equation}
This replaces the Stokes vertical velocity $w|_s$ with the convex combination $(1-\eps)w|_s + \eps\, \wSIA|_s$.  If $\Phi$ is Lipschitz on bounded subsets of $\cK$ (Lemma \ref{lem:philipschitz}) then so is $\Phi^\eps$.  Regularization \eqref{eq:defineregularizedPhi} might generate a $4$-coercive operator over $W^{1,4}(\Omega)$ because of properties of the (weak) $\pp$-Laplacian operator \cite[for example]{ChoeLewis1991}; the regularization is a multiple of the $4$-Laplacian, which is $4$-coercive over $W^{1,4}(\Omega)$.

The numerical experiments in Appendix \ref{app:numerical} use values $\eps=0.1$ and $\eta_0=1000$ meters in \eqref{eq:defineregularizedPhi}.  The observed effect is that negative coercivity ratios \eqref{eq:ratios} disappears at high spatial resolution, and that the experimental distribution of ratios is bounded away from zero.  However, we are not able to \emph{prove} that $\Phi^\eps$ in \eqref{eq:defineregularizedPhi} is $4$-coercive.  A proof would seem to depend on novel and nontrivial insights into the glaciological Stokes problem.  The numerical analysis in Section \ref{sec:application} will therefore be based on the following second conjecture.

\begin{conjecture}[Regularized surface motion $\Phi^\eps(s)$ is $4$-coercive over admissible surface elevations in $W^{1,4}$] \label{conj:regcoercive}  Suppose Conjecture \ref{conj:lipschitz} holds for $\rr=4$, and let $\cX = W^{1,4}(\Omega)$.  Fix $b\in C^1(\bar\Omega)$ and let $\cK=\{\sigma\in\cX\,:\,\sigma\ge b \text{ and } \sigma|_{\partial\Omega}=b|_{\partial\Omega}\}$.  There exists $\eps \in (0,1)$ and $H_0>0$, and $\alpha>0$, so that
\begin{equation}
\left(\Phi^\eps(\sigma) - \Phi^\eps(s)\right)[\sigma-s] \ge \alpha \|\sigma-s\|_{\cX}^4 \qquad \text{for all } \sigma,s\in\cK. \label{eq:regcoercive}
\end{equation}
\end{conjecture}

To complete this Section we prove that this Conjecture suffices for well-posedness of the regularized VI problem; compare \eqref{eq:be:Fdefine}, \eqref{eq:be:vi} for the unregularized problem.

\begin{theorem} \label{thm:regularizedwellposed}  Assume Conjectures \ref{conj:lipschitz} and \ref{conj:regcoercive}.  Suppose that $s^{n-1}\in\cK$ and define the source term $\ell^n \in \cX'$ as in \eqref{eq:be:source}.  For positive $\Delta t,\eps,\eta_0$ define
\begin{equation}
F^\eps_{\Delta t}(s)[\omega] = \Delta t\,\Phi^\eps(s)[\omega] + \int_\Omega s \omega, \label{eq:regularizedF}
\end{equation}
the regularized update operator.  This operator is Lipschitz on bounded subsets and $4$-coercive, thus there exists a unique surface elevation $s\in\cK$ satisfying the VI problem%
\begin{equation}
F^\eps_{\Delta t}(s)[\sigma-s] \ge \ell^n[\sigma-s] \quad \text{for all } \sigma \in \cK. \label{eq:regularizedvi}
\end{equation}
\end{theorem}

\begin{proof}  The Lipschitz and $4$-coercive inequalities follow straightforwardly from the Conjectures.  In particular, $F^\eps_{\Delta t}$ is $4$-coercive over $\cK$ with constant $\alpha\Delta t>0$, and thus it is also coercive and strictly-monotone (Definition \ref{def:monotonecoercive}).  Corollary III.1.8 of \cite{KinderlehrerStampacchia1980} shows unique existence for \eqref{eq:regularizedvi}.
\end{proof}

Theorem \ref{thm:regularizedwellposed} addresses only the well-posedness of a single time-step, over $[t_{n-1},t_n]$.  Its conclusion is certainly not sufficient to show well-posedness of the time-dependent VI problem corresponding to NCP \eqref{eq:ncp}, nor to show that implicit steps converge in the $\Delta t\to 0$ limit.  However, it is a first mathematical step in these directions.  Time-stepping numerical models for the evolution of glacier geometry, using Stokes dynamics, are already in common use.  Practitioners apparently expect that each time-step problem, implicit or not, is well-posed, and that computed surface elevations will converge to continuum solutions in some well-behaved manner.


\section{Abstract error estimate for finite element approximation of VIs} \label{sec:abstractestimate}

In this Section we consider the FE approximation of an abstract VI problem in a Banach space.  We will return to the glaciological problem \eqref{eq:regularizedvi} in Section \ref{sec:application}.

Let $\cX$ be a real, reflexive Banach space with norm $\|\cdot\|$ and topological dual space $\cX'$.  Denote the dual pairing of $\ell \in \cX'$ and $v\in\cX$ by $\ell[v]$, and define $\|\ell\|_{\cX'} = \sup_{\|v\|=1} \big|\ell[v]\big|$.  Let $\cK \subset \cX$ be a nonempty, closed, and convex subset, the admissible set.  For a continuous and (generally) nonlinear operator $f:\cK \to \cX'$, and a source functional $\ell\in \cX'$, the VI problem is to find $u\in \cK$ such that
\begin{equation}
f(u)[v-u] \ge \ell[v-u] \quad \text{for all } v\in \cK. \label{eq:vi}
\end{equation}
While \eqref{eq:regularizedvi} is in this form, the best known example is the obstacle problem for the Laplacian operator; see \cite{Ciarlet2002,Evans2010,KinderlehrerStampacchia1980} for theory and FE analysis.  If $f:\cK \to \cX'$ is monotone and coercive (Definition \ref{def:monotonecoercive}), and continuous on finite-dimensional subspaces, then problem \eqref{eq:vi} has a solution \cite[Corollary III.1.8]{KinderlehrerStampacchia1980}, and if $f$ is strictly monotone then the solution is unique.  If $f$ is $\qq$-coercive then it coercive and strictly monotone, and if $f$ is Lipschitz on bounded subsets (Definition \ref{def:lipshitz}) then it is continuous.  Recall that $f(u)-\ell \in \cX'$ is generally nonzero if $u$ solves \eqref{eq:vi}, but $f(u)-\ell=0$ when $u$ is in the interior of $\cK$.  Under sufficient regularity assumptions, an NCP strong form generally follows from \eqref{eq:vi}.

Now suppose $\cX_h \subset \cX$ is a finite-dimensional subspace, typically some FE space of continuous, piecewise-polynomial functions defined on a mesh.  Let $\cK_h\subset \cX_h$ be a closed and convex subset, and suppose $f_h:\cK_h\to\cX'$ is an approximation of $f$.  Generally $\cK_h \nsubset \cK$---we will \emph{not} assume initially that FE states are continuum admissible---and generally $f_h\ne f$ because of quadrature and other approximations.  In fact, in standard FE Stokes-type glacier simulations both $\cK_h \nsubset \cK$ and $f_h\ne f$ apply for practical reasons, though in Section \ref{sec:application} we arrange that $\cK_h\subset\cK$.

The FE method for \eqref{eq:vi} is the finite-dimensional VI problem
\begin{equation}
f_h(u_h)[v_h-u_h] \ge \ell[v_h-u_h] \quad \text{for all } v_h\in \cK_h. \label{eq:fe:vi}
\end{equation}
We will assume that this problem has a solution $u_h\in\cK_h$.  The following \emph{a priori} estimation theorem extends the well-known result of Falk \cite{Falk1974}; see also Theorem 5.1.1 in \cite{Ciarlet2002}.  We assume that $f$ is coercive and Lipschitz on an (unnamed) larger set than $\cK$, which contains the FE solution $u_h$, a technical assumption permiting a clean and general result.  However, we do \emph{not} assume any of the following: $\cK_h \subset \cK$, $f$ is linear, $f_h=f$, $f_h$ is continuous, $f_h$ is $\qq$-coercive, or that the solution of \eqref{eq:fe:vi} is unique.

\begin{theorem} \label{thm:abstractestimate}  Suppose $u\in\cK$ solves \eqref{eq:vi} and $u_h\in\cK_h$ solves \eqref{eq:fe:vi}.  For $\qq>1$, with conjugate exponent $\qq'=\qq/(\qq-1)$, assume that $f$ is $\qq$-coercive, with constant $\alpha>0$, on a set which contains $\cK$ and the numerical solution $u_h$.  Assume $f$ is Lipschitz on bounded subsets of the same set.  Let $R_h=\max\{\|u\|,\|u_h\|\}$.  Then there is a constant $c(R_h,\alpha)>0$, not otherwise depending on $u$ or $u_h$, so that
\begin{align}
\|u-u_h\|^\qq &\le \frac{2}{\alpha} \left(\inf_{v\in\cK} \left(f(u)-\ell\right)[v-u_h] + \inf_{v_h\in\cK_h} \left(f(u)-\ell\right)[v_h-u]\right)\label{eq:abstractestimate} \\
   &\quad\, + \frac{2}{\alpha} \left(f(u_h)-f_h(u_h)\right)[u_h] + c(R_h,\alpha) \inf_{v_h\in\cK_h} \|v_h - u\|^{\qq'}. \notag
\end{align}
\end{theorem}

\begin{proof}  It follows from $\qq$-coercivity of $f$ that
\begin{align}
\alpha \|u-u_h\|^\qq &\le \left(f(u)-f(u_h)\right)[u-u_h] \notag \\
  &= f(u)[u] + f(u_h)[u_h] - f(u)[u_h] - f(u_h)[u] \notag \\
  &= f(u)[u] + f_h(u_h)[u_h] - f(u)[u_h] - f(u_h)[u] + \left(f(u_h)-f_h(u_h)\right)[u_h]. \label{eq:abstract:one}
\end{align}
Next, for arbitrary $v\in\cK$ and $v_h\in\cK_h$, rewrite \eqref{eq:vi} as $f(u)[u]     \le f(u)[v] + \ell[u-v]$ and \eqref{eq:fe:vi} as $f_h(u_h)[u_h] \le f_h(u_h)[v_h] + \ell[u_h-v_h]$.  Then
\begin{align}
\alpha \|u-u_h\|^\qq &\le f(u)[v] + \ell[u-v] + f(u_h)[v_h] + \ell[u_h-v_h] \label{eq:abstract:two} \\
  &\qquad - f(u)[u_h] - f(u_h)[u] + \left(f(u_h)-f_h(u_h)\right)[u_h] \notag \\
  &= \left(f(u)-\ell\right)[v-u_h] + \left(f(u)-\ell\right)[v_h-u] \notag \\
  &\qquad + \left(f(u)-f(u_h)\right)[u-v_h] + \left(f(u_h)-f_h(u_h)\right)[u_h] \notag
\end{align}
Since $u,u_h\in B_{R_h}$, by the Lipschitz assumption there is $C(R_h)>0$ so that
\begin{equation}
\left(f(u)-f(u_h)\right)[u-v_h] \le C(R_h) \|u-u_h\|\|u-v_h\|. \label{eq:abstract:three}
\end{equation}
Noting $1<\qq<\infty$, use Young's inequality with $\eps>0$ \cite[Appendix B.2]{Evans2010} on \eqref{eq:abstract:three}:
\begin{align}
\alpha \|u-u_h\|^\qq &\le \left(f(u)-\ell\right)[v-u_h] + \left(f(u)-\ell\right)[v_h-u]  \label{eq:abstract:four} \\
  &\qquad + C(R_h) \left(\eps\|u-u_h\|^\qq + \tilde C(\eps) \|u-v_h\|^{\qq'}\right) \notag \\
  &\qquad + \left(f(u_h)-f_h(u_h)\right)[u_h], \notag
\end{align}
where $\tilde C(\eps) = (\eps \qq)^{-\qq'/\qq} {\qq'}^{-1}$.  Choose $\eps>0$ so that $C(R_h) \eps \le \alpha/2$, and subtract:
\begin{align}
\frac{\alpha}{2} \|u-u_h\|^\qq &\le \left(f(u)-\ell\right)[v-u_h] + \left(f(u)-\ell\right)[v_h-u]  \label{eq:abstract:five} \\
  &\qquad + C(R_h) \tilde C(\eps) \|u-v_h\|^{\qq'} + \left(f(u_h)-f_h(u_h)\right)[u_h] \notag
\end{align}
Take infimums to show \eqref{eq:abstractestimate}.
\end{proof}

We can significantly refine the estimate in the case of a unilateral obstacle problem, where $\cX$ are scalar functions.  In this case the residual $f(u)-\ell\in \cX'$ is a positive measure $d\mu_u$ supported in the active set \cite[Theorem II.6.9]{KinderlehrerStampacchia1980}.

\begin{corollary}  \label{cor:abstractwithmeasure}  Suppose $\cX=W^{1,\rr}(\Omega)$ over a domain $\Omega\subset\RR^d$, and suppose $\cK\subset\cX$ is defined as in \eqref{eq:be:admissible} for an obstacle $b$.  Let $A_u=\{x\in\Omega\,:\,u(x)>b(x)\}$ be the active set.  Under the assumptions of Theorem \ref{thm:abstractestimate} we have the bound
\begin{align}
\|u-u_h\|^\qq &\le \frac{2}{\alpha} \left(\inf_{v\in\cK} \int_{A_u} v - u_h\,d\mu_u + \inf_{v_h\in\cK_h} \int_{A_u} v_h - b\,d\mu_u\right)
  \label{eq:abstractwithmeasure} \\
   &\quad\, + \frac{2}{\alpha} \left(f(u_h)-f_h(u_h)\right)[u_h] + c \inf_{v_h\in\cK_h} \|v_h - u\|^{\qq'}. \notag
\end{align}
\end{corollary}

The ``$\inf_{v\in\cK}$'' terms in bound \eqref{eq:abstractwithmeasure} is generally nonzero when $\cK_h \not\subset \cK$.  To see how this can occur even for an arbitrarily smooth obstacle, with $\cK$ defined as in \eqref{eq:be:admissible}, suppose $b_h=\pi_h b$ is the FE interpolant of $b$, and define $\cK_h=\{v_h \in \cX_h\,:\,v_h\ge b_h \text{ and } v_h|_{\partial\Omega}=b_h|_{\partial\Omega}\}$.  While $b_h(x_j)=b(x_j)$ for the interpolation nodes $x_j$, generally $b_h(x) \ge b(x)$ will not hold for all $x\in\Omega$ (Figure \ref{fig:nonadmissible}; compare \cite[Figure 5.1.3]{Ciarlet2002}).  That is, nodal admissibility does not imply admissibility.

\begin{figure}[ht]
\begin{center}
\begin{tikzpicture}[scale=1.0, domain=0.0:4.0, samples=200]
  \draw[black,thin,->] (-0.3,0.0) -- (4.5,0.0) node [xshift=2mm] {$x$};
  \draw plot (\x, {1.5 + 0.8 * cos(1.8*\x r)});
  \node[yshift=-2mm] at (2.0, 0.7) {$\psi$};
  \newcommand{\xlist}{0.0, 1.0, 1.4, 2.4, 3.0, 4.0}
  \foreach \x [remember=\x as \lastx] in \xlist {
      \draw (\x, 0.0) circle (1.5pt);
      \filldraw (\x, {1.5 + 0.8 * cos(1.8*\x r)}) circle (1.5pt);
      \draw[dashed] (\lastx, {1.5 + 0.8 * cos(1.8*\lastx r)}) -- (\x, {1.5 + 0.8 * cos(1.8*\x r)});
  }
  \node[yshift=3mm] at (3.0, 0.0) {$x_i$};
  \node[xshift=3mm, yshift=-7mm] at (0.0, 2.3) {$\psi_h$};
\end{tikzpicture}
 \quad \begin{tikzpicture}[scale=1.1, domain=0.0:4.0, samples=200]
  \draw[black,thin,->] (-0.3,0.0) -- (4.5,0.0) node [xshift=2mm] {$x$};
  \draw plot (\x, {1.5 + 0.8 * cos(1.8*\x r)});
  \newcommand{\xlist}{0.0, 1.0, 1.4, 2.4, 3.0, 4.0}
  \newcommand{\xymaxlist}{0.0/2.3, 1.0/2.3, 1.4/1.3182, 2.4/2.0078, 3.0/2.3, 4.0/2.3}
  \foreach \x/\y in \xymaxlist {
      \draw (\x, 0.0) circle (1.5pt);
      \filldraw (\x, \y) circle (1.5pt);
  }
  \node[yshift=3mm] at (3.0, 0.0) {$x_i$};
  \draw[dashed] (0.0,2.3) -- (1.0,2.3) -- (1.4,1.3182) -- (2.4,2.0078) -- (3.0,2.3) -- (4.0,2.3);
  \node[yshift=-2mm] at (2.0, 0.7) {$\psi$};
  \node[xshift=5mm, yshift=-3mm] at (1.0, 2.3) {$\psi_h$};
\end{tikzpicture}

\end{center}
\caption{Left: If $b_h=\pi_h b$ is the interpolant of $b$ then generally $\cK_h\nsubset\cK$.  Right: Defining $b_h=R_h^{\oplus} b$, using \eqref{eq:monotoneop}, implies $b_h\ge b$ and $\cK_h\subset\cK$.}
\label{fig:nonadmissible}
\end{figure}

For unilateral obstacle problems and $P_1$ or $Q_1$ elements we may force $\cK_h\subset\cK$ by using a monotone restriction operator \cite{BuelerFarrell2024}, defined as follows.  For a node $x_i$ let $N_i$ be the union of elements adjacent to $x_i$.  Define $R_h^{\oplus}b\in\cX_h$ by
\begin{equation}
(R_h^{\oplus}b)(x_i) = \sup_{x \in N_i} b(x). \label{eq:monotoneop}
\end{equation}
If $b_h=R_h^{\oplus} b$ then $b_h(x)\ge b(x)$ for all $x\in \Omega$ (Figure \ref{fig:nonadmissible}; right), thus $\cK_h\subset \cK$ and the ``\,$\inf_{v\in\cK}$'' term vanishes from the bounds.

The original result by Falk \cite{Falk1974}, which computes residual norms, loses the active-set information in Corollary \ref{cor:abstractwithmeasure}.  The next Corollary makes this historical connection by applying a norm-based approach.  It recovers the Falk result \cite{Falk1974} when $\cX,\cB$ are Hilbert spaces, and $f(v)[w]=f_h(v)[w]=a(v,w)$ is bilinear, uniformly-elliptic, and continuous.  (The definition of uniform ellipticity coincides with definition \eqref{eq:qcoercive} of $2$-coercive, and continuity implies \eqref{eq:liponbounded}.)  With these simplifications, Corollary \ref{cor:abstractestimate:Bnorm} reduces to Theorem 1 in \cite{Falk1974}.

\begin{corollary}  \label{cor:abstractestimate:Bnorm}  Suppose that $\cX$ continuously and densely embeds into a larger Banach space: $\cX \hookrightarrow \cB$, $\bar{\cX} = \cB$, and $\cB' \subset \cX'$.  Suppose $f:\cK\to\cB'$ and $\ell\in\cB'$.  Under the assumptions of Theorem \ref{thm:abstractestimate} we have
\begin{align}
\|u-u_h\|^\qq &\le \frac{2}{\alpha} \|f(u)-\ell\|_{\cB'} \left( \inf_{v\in\cK} \|v-u_h\|_{\cB} +   \inf_{v_h\in\cK_h} \|v_h-u\|_{\cB} \right) \label{eq:abstractestimate:Bnorm} \\
   &\qquad + \frac{2}{\alpha} \left(f(u_h)-f_h(u_h)\right)[u_h] + c\,\inf_{v_h\in\cK_h} \|v_h - u\|^{\qq'}. \notag
\end{align}
\end{corollary}

Note that if $u$ is in the interior of $\cK$, and if $f=f_h$, then all bounds \eqref{eq:abstractestimate}, \eqref{eq:abstractwithmeasure}, and \eqref{eq:abstractestimate:Bnorm} reduce to their last error term.  Thus we get Cea's lemma \cite[Theorem 2.4.1]{Ciarlet2002}, but now stated in a Banach space for a coercive nonlinear operator.  For glaciers this case occurs when the entire domain $\Omega$ is covered by ice.  In summary, Theorem \ref{thm:abstractestimate} is a powerful and nontrivial generalization of Falk's bound \cite{Falk1974} and Cea's lemma.


\section{Application to numerical glacier models} \label{sec:application}

Now we can apply the above theory to a time-step of an evolving-surface, Stokes-based glacier simulation.  Our approach essentially defines the phrase ``conforming FE method'' for such models.  We will combine the major threads so far: a surface velocity bound for the glaciological Stokes problem on a fixed domain (Section \ref{sec:stokes}), conjectural Lipschitz continuity and $4$-coercivity of the regularized surface-motion operator (Sections \ref{sec:model} and \ref{sec:conjectural}), and an abstract error estimate for FE solutions of unilateral obstacle problems (Section \ref{sec:abstractestimate}).

Practical models for glaciers, based on unstructured meshes in the 2D map-plane, use polygonal domain geometry for the 3D Stokes problem \cite{Brinkerhoff2023,IsaacStadlerGhattas2015,Jouvetetal2008,Lengetal2012}.  For this reason we assume that the domain $\Omega \subset \RR^2$ is a polygon, and that the discrete bed and surface elevations $b_h,s_h$ are from the continuous $P_1$ FE space $\cX_h$ over a triangulation $\cT_h$ of $\Omega$ (Figure \ref{fig:fe:domain}).  The 3D mesh need not be extruded vertically from $\cT_h$, as shown in Figure \ref{fig:fe:domain}, but this is a possibility.

\begin{figure}[ht]
\begin{center}
\includegraphics[width=0.7\textwidth]{genfigs/extruded.pdf}
\end{center}
\caption{The Stokes problem for the velocity $\bu_h$ is solved on a meshed polygonal domain, between $P_1$ functions $b_h$ and $s_h$.  Computation of $F_{\Delta t}^{\eps,h}$ requires evaluation of the surface trace $\bu_h|_{s_h}$.}
\label{fig:fe:domain}
\end{figure}

We will assume that the original bed elevation $b$ is well-behaved, $b\in C^1(\bar\Omega)$.  In practice $b$ is provided via a high resolution map derived from ice-penetrating radar \cite{Morlighemetal2017}, often over a much finer mesh than $\cT_h$.  For the discrete bed $b_h\in\cT_h$ we suppose $b_h=b$ along $\partial\Omega$.  By application of monotone restriction, $b_h = R_h^{\oplus} b$ \eqref{eq:monotoneop}, or otherwise, we achieve $b_h \ge b$; this eliminates a term in bound \eqref{eq:abstractestimate} when Theorem \ref{thm:abstractestimate} is applied below.  Now, following \eqref{eq:be:admissible}, we define $\cK_h=\{\sigma_h \in \cX_h\,:\,\sigma_h|_{\partial\Omega}=b_h|_{\partial\Omega} \text{ and } \sigma_h\ge b_h\}$, so that $\cK_h\subset\cK \subset \cX$ where $\cX = W^{1,4}(\Omega)$.

From these geometric and discrete-space choices, recall the regularised VI problem \eqref{eq:regularizedvi}, with $F^\eps_{\Delta t}$ defined by \eqref{eq:defineregularizedPhi} and \eqref{eq:regularizedF}.  Approximate evaluation of $F^\eps_{\Delta t}(s_h)$ requires the numerical solution of the glaciological Stokes problem \eqref{eq:glenstokes:weak} over a meshed 3D domain $\Lambda(s_h)$; see equation \eqref{eq:domainfroms}.  The upper and lower surfaces of $\Lambda(s_h)$, where boundary conditions \eqref{eq:stokes:stressfreesurface} and \eqref{eq:stokes:noslide} are applied, are given by admissible FE functions $s_h \in \cK_h$ and $b_h\in\cX_h$.  Solvability of this Stokes problem using inf-sup stable elements is addressed by \cite{Belenkietal2012,JouvetRappaz2011}.  The numerical velocity from solving \eqref{eq:glenstokes:weak} over $\Lambda(s_h)$ is denoted $\bu_h$, and its surface trace is denoted $\bu_h|_{s_h}$.  (Note that $\bu_h$ is distinct from the exact solution of the (same) Stokes problem over $\Lambda(s_h)$.)  We make no assumptions on the solution process for \eqref{eq:glenstokes:weak}, except that it must generate an approximate surface trace for use in evaluating $\Phi^\eps(s_h)$ and $F^\eps_{\Delta t}(s_h)$. 

These preliminaries allow us to define the regularized and numerical operator.  Following \eqref{eq:defineregularizedPhi} and \eqref{eq:regularizedF}, for $\Delta t>0$ and $s_h,\omega \in \cK_h$ we define
\begin{align}
F^{\eps,h}_{\Delta t}(s_h)[\omega] &= \int_\Omega s_h \omega + \Delta t\,\int_\Omega \Big(\big(u^h_1|_{s_h},u^h_2|_{s_h}\big) \cdot \grad s_h - (1-\eps) w^h|_{s_h}\Big) \omega \label{eq:fe:regularizedF} \\
  &\qquad + \Delta t \int_\Omega \eps\, \Gamma H_0^4 |\grad s_h|^2 \grad s_h \cdot \grad \omega, \notag
\end{align}
where $\bu_h = (u^h_1,u^h_2,w^h)$ is from numerically solving over $\Lambda(s_h)$.  This numerical operator $F^{\eps,h}_{\Delta t}(s_h)$ uses the FE function $s_h$ directly (in the above integrals) and indirectly (in defining the domain of the Stokes problem for $\bu_h$).  Corresponding to the continuum problem \eqref{eq:regularizedvi}, we seek $s_h\in\cK_h$ solving the VI
\begin{equation}
F^{\eps,h}_{\Delta t}(s_h)[\sigma_h-s_h] \ge \ell^n[\sigma_h-s_h] \quad \text{for all } \sigma_h \in \cK_h. \label{eq:fe:regularizedvi}
\end{equation}
The source term $\ell^n$ is defined exactly as in \eqref{eq:be:source}; straightforward adjustments apply if quadrature is used in the integral.  The previous surface elevation is general, $s^{n-1} \in \cK$, so our analysis applies to the first time-step.

Collecting the above, we identify some standard assumptions regarding the data and function spaces used when solving VI problem \eqref{eq:regularizedvi} by numerical scheme \eqref{eq:fe:regularizedvi}.  Note that the conforming condition $\cK_h\subset \cK$ follows from assumption \ref{item:goodbh}.

\begin{stdass}
The following data are given:
\renewcommand{\labelenumi}{\arabic{enumi}.}
\begin{enumerate}
\item A bounded, polygonal domain $\Omega\subset\RR^2$.
\item A time-dependent SMB function $a\in C([0,T]; L^{4/3}(\Omega))$.
\item A bed elevation function $b \in C^1(\bar\Omega)$, piecewise-linear along $\partial\Omega$.
\end{enumerate}
We make these continuum definitions:
\begin{enumerate}
\setcounter{enumi}{3}
\item $\cX = W^{1,4}(\Omega)$, with the norm defined in \eqref{eq:norm:Omega}.
\item $\cK = \{\sigma\in\cX\,:\,\sigma|_{\partial \Omega} = b|_{\partial \Omega} \text{ and } \sigma \ge b\}$.
\end{enumerate}
We assume this FE configuration:
\begin{enumerate}
\setcounter{enumi}{5}
\item $\cT_h$ is a triangulation of $\bar\Omega$.
\item $\cX_h \subset \cX$ is the continuous $P_1$ space over $\cT_h$. \label{item:pone}
\item $b_h\in\cX_h$ satisfies $b_h\ge b$ in $\Omega$ and $b_h=b$ along $\partial \Omega$. \label{item:goodbh}
\item $\cK_h = \{\sigma_h\in\cX_h\,:\,\sigma_h|_{\partial \Omega} = b_h|_{\partial \Omega} \text{ and } \sigma_h \ge b_h\}$. \label{item:defineKh}
\end{enumerate}
\end{stdass}

For the continuum solution $s\in\cK$, from Theorem \ref{thm:regularizedwellposed}, application of Theorem \ref{thm:abstractestimate} with $\qq=4$ yields the following Lemma.  Because $s_h\in\cK_h\subset \cK$, the first term in \eqref{eq:abstractestimate} disappears and the technical set construction in Theorem \ref{thm:abstractestimate} is not needed.

\begin{lemma} \label{lem:preglacierapp}  Make the Standard Assumptions, and assume Conjecture \ref{conj:lipschitz} holds for $\rr=4$, and assume Conjecture \ref{conj:regcoercive} holds for parameters $\eps\in(0,1)$, $\eta_0>0$, and $\alpha>0$.  Suppose that $s^{n-1}\in\cK$ and define $\ell^n \in \cX'$ by \eqref{eq:be:source}.  For $\Delta t>0$ let $s\in\cK$ be the unique surface elevation satisfying the regularized implicit time-step VI problem \eqref{eq:regularizedvi}, and assume that $s_h\in\cK_h$ is any solution of numerical problem \eqref{eq:fe:regularizedvi}.  Let $R_h=\max\{\|s\|_\cX,\|s_h\|_\cX\}$.  Then there is a constant $c_0=c_0(R_h,\alpha\Delta t)>0$ so that
\begin{align}
\|s-s_h\|_\cX^4 &\le \quad \frac{2}{\alpha \Delta t} \inf_{\sigma_h\in\cK_h} \left(F^\eps_{\Delta t}(s)-\ell^n\right)[\sigma_h-s] \label{eq:preglacierestimate} \\
   &\quad\, + \frac{2}{\alpha \Delta t} \left(F^\eps_{\Delta t}(s_h)-F^{\eps,h}_{\Delta t}(s_h)\right)[s_h] \notag \\
   &\quad\, + c_0 \inf_{\sigma_h\in\cK_h} \|\sigma_h - s\|_{\cX}^{4/3}. \notag
\end{align}
\end{lemma}

The meaning of each term in bound \eqref{eq:preglacierestimate} turns out to be reasonably clear, especially in the refined form from the next Theorem.  Recall that $\cV=W_0^{1,\pp}(\Lambda(s_h); \RR^3)$ is the velocity space for Stokes problem \eqref{eq:glenstokes:weak}.  For $\omega\in\cX$ define $\Pi_h(\omega)\in\cX_h$ by
\begin{equation}
\Pi_h(\omega)(x_j) = \max \,\{b_h(x_j), \omega(x_j)\}, \label{eq:definePi}
\end{equation}
over nodes $x_j \in \cT_h$.  If $\sigma\in \cK$ then $\Pi_h(\sigma) \in \cK_h$ because nodal admissibility implies admissibility under assumption \ref{item:pone}.

\begin{theorem} \label{thm:glacierapp}  Make the same assumptions as in Lemma \ref{lem:preglacierapp}.  Define
\begin{equation}
\Omega_A(s) = \left\{x\in\Omega\,:\,s(x)=b(x)\right\},
\end{equation}
the closed ice-free region (active set) for the exact solution.  Then
\begin{align}
\|s_h-s\|_\cX^4 &\le \quad \frac{2}{\alpha \Delta t} \int_{\Omega_A(s)}  (b_h - b) (b - \ell^n) &&\text{\textnormal{[term 1]}} \label{eq:glacierestimate} \\
   &\quad\, + \frac{2G}{\alpha} \big\|\bu_h - \bu\big\|_{\cV} + \frac{2\eps}{\alpha} \left\|(w^h - w)|_{s_h}\right\|_{L^\pp} \|s_h\|_{L^{\pp'}} &&\text{\textnormal{[term 2]}} \notag \\
   &\quad\, + c_0 \|\Pi_h(s) - s\|_\cX^{4/3}, &&\text{\textnormal{[term 3]}} \notag
\end{align}
where the coefficient $G=G(\|s_h\|_{\cX})$ is derived in the proof.
\end{theorem}

\begin{proof}  Because $s$ solves \eqref{eq:regularizedvi}, the residual $F^\eps_{\Delta t}(s)-\ell^n \in \cX'$ is represented by a positive Borel measure $\mu$ \cite[Theorem 6.22]{LiebLoss1997}.  However, by the proof of Theorem II.6.9 in \cite{KinderlehrerStampacchia1980} this measure is supported in $\Omega_A(s)$, and it has density $b-\ell^n$ (Section \ref{sec:model}; argument leading to \eqref{eq:be:viearly}).  Apply Lemma \ref{lem:preglacierapp} and note that $\bu|_{s}=\bzero$ and $s=b$ on $\Omega_A(s)$.  Set $\sigma_h = b_h \in \cK_h$ to give term 1 in \eqref{eq:preglacierestimate}:
\begin{equation}
\left(F^\eps_{\Delta t}(s)-\ell^n\right)[b_h-s] = \int_\Omega (b_h - s) \,d\mu = \int_{\Omega_A(s)} (b_h - b) (b - \ell^n).
\end{equation}

For term 2, first expand \eqref{eq:regularizedF} and \eqref{eq:fe:regularizedF}:
\begin{align}
\frac{1}{\Delta t} &\big(F^\eps_{\Delta t}(s_h)-F^{\eps,h}_{\Delta t}(s_h)\big)[s_h] = \int_\Omega \Big(\big(u_1|_{s_h},u_2|_{s_h}\big) \cdot \grad s_h - (1-\eps) w|_{s_h}\Big) s_h \label{eq:techythm} \\
&\hspace{45mm} - \int_\Omega \Big(\big(u^h_1|_{s_h},u^h_2|_{s_h}\big) \cdot \grad s_h - (1-\eps) w^h|_{s_h}\Big) s_h \notag \\
&\qquad = - \int_\Omega \left(\bu|_{s_h} - \bu_h|_{s_h}\right) \cdot \bn_{s_h} \,s_h + \eps \int_\Omega \left(w|_{s_h} - w^h|_{s_h}\right) s_h \notag
\end{align}
(This form assumes that quadrature for $\cX_h$ is such that the integrals in \eqref{eq:fe:regularizedF} are computed exactly.)  For the first of these integrals, apply the triangle inequality, the H\"older inequality (twice), the trace inequality (Lemma \ref{lem:trace}), and a Sobolev inequality:
\begin{align}
\Big|\int_\Omega \left(\bu|_{s_h} - \bu_h|_{s_h}\right) \cdot &\bn_{s_h} \,s_h\Big| \le \int_\Omega \Big|\bu|_{s_h} - \bu_h|_{s_h}\Big| |\bn_{s_h}|^{1/\pp} |\bn_{s_h}|^{1/\pp'} |s_h| \\
 &\le \left(\int_\Omega \Big|\bu|_{s_h} - \bu_h|_{s_h}\Big|^\pp |\bn_{s_h}|\right)^{1/\pp} \left(\int_\Omega |\bn_{s_h}| |s_h|^{\pp'}\right)^{1/\pp'} \notag \\
  &\le \left(\int_{\Gamma_{s_h}} \big|\bu - \bu_h\big|^\pp\right)^{1/\pp} \left(\int_\Omega |\bn_{s_h}|^4\right)^{1/(4\pp')} \|s_h\|_{L^{4\pp'/3}} \notag \\
  &\le C \|\bu - \bu_h\|_{\cV} \left(\int_\Omega 1 + |\grad s_h|^4\right)^{1/(4\pp')} \|s_h\|_{\cX}  \notag \\
  &= G(\|s_h\|_{\cX}) \|\bu - \bu_h\|_{\cV}.  \notag
\end{align}
(Also recall that $|\bn_{s_h}|\,dx$ is the area element for $\Gamma_{s_h} \subset \partial \Lambda(s_h)$.)  Note that $C$ depends nontrivially on $\Lambda(s_h)$, and and that $G(\|s_h\|_{\cX})$ simply collects the factors shown.  H\"older applied the second integral from \eqref{eq:techythm} completes term 2.

Term 3 follows by substituting $\sigma_h=\Pi_h(s)$ into the last term in \eqref{eq:preglacierestimate}.
\end{proof}

Now we consider the meaning of each term in the bound \eqref{eq:glacierestimate}:

\medskip
\begin{itemize}
\item[term 1:]  This term comes from any bed elevation approximation errors within the unknown, exact ice-free area $\Omega_A(s)$.  If the bed is exactly represented ($b_h=b$) then this term is zero.  In portions of the computed ice-free area $\Omega_A(s_h)$ which are far from the nearest flowing glacier, and where the climate is ablating ($a < 0$), one might simply declare that no ice geometry error is made, by accepting $b_h$ as representing bare land $b$.  However, one cannot neglect term 1 near the unknown, exact glacier margin $\Omega \cap \partial \Omega_A(s)$, the free boundary.  Putting the free boundary in the wrong place on inaccurate bed topography inherently generates an ice surface elevation error.

\medskip
\item[term 2:]  This term quantifies how numerical velocity errors over the domain $\Lambda(s_h)$, from solving Stokes problem \eqref{eq:glenstokes:weak}, affect the numerical surface elevation error.  If \eqref{eq:glenstokes:weak} were solved exactly then this term would be zero.  The second contribution to term 2 scales with the regularization $\eps$ used in \eqref{eq:regularizedF}.

\medskip
\item[term 3:]  An interpolation error term like this arises in the classical Cea's lemma argument for quasi-optimality of FE methods for PDEs \cite{Ciarlet2002}.  Here $\Pi_h(s)$ \eqref{eq:definePi} is the $P_1$ interpolant of $s$, but also nonlinearly truncated into $\cK_h$ to achieve discrete admissibility.
\end{itemize}

\medskip
One might try to derive a convergence rate from Theorem \ref{thm:glacierapp}, perhaps starting by applying interpolation theory to term 3.  From \cite[Theorem 3.1.6]{Ciarlet2002}, for example, if $\mu \in [4,+\infty]$ then for $\sigma \in W^{2,\mu}(\Omega)$ we have $\|\pi_h(\sigma) - \sigma\|_{\cX} \le C h \|\sigma\|_{W^{2,\mu}}$.  (Here $\pi_h$ is ordinary interpolation into $\cX_h$, without truncation into $\cK_h$.)  Suppose also that the exact solution $s\in\cK$ satisfies $s\in W^{2,\mu}$ for some $\mu \in [\rr,+\infty]$.  It would follow that term 3 was $O(h)$ with a coefficient depending on $\|s\|_{W^{2,\mu}}$.  (Theorem 4.3 in \cite{JouvetBueler2012} makes a comparably-strong assumption in a shallow ice model.)  However, $s\in W^{2,\mu}$ would imply a glacier margin which is tangential to the bed.  For real glaciers the surface gradient generally does not approach the bed gradient at the ice margin; while $s \in \cX = W^{1,4}$ is credible, $s\in W^{2,\mu}$ is not.  Thus, attempting to prove convergence via Theorem \ref{thm:glacierapp}, via direct use of FE interpolation theory, is difficult.

Consideration of term 2 adds further difficulties.  Suppose one assumes solution regularity for the Stokes problem \eqref{eq:glenstokes:weak}, specifically that $\bu\in W^{2,\kappa}(\Lambda(s_h);\RR^3)$ and $p \in W^{1,\kappa'}(\Lambda(s_h))$ for some $\kappa \in [\pp,2]$.  Suppose that a mixed FE method for that problem satisfies Bramble-Hilbert interpolation bounds for the discrete velocity and pressure spaces, respectively, and that the method satisfies a discrete inf-sup condition; for example, see inequalities (4.1), (4.26), (4.27) in \cite{JouvetRappaz2011}.  One then concludes that $\|\bu_h - \bu\|_{\cV} \le C h^{\kappa/2}$, for a constant $C>0$ which depends on the regularity norms of $\bu,p$, the discrete inf-sup constant, \emph{and the domain} $\Lambda(s_h)$.  The last dependence is problematic for convergence of the coupled problem.  For example, the domain $\Lambda(s_h)$ will change under 2D mesh refinement of $\Omega$, which is necessary to reduce terms 1 and 3.  Concretely, one would need to know how the constant in the $O(h^{\kappa/2})$ velocity convergence rate depends on $s_h$.

In a different direction, note that the mass-conservation barrier theory in \cite{Bueler2021conservation} argues that any time-stepping FE solution of a fluid-layer VI problem, like the glacier problem here, commits a mass conservation error near the (unknown) exact free boundary.  However, that theory is phrased in terms of layer thickness, where the zero obstacle is exactly represented.  Term 1 in \eqref{eq:glacierestimate} is thus new relative to the errors identified in \cite{Bueler2021conservation}.


\section{Discussion and conclusion} \label{sec:conclusion}

We know of no existing literature of Stokes-based glacier geometry evolution which is mathematical in a comparable sense to this paper.  Specifically, a coupled weak form for the glacier evolution problem with Stokes dynamics, which is the only conceivable basis for rigorously addressing that problem, seems not to have appeared in the literature.  Nor has a Sobolev space for the surface elevation functions, in such a theory, been proposed.  However, numerical solvers for this, until now largely unstated, continuum problem are in wide-spread use for glacier modeling.

FIXME no results known to the author prove such well-posedness, \emph{nor for any glacier geometry evolution problem based on Stokes dynamics}

The major result of this paper is Theorem \ref{thm:glacierapp}, which bounds the numerical surface elevation error, in a single backward Euler time step, when a glacier is modeled using non-shallow Stokes dynamics.  The bound, inequality \eqref{eq:glacierestimate}, only estimates the FE error made in solving VI problem \eqref{eq:be:vi}.  The first two terms in the bound can be reduced by improved bed elevation interpolation, and by solving the Stokes problem more accurately, respectively.  All terms in the bound are reduced by increased mesh resolution over the fixed domain $\Omega\subset \RR^2$.  However, the surface elevation solution to \eqref{eq:be:vi} must be expected to have low regularity, especially across the ice margin (free boundary).  Near-margin mesh refinement may be the only technique which reduces the interpolation/truncation error in the surface elevation, namely term 3 in \eqref{eq:glacierestimate}.

The results here leave us far from an FE convergence proof for the main time-evolution problem of glaciology, which is the problem stated in the Introduction.  This problem couples NCP \eqref{eq:ncp} to the Stokes problem \eqref{eq:glen}--\eqref{eq:stokes}.  It seems to be a parabolic VI \cite{Glowinski1984}, on which analysis may be more difficult than for the (roughly-speaking) elliptic single-step VI problem \eqref{eq:be:vi}.  Converting Theorem \ref{thm:glacierapp} to a convergence proof for the time-dependent theory would require significant extensions.

We do not possess a proof of the well-posedness of the continuous-space, implicit-step problem \eqref{eq:be:vi} itself.  While part of the current paper is devoted to conjecturing such well-posedness (Sections \ref{sec:model}--\ref{sec:conjectural}), the abstract FE bound in Theorem \ref{thm:abstractestimate} applied to problem \eqref{eq:be:vi} is the main result, that is, Theorem \ref{thm:glacierapp}.  The abstract bound applies because of Theorem \ref{thm:regularizedwellposed}, a consequence of Conjectures \ref{conj:a}, \ref{conj:b}, and \ref{conj:c}.  The heart of the matter is coercivity of the surface motion, namely Conjecture \ref{conj:c}, but a strategy for proving this is not clear to this author.  Our results here at least clarify which properties of the surface motion, as part of a free-boundary problem for the surface kinematical equation, need to hold so that mathematically sensible time steps can be taken in a non-shallow glacier model.


%\clearpage\newpage
\bibliographystyle{siamplain}
\bibliography{estimate}

\appendix
\section{Proof of Lemma \ref{lem:stokesapriori}} \label{app:provestokesapriori}

The proof below of Lemma \ref{lem:stokesapriori} is hinted in \cite{JouvetRappaz2011}, and it is standard.  It requires the following Poincar\'e's and Korn's inequalities, which can be proven by (7.44) in \cite{GilbargTrudinger2001} for \eqref{eq:poincare}, and by setting $F(x)$ to the identity in Corollary 4.1 of \cite{Pompe2003} for \eqref{eq:korns}.

\begin{lemma} \label{lem:poincarekorns}
There exist dimensionless constants $C$, depending on the geometry of $\Lambda$, so that for all $\bv \in \cV$,
\begin{equation}
\int_\Lambda |\bv|^\pp \le C [H]^\pp \int_\Lambda |\grad\bv|^\pp, \label{eq:poincare}
\end{equation}
thus $\|\bv\|_{\cV}^\pp \le (C + 1) [H]^\pp \int_\Lambda |\grad\bv|^\pp$, and
\begin{equation}
\int_\Lambda |\grad\bv|^\pp \le C \int_\Lambda |D\bv|^\pp. \label{eq:korns}
\end{equation}
\end{lemma}

\begin{proof}[Proof of Lemma \ref{lem:stokesapriori}]
From \eqref{eq:glenstokes:weak} and $\bu \in\Vdiv$ it follows that
\begin{equation}
0= F_\Lambda(\bu,p)[\bu,p] = \int_\Lambda 2 \nu(D\bu) D\bu : D\bu - \rhoi \bg \cdot \bu.  \label{eq:stokes:substituteu}
\end{equation}
Apply \eqref{eq:korns}, the facts that $\pp>0$ and $(\pp-2)/2 \le 0$, and equation \eqref{eq:glen}:
\begin{align}
\int_\Lambda |\grad\bu|^\pp &\le C \int_\Lambda |D\bu|^\pp \le C \int_\Lambda \left(|D\bu|^2 + \mu_0\right)^{(\pp-2)/2} \left(D\bu:D\bu + \mu_0\right) \label{eq:stokes:startapriori} \\
	&= C \left[\mu_0^{\pp/2} |\Lambda| + (2 \nu_\pp)^{-1} \int_\Lambda 2\nu(D\bu) D\bu:D\bu\right]. \notag
\end{align}
By equation \eqref{eq:stokes:substituteu}, Cauchy-Schwarz, H\"older's inequality, and \eqref{eq:poincare} we thus have
\begin{equation}
\int_\Lambda |\grad\bu|^\pp \le C [H]^\pp \left[\mu_0^{\pp/2} |\Lambda| + (2 \nu_\pp)^{-1} \rhoi |\bg| |\Lambda|^{1/\pp'} \|\bu\|_\cV\right]. \label{eq:stokes:workapriori}
\end{equation}
Now let $z=\|\bu\|_\cV$.  We have proved that $z^\pp \le c_0 + c_1 z$ for $\pp>1$ and constants $c_i>0$.  Furthermore $g(y) = y^\pp - c_1 y - c_0$ is smooth with $g(0)=-c_0<0$, and $g(y) \to +\infty$ as $y \to +\infty$ since $\pp>1$.  Thus there exists a right-most root $\tilde y>0$ with $\tilde y = f(\pp,c_0,c_1)$.  Since $g(z)\le 0$ we have $z \le \tilde y$.  This proves \eqref{eq:stokesapriori} with $C=\tilde y$.
\end{proof}


\section{The surface motion map is not coercive} \label{app:noncoercive}

It is reasonable to suppose that a glacier flow model will produce no flow for an admissible surface elevation $s\in \cX=W^{1,r}(\Omega)$ which is flat (horizontal) everywhere on the ice.  In terms of the inactive set $\Omega_I=\{x\in\Omega\,:\,s(x)>b(x)\}$ and the domain $\Lambda(s)$ from \eqref{eq:icydomain}, this property says that if $\grad s=0$ a.e.~in $\Omega_I$ then $\bu=\bzero$ on $\Lambda(s)$.  This holds for the non-sliding Stokes model considered in the paper, but it would also hold for most sliding laws, and in shallow ice approximation models \cite{JouvetBueler2012}.

\begin{figure}[ht]
\begin{center}
\mbox{\includegraphics[width=0.4\textwidth]{genfigs/filled.pdf} \qquad \includegraphics[width=0.4\textwidth]{genfigs/icefree.pdf}}
\end{center}
\caption{For a smooth bedrock elevation $b$ with local minima, one surface elevation state $s_1$ fills-in the minima with flat ice (left), while the other is ice-free $s_2=b$ (right).  Because neither state generates flow, $\bu|_{s_i}=\bzero$, the surface motion map $\Phi$ in equation \eqref{eq:definePhi} cannot be coercive.}
\label{fig:noncoercive}
\end{figure}

Using this property, for any bedrock elevations $b$ possessing local minima we can construct distinct surface elevations $s_1,s_2\in \cX$, $\|s_1-s_2\|_{\cX}>0$, with zero surface velocity $\bu|_{s_i}=\bzero$ \eqref{eq:defineus}, and therefore zero surface motion ($\Phi(s_i)=0$ from \eqref{eq:definePhi}).  The construction (Figure \ref{fig:noncoercive}) simply fills-in the local minima with flat ice for surface $s_1$, and leaves $s_2=b$ as bare ground.  Note that $s_1$ can be constructed with any number of local minima filled-in, or with minima only partially filled.  Two consequences of this construction follow, as propositions.

\begin{proposition} \label{prop:noncoercive}
The surface motion map $\Phi$ defined in \eqref{eq:definePhi} is not $\qq$-coercive (Definition \ref{def:monotonecoercive}), for any $\qq$, nor is it strictly-monotone.
\end{proposition}

\begin{proof}
Note $(\Phi(s_1) - \Phi(s_2))[s_1-s_2]=0$ while $\|s_1-s_2\|_\cX > 0$.
\end{proof}

\begin{proposition} \label{prop:notunique}
When the SMB is identically zero, $a=0$, there are smooth bedrock elevations for which the steady-state version of a glacier geometry model \eqref{eq:ncp} will have more than one solution.
\end{proposition}

\begin{proof}
The strong (NCP) form of the steady-state and $a=0$ model is
	$$s - b \ge 0, \quad - \bu|_s \cdot \bn_s \ge 0, \quad (s - b) \left(- \bu|_s \cdot \bn_s\right) = 0,$$
with corresponding weak form VI similar to \eqref{eq:be:vi}.  If $b$ is a smooth function with local minima then the above construction of $s_1$ and $s_2$ gives distinct solutions.
\end{proof}
Proposition \ref{prop:notunique} answers negatively, via a surprisingly-simple construction which seems not to appear in the literature, the uniqueness question which has been open since the proof of existence for the general-bed, steady shallow ice approximation \cite{JouvetBueler2012}.  Non-uniqueness here is for an elevation-\emph{in}dependent SMB function, namely $a=0$.  This result is very different from the better-known non-uniqueness \cite{Bodvardsson1955} and non-existence \cite{Jouvetetal2011} results for certain elevation-dependent SMB models.


\section{Experiments on coercivity of the surface motion map} \label{app:numerical}

Conjecture \ref{conj:regcoercive} represents a well-posedness framework for the continuum implicit time-step problem for glaciers (Sections \ref{sec:stokes}--\ref{sec:model}).  It is critical to the application of \emph{a priori} Theorem \ref{thm:abstractestimate}, in our main goal of bounding the surface elevation errors in FE simulations (Theorem \ref{thm:glacierapp}).  The reasonableness of the Conjecture can be tested by sampling from numerical simulations.

The numerical experiments here, using Python and the Firedrake FE library \cite{Hametal2023},\footnote{Source code is at the public repository \href{https://github.com/bueler/glacier-fe-estimate}{\texttt{github.com/bueler/glacier-fe-estimate}}, in the \texttt{py/} directory.  The codes call the library at \href{https://github.com/bueler/stokes-extrude}{\texttt{github.com/bueler/stokes-extrude}}.} are not intended to demonstrate time-stepping.  They simply generate admissible surface elevation pairs $\sigma,s\in\cK \subset \cX = W^{1,4}(\Omega)$, to use as samples in a coercivity test.  For each pair of surfaces we evaluate the coercivity ratios based on formula \eqref{eq:defineregularizedPhi}:
\begin{equation}
\rho^\eps(\sigma,s) = \frac{\left(\Phi^\eps(\sigma) - \Phi^\eps(s)\right)[\sigma-s]}{\|\sigma-s\|_{\cX}^4}. \label{eq:Phiratio}
\end{equation}
In the experiments below we use $\eps=0$ first and then $\eps=0.1$.

Appendix \ref{app:noncoercive} shows that for bumpy beds there exist pairs that give $\rho^0(\sigma,s) = 0$.  However, by coercivity of the $4$-Laplacian, for $\eps>0$ these pairs generate $\rho^\eps(\sigma,s)>0$.

If the set of all possible ratios $\{\rho^\eps(\sigma,s)\}_{\sigma,s\in\cK}$ were bounded below by a positive constant $\alpha>0$, then this would confirm bound \eqref{eq:regcoercive} and Conjecture \ref{conj:regcoercive}.  However, an experiment obviously allows only finite sampling using particular data and discretizations.  Here the domain is the 1D interval $\Omega=(-L,L)$ with $L=100$ km, meshed into equal intervals.  The FE space $\cX_h\subset \cX$ is the set of continuous $P_1$ (piecewise-linear) functions, thus the Stokes domain $\Lambda(s)$ is polygonal.  Regarding the data, two bed profiles (Figure \ref{fig:cases}) are considered, \emph{flat} ($b_1(x)=0$) and \emph{rough} ($b_2(x)$); the latter is a linear combination of several sinusoids down to 4 km wavelength.  Three spatially-constant SMB values $a_j\in\{-2.5,0.0,1.0\}\times 10^{-7}$\, $\text{m}\,\text{s}^{-1}$ were chosen.

\begin{figure}[ht]
\centering
\includegraphics[width=0.35\textwidth]{figs/snapsflat.png} \quad \includegraphics[width=0.37\textwidth]{figs/snapsrough.png}
\caption{Two bed profiles, flat and rough, define constraint sets $\cK_h^i\subset\cX_h$, $i=1,2$, in the numerical experiment.  For each $\cK_h^i$, runs using different constant values of the SMB (see text) generated a large number of admissible surface elevations, representing a range of glacier sizes covering different portions of the domain.}
\label{fig:cases}
\end{figure}

Using definition \eqref{eq:be:admissible}, each bed elevation $b_i$, $i=1,2$, corresponds to a constraint set $\cK_h^i \subset \cX_h$.  A time-dependent, Stokes-dynamics simulation of $T=200$ years was done for each combination of set $\cK_h^i$ and value $a_j$.  One thousand sample pairs were taken from the three runs with a given $\cK_h^i$, at random times in $[0,T]$, with samples coming equally from different $a_j$ runs.  The different $a_j$ values caused advance or retreat of the ice margins, and indeed for $a_1<0$ the glacier disappears.  Thus the computed ratios came from both similar and very-different surface elevation pairs.  The initial surface for these runs is a Halfar profile\footnote{For the $b_1=a_2=0$ case the exact time-dependent surface elevation solution \emph{under shallow dynamics} would be known.  Though Stokes dynamics was used in the numerical experiment, the computed surfaces indeed agree closely with this shallow exact solution; compare \cite{LofgrenAhlkronaHelanow2022}.} \cite{Halfar1981} (Figure \ref{fig:cases}, dotted).

In these simulations each time-step was semi-implicit, approximating VI \eqref{eq:be:vi} by using $\bu|_{s^{n-1}}\cdot\bn_{s^n}$, and the free-surface stabilization technique from \cite{LofgrenAhlkronaHelanow2022} was applied to lengthen the stable time steps.  (The resulting time-dependent numerical method is only conditionally-stable.)  The VI problems at each step were solved using a reduced-space Newton method with line search \cite{BensonMunson2006}.  The Stokes problems \eqref{eq:glenstokes:weak}, with viscosity regularization $\mu_0=10^{-19}\, \text{s}^{-2}$ in \eqref{eq:glen}, were solved on the domains $\Lambda(s^{n})$ using a vertically-extruded mesh of quadrilaterals (Figure \ref{fig:fe:domain}), a $Q_2\times Q_1$ Taylor-Hood stable method \cite{Elmanetal2014}, and a Newton solver, with direct solution of the linear Newton step equations.

The essential result of these experiments is shown in Figure \ref{fig:regratios}, with histograms of sample ratios $\rho^{0.1}(\sigma,s)$ on a logarithmic axis.  These are from the highest spatial resolution ($\Delta x=500$ m and 40 elements in each column).  More than $99\%$ of the ratios are positive for the flat bed $b_1$, and $100\%$ are positive for the more-generic rough bed $b_2$.

\begin{figure}[ht]
\centering
\includegraphics[width=0.40\textwidth]{figs/bflat500mREG.png} \, \includegraphics[width=0.40\textwidth]{figs/brough500mREG.png}
\caption{Histograms of ratios $\rho^{0.1}(\sigma,s)$ for 1000 sample pairs $\sigma,s$ from the flat-bed (left) and rough-bed (right) cases.}
\label{fig:regratios}
\end{figure}

For the rough-bed experiment a coercivity constant $\alpha=10^{-22}$ is credible.  Note that dimensional values around $10^{-20}$ are expected because $\|s\|_{\cX}\sim 10^{5}$ is typical in $\cX = W^{1,4}(\Omega)$, using the norm scaling \eqref{eq:norm:Omega}, while residuals in the numerator of \eqref{eq:Phiratio} have typical magnitudes in $(10^{-5},1)$.

An underlying reason for this numerical evidence of coercivity is that \emph{the regularization does not need to do much work}.  Figure \ref{fig:noregratios} shows the unregularized ratios from the same sample pairs.  About $10\%$ of ratios are negative, which violates coercivity, but their absolute values are about two orders of magnitude smaller than the positive ratios.

\begin{figure}[ht]
\centering
\includegraphics[width=0.40\textwidth]{figs/bflat500mNOREG.png} \, \includegraphics[width=0.40\textwidth]{figs/brough500mNOREG.png}
\caption{Unregularized ratios $|\rho^0(\sigma,s)|$ for the same sample pairs as in Figure \ref{fig:regratios}.}
\label{fig:noregratios}
\end{figure}

Glacier evolution under Stokes dynamics seemingly ``wants'' to be coercive.  In fact, the continuum problem may be more coercive than any discretization.  Figure \ref{fig:resolutionratios} shows the results from running the same rough-bed experiment as in Figure \ref{fig:regratios}, but at lower resolutions.  Poorer discretizations produced negative ratios $\rho^\eps(\sigma,s)$ in this case where finer resolution did not.  Higher marginal resolution, e.g.~via free-boundary targeted adaptive mesh refinement \cite{FochesattoBueler2025}, will likely improve numerical coercivity.

\begin{figure}[ht]
\mbox{\includegraphics[width=0.31\textwidth]{figs/brough2000mREG.png} \, \includegraphics[width=0.31\textwidth]{figs/brough1000mREG.png} \, \includegraphics[width=0.31\textwidth]{figs/brough500mREG.png}}
\caption{Resolution effects: Ratios $|\rho^{0.1}(\sigma,s)|$ for horizontal mesh resolutions $\Delta x=2000$ m (left), $1000$ m (middle), and $500$ m (right; same as Figure \ref{fig:regratios} right).}
\label{fig:resolutionratios}
\end{figure}

% flat: 2.576e-07
% rough: 2.230e-07
Regarding Conjecture \ref{conj:lipschitz}, namely Lipschitz continuity for the surface velocity trace, the maximum of the ratios $\big\|\bu|_\sigma - \bu|_s\big\|_{L^{4/3}}\,\|\sigma-s\|_{\cX}^{-1}$ was $2.6\times 10^{-7}$ for the sample pairs from Figure \ref{fig:regratios}. 

In summary we have substantial, but necessarily very incomplete, evidence for Conjecture \ref{conj:regcoercive}.  This Conjecture, which implies well-posedness of the regularized implicit time-step VI problem \eqref{eq:regularizedvi}, is critical to our glaciers application of the new \emph{a priori} error bound for VIs, namely Theorem \ref{thm:abstractestimate}.

\end{document}
