\documentclass[12pt]{amsart}
%prepared in AMSLaTeX, under LaTeX2e
\addtolength{\oddsidemargin}{-.6in}
\addtolength{\evensidemargin}{-.6in}
\addtolength{\topmargin}{-0.6in}
\addtolength{\textwidth}{1.1in}
\addtolength{\textheight}{1.3in}

\renewcommand{\baselinestretch}{1.1}

\usepackage{verbatim} % for "comment" environment

\newcommand{\mtt}{\texttt}
\usepackage{alltt,xspace}

\usepackage[final]{graphicx}

\usepackage[pdftex, colorlinks=true, plainpages=false, linkcolor=black, citecolor=red, urlcolor=red]{hyperref}

% macros
\usepackage{amssymb}
\newcommand{\bA}{\mathbf{A}}
\newcommand{\bB}{\mathbf{B}}
\newcommand{\bE}{\mathbf{E}}
\newcommand{\bF}{\mathbf{F}}
\newcommand{\bJ}{\mathbf{J}}
\newcommand{\br}{\mathbf{r}}
\newcommand{\bx}{\mathbf{x}}
\newcommand{\hbi}{\mathbf{\hat i}}
\newcommand{\hbj}{\mathbf{\hat j}}
\newcommand{\hbk}{\mathbf{\hat k}}
\newcommand{\hbn}{\mathbf{\hat n}}
\newcommand{\hbr}{\mathbf{\hat r}}
\newcommand{\hbt}{\mathbf{\hat t}}
\newcommand{\hbx}{\mathbf{\hat x}}
\newcommand{\hby}{\mathbf{\hat y}}
\newcommand{\hbz}{\mathbf{\hat z}}
\newcommand{\hbphi}{\mathbf{\hat \phi}}
\newcommand{\hbtheta}{\mathbf{\hat \theta}}
\newcommand{\complex}{\mathbb{C}}
\newcommand{\ppr}[1]{\frac{\partial #1}{\partial r}}
\newcommand{\ppt}[1]{\frac{\partial #1}{\partial t}}
\newcommand{\ppx}[1]{\frac{\partial #1}{\partial x}}
\newcommand{\ppy}[1]{\frac{\partial #1}{\partial y}}
\newcommand{\ppz}[1]{\frac{\partial #1}{\partial z}}
\newcommand{\pptheta}[1]{\frac{\partial #1}{\partial \theta}}
\newcommand{\ppphi}[1]{\frac{\partial #1}{\partial \phi}}
\newcommand{\pp}[2]{\frac{\partial #1}{\partial #2}}
\newcommand{\ppp}[2]{\frac{\partial^2 #1}{\partial^2 #2}}
\newcommand{\pppp}[3]{\frac{\partial^2 #1}{\partial #2 \partial #3}}
\newcommand{\Div}{\ensuremath{\nabla\cdot}}
\newcommand{\Curl}{\ensuremath{\nabla\times}}
\newcommand{\curl}[3]{\ensuremath{\begin{vmatrix} \hbi & \hbj & \hbk \\ \partial_x & \partial_y & \partial_z \\ #1 & #2 & #3 \end{vmatrix}}}
\newcommand{\cross}[6]{\ensuremath{\begin{vmatrix} \hbi & \hbj & \hbk \\ #1 & #2 & #3 \\ #4 & #5 & #6 \end{vmatrix}}}
\newcommand{\eps}{\epsilon}
\newcommand{\grad}{\nabla}
\newcommand{\image}{\operatorname{im}}
\newcommand{\integers}{\mathbb{Z}}
\newcommand{\ip}[2]{\ensuremath{\left<#1,#2\right>}}
\newcommand{\lam}{\lambda}
\newcommand{\lap}{\triangle}
\newcommand{\Matlab}{\textsc{Matlab}\xspace}
\newcommand{\exers}[1]{\bigskip\noindent\textbf{Exercises} #1}
\newcommand{\fexer}[2]{\bigskip\noindent\textbf{Lesson #1, \##2}\quad }
\newcommand{\prob}[1]{\bigskip\noindent\textbf{#1} }
\newcommand{\pts}[1]{(\emph{#1 pts}) }
\newcommand{\epart}[1]{\medskip\noindent\textbf{(#1)}\quad }
\newcommand{\ppart}[1]{\,\textbf{(#1)}\quad }
\newcommand{\note}[1]{[\scriptsize #1 \normalsize]}
\newcommand{\MatIN}[1]{\mtt{>> #1}}
\newcommand{\onull}{\operatorname{null}}
\newcommand{\rank}{\operatorname{rank}}
\newcommand{\range}{\operatorname{range}}
\renewcommand{\P}{\mathcal{P}}
\newcommand{\real}{\mathbb{R}}
\newcommand{\trace}{\operatorname{tr}}
\renewcommand{\Re}{\operatorname{Re}}
\renewcommand{\Im}{\operatorname{Im}}
\newcommand{\Arg}{\operatorname{Arg}}

\newcommand{\pl}[2]{\item \emph{page #1, line #2}:\, }
\newcommand{\pf}[2]{\item \emph{page #1, #2}:\, }



\begin{document}
\graphicspath{{../}}

% note sighorizblack.pdf is derived from Math_Stats_sig_horiz_black.pdf by using pdfcrop
\noindent \includegraphics[width=4.5in]{sighorizblack}

\thispagestyle{empty}
\vspace{0.2in}

\hfill \today
\medskip
\bigskip

\noindent To the Editor of ESAIM: Mathematical Modelling and Numerical Analysis --

\bigskip

\noindent  Please consider this manuscript for publication in M2AN.  This is an original submission, with a preprint posted at \href{https://arxiv.org/abs/2408.06470}{arxiv.org/abs/2408.06470}.

\medskip
\noindent  This work contains new results both in modeling \emph{and} numerical analysis.  The most original theorem is actually Theorem 5.1 (and its corollaries); it is a new and general finite element quasi-optimality result for variational inequalities (VIs), allowing a nonlinear operator.  It has a glacier-specific corollary Theorem 6.1.  On the other hand, the most original and significant \emph{content} is actually the weak-form mathematical model for glacier evolution using Stokes non-shallow dynamics, namely VI (45), (46).  Although I make rigorous progress on the new model, well-posedness is based on conjectural elements, which I make very clear.

\medskip
\noindent  Researchers with an interest in glaciers and ice sheets, plus those who study other VI and/or Stokes models, are in the target audience.  The VI here is different from most literature because its operator requires the solution of a separate PDE problem; it requires the surface trace of a non-Newtonian Stokes solution.  The manuscript makes this novel structure as clear as possible.

\medskip
\noindent  There are relatively few mathematical glaciologists out there, but here are some referee suggestions, listed alphabetically, supposing suggestions are appropriate, and asking forgiveness if not:
\begin{itemize}
\item[] Josefin Ahlkrona, Stockholm University, Sweden, \href{mailto:ahlkrona@math.su.se}{ahlkrona@math.su.se}
\item[] Ian Hewitt, Oxford University, UK, \href{mailto:hewitt@maths.ox.ac.uk}{hewitt@maths.ox.ac.uk}
\item[] Guillaume Jouvet, University of Lausanne, Switzerland, \href{mailto:guillaume.jouvet@unil.ch}{guillaume.jouvet@unil.ch}
\item[] Christian Schoof, University of British Columbia, Canada, \href{mailto:cschoof@eoas.ubc.ca}{cschoof@eoas.ubc.ca}
\item[] Dan Shapero, University of Washington, \href{mailto:shapero@uw.edu}{shapero@uw.edu}
\end{itemize}
Only one of these, Jouvet, is a former co-author.

\medskip
\noindent  Any mistakes in the writing of the R\'esum\'e are those of google translate.  I apologize that I do not speak or write French.

\bigskip

\noindent Best regards, and very appreciatively,

\medskip

%\vspace{0.35in}
\includegraphics[height=0.45in]{elb_sign}

\small

\noindent Ed Bueler, Professor of Mathematics (Applied)

\noindent University of Alaska, Fairbanks, USA 99775-6660 \hfill \texttt{elbueler\@@alaska.edu}

\end{document}
