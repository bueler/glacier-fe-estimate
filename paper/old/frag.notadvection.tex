
\subsection{The problem is not of advection type} \label{subsec:notadv}  Based on the appearance of SKE \eqref{eq:ske}, often written $\frac{\partial s}{\partial t} + u \frac{\partial s}{\partial x} + v \frac{\partial s}{\partial y} = a + w$, where $(u,v,w)$ denotes the surface velocity \cite{GreveBlatter2009,SchoofHewitt2013}, it is common in the literature to regard this equation as an advection.\footnote{References \cite{Chengetal2020,WirbelJarosch2020} are examples where ``advection'' is specifically stated, but the idea is pervasive among model descriptions.  Once a numerical model framework views this equation as an advection, or if the dependence of surface velocity on glacier geometry is linearized away \cite[section 3.5, for example]{Durandetal2009}, the scheme inevitably ceases to be meaningfully implicit.  The stabilizing effect of gravity, which acts through the vertical velocity at the ice surface, gets delayed till the next time step, with corresponding undesirable consequences for numerical stability.}  However, this is far from the whole truth.  Mathematically, it is not an advection because the surface velocity is not determined externally, but instead through coupled stress balance equations over the domain determined by the surface elevation.  That is, $s$ is found \emph{simultaneously} with the ``advecting'' velocity $\bu|_s$.  Physically, this free-surface, viscous flow is gravity-driven, and ice flows predominantly downhill.  The surface therefore typically responds to perturbations with negative feedback; flow tends to remove raised surface bump and fill-in surface indentations.  This response is diffusive in the large.  This diffusive character also corresponds to the infinite speed of propagation of boundary information in a Stokes model.\footnote{Note that for glaciers the role of inertia is many orders of magnitude smaller than viscous stresses.  The Froude number is typically $10^{-15}$ \cite{GreveBlatter2009}.}  Such a diffusive response explains the relatively smooth large-scale appearance of actual surface elevations (Figure \ref{fig:giscross}).

As is well known, the shallow ice approximation (SIA) is a diffusion equation.  In fact, for the non-sliding and isothermal SIA model \cite{GreveBlatter2009,JouvetBueler2012}, SKE \eqref{eq:ske} becomes
\begin{equation}
\frac{\partial s}{\partial t} - \Gamma (s-b)^{\nn+1} |\grad s|^{\nn+1} - \Div \left(\frac{\nn+1}{\nn+2} \Gamma (s-b)^{\nn+1} |\grad s|^{\nn-1} \grad s\right) - a = 0  \label{eq:ske:sia}
\end{equation}
% \Gamma = (2/(n+1)) A (\rho g)^\nn
Here $\nn\approx 3$ is Glen's exponent \cite{GreveBlatter2009} and $\Gamma>0$ is a constant (equivalent to $\nu_\pp$ in \eqref{eq:glen}).  The divergence term in \eqref{eq:ske:sia} arises from the vertical velocity term in the SKE.  Now, partial well-posedness results are known for SIA models, usually parameterized using the ice thickness.  With $H=s-b \ge 0$, existence holds for the steady state of \eqref{eq:ske:sia} \cite{JouvetBueler2012}.  Appendix \ref{app:noncoercive} includes the apparently-new observation that the steady-state of \eqref{eq:ske:sia} does \emph{not} have a unique solution when $a=0$.  In time-dependent cases both existence and uniqueness hold in the easier case when the bedrock is flat \cite{Calvoetal2003,PiersantiTemam2023}.  Furthermore, scalable and implicit FE methods for the VI problem behind \eqref{eq:ske:sia} are available; see \cite{Bueler2016} and \cite[Example 8.4]{BuelerFarrell2024}.  However, the regularity and smoothness exhibited by solutions to \eqref{eq:ske:sia} probably does not persist for solutions to the Stokes-based SKE \eqref{eq:ske} because the latter model is known to have a significantly different small-wavelength limit relative to the SIA \cite{Pattynetal2008}.

In addition to not being an advection, VI problem \eqref{eq:be:vi} is also not of optimization type.  This is directly clear in SIA model \eqref{eq:ske:sia}, where the problem has porous medium character, that is, the diffusivity scales with a power of the ice thickness.  To illustrate the essential reasoning, it can be shown that the simplest elliptic, quasilinear, and steady porous medium equation $(u(x) u'(x))' = f(x)$ does not have the symmetry of an optimization problem, that is, there is no objective function so that this equation is the first-order condition.  Because the flow of ice under Stokes dynamics scales in some manner with the ice thickness, and thin ice which is frozen to the bed has low velocity regardless of surface slope. we likewise expect that  \eqref{eq:be:vi}, or any similar glacier problem, is not the first-order condition of an inequality-constrained minimization.
