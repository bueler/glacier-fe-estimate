Regarding the specific Lipschitz statement \eqref{eq:philipschitz}, suppose we knew instead that $\Big|\Phi(r)[q] - \Phi(s)[q]\Big| \le C(R)\, (\|r-s\|_{\cX})^\omega \|q\|_{\cX}$ for some exponent $\omega>0$.  This would provide sufficient continuity for the well-posedness Theorem \ref{thm:stepwellposed} below.  However, the finite element error theorem in Section \ref{sec:abstractestimate} needs \eqref{eq:philipschitz} as stated with $\omega=1$.

If the horizontal components of the surface velocity are differentiable then one might revise operator definition \eqref{eq:be:Fdefine} as follows.  Write $\bu=(u,v,w)$ in cartesian coordinates, and define $\bU=(u,v)$.  Assuming $\bU|_s=\bzero$ along the fixed boundary $\partial\Omega$, integrate \eqref{eq:be:Fdefine} by parts to give
\begin{equation}
F_{\Delta t}(s)[q] = \int_\Omega \left(s - \Delta t\, w|_s\right) q - \Delta t\, \Div\left(\bU|_s\, q\right).  \label{eq:be:Fdefine:alt}
\end{equation}
However, form \eqref{eq:be:Fdefine:alt} seems not to represent a good regularity trade-off between $\bu|_s$ and $s$.  We have proven in Section \ref{sec:stokes} that $\bu|_s$ is an $L^\pp$ function over $\Gamma_s$ (Corollary \ref{cor:surfacetracebound}), but we have no proof that it is more regular than that.  On the other hand, we are indeed hypothesizing that $\grad s$ is a well-defined function in $L^\rr$, because $s\in\cX=W^{1,\rr}(\Omega)$.  Thus we will keep definition \eqref{eq:be:Fdefine}.  Observe that \eqref{eq:be:Fdefine:alt} looks more like the divergence-form operators written for thickness-based models, e.g.~\cite{Bueler2021conservation,JouvetBueler2012}.
